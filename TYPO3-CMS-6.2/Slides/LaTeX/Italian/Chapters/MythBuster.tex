% ------------------------------------------------------------------------------
% TYPO3 CMS 6.2 LTS - What's New - Chapter "MythBuster" (Italian Version)
%
% @author Roberto Torresani <roberto.torresani@typo3.org>
% @license	Creative Commons BY-NC-SA 3.0
% @link		http://typo3.org/download/release-notes/whats-new/
% @language	Italian
% ------------------------------------------------------------------------------
% Chapter: MythBuster
% ------------------------------------------------------------------------------

\section{MythBuster}
\begin{frame}[fragile]
	\frametitle{Miti da sfatare}

	\begin{center}\huge{Capitolo 10:}\end{center}
	\begin{center}\huge{\color{typo3darkgrey}\textbf{TYPO3 CMS 6.2 LTS - Miti da sfatare}}\end{center}

\end{frame}

% ------------------------------------------------------------------------------
% MythBuster
% ------------------------------------------------------------------------------

\begin{frame}[fragile]
	\frametitle{Miti da sfatare}
	\framesubtitle{Miti su TYPO3 CMS 6.2}

	\begin{itemize}
		\item TYPO3 CMS 6.2 LTS sarà l'ultimo rilascio di TYPO3 CMS
			\tabto{9cm}\color{red}\textbf{\textrightarrow Falso!}\color{black}

			\smaller
				La verità è che nonostante il rilascio di \href{http://neos.typo3.org}{TYPO3 Neos}, lo sviluppo di TYPO3 CMS continuerà e vedremo nuovi rilasci in futuro.
			\normalsize

		\item Il core di TYPO3 è stato completamente riscritto nella 6.x
			\tabto{9cm}\color{red}\textbf{\textrightarrow Falso!}\color{black}

			\smaller
				La verità è che è stato introdotto il concetto di namespace PHP con TYPO3 CMS 6.0, che significa nuovi nomi per le classi. Tuttavia, un layer di compatibilità garantisce agli sviluppatori la possibilità di utilizzare i vecchi nomi di classe nelle loro estensioni.
			\normalsize

		\item Le estensioni fatte per la 4.5 non funzioneranno nella 6.2
			\tabto{9cm}\color{red}\textbf{\textrightarrow Falso!}\color{black}

			\smaller
				La verità è che le API del core non sono cambiate completamente e le funzionalità mantengono una retro-compatibilità, in accordo con la \href{http://forge.typo3.org/projects/typo3v4-core/wiki/CoreDevPolicy}{strategia di deprecazione}. Il core di TYPO3 CMS 6.2 permette di utilizzare il maggior numero di estensioni scritte per la 4.5 senza, o con poche, modifiche.
			\normalsize

	\end{itemize}

\end{frame}

% ------------------------------------------------------------------------------
% MythBuster
% ------------------------------------------------------------------------------

\begin{frame}[fragile]
	\frametitle{Miti da sfatare}
	\framesubtitle{Miti su TYPO3 CMS 6.2}

	\begin{itemize}
		\item TemplaVoila non può più essere usato con TYPO3 6.2
			\tabto{9cm}\color{red}\textbf{\textrightarrow Falso!}\color{black}

			\smaller
				La verità è che la comunità sta lavorando per una versione compatibile, che permetterà a Templavoila di essere utilizzato in TYPO3 CMS 6.2. Tuttavia, Templavoila non sarà ulteriormente sviluppato e gli utenti sono caldamente incoraggiati a cercare alternative per sviluppi futuri.
			\normalsize

		\item Le estensioni basate su \texttt{tslib\_pibase} non funzionano
			\tabto{9cm}\color{red}\textbf{\textrightarrow Falso!}\color{black}

			\smaller
				La verità è che \texttt{tslib\_pibase} esiste ancora nella 6.2, ma ha un nuovo nome per la convenzione dei namespace: \texttt{\textbackslash TYPO3\textbackslash CMS\textbackslash Frontend\textbackslash Plugin\textbackslash AbstractPlugin}.\newline
				Una classe alias garantisce il funzionamento al vecchio nome (layer di compatibilità).
			\normalsize

		\item Non c'è modo per migrare i record DAM al FAL della 6.2
			\tabto{9cm}\color{red}\textbf{\textrightarrow Falso!}\color{black}

			\smaller
				Di fatto DAM non funziona con TYPO3 6.x. Tuttavia, il FAL vuole fornire un API che permette di ricreare ciò che era possibile fare con il DAMC'e anche disponibile un \href{https://github.com/fnagel/t3ext-dam_falmigration}{progetto di migrazione da DAM a FAL}.
			\normalsize

	\end{itemize}

\end{frame}

% ------------------------------------------------------------------------------
% MythBuster
% ------------------------------------------------------------------------------

\begin{frame}[fragile]
	\frametitle{Miti da sfatare}
	\framesubtitle{Miti su TYPO3 CMS 6.2}

	\begin{itemize}
		\item Si può passare dalla 4.5 alla 6.2 con un aggiornamento automatico
			\tabto{9cm}\color{red}\textbf{\textrightarrow Falso!}\color{black}

			\smaller
				Si dice che il progetto "Smooth Migration" fornisce un grande aggiornamento automatico da TYPO3 4.5 alla 6.2. La verità è che il progetto mira a fornire informazioni, documentazione, individuare le incompatibililtà, ecc. per aiutare gli utenti nel processo di migrazione.
			\normalsize

		\item TYPO3 6.2 richiede un hardware migliore
			\tabto{9cm}\color{red}\textbf{\textrightarrow Falso!}\color{black}
			% \tabto{8.2cm}\color{red}\textbf{\textrightarrow partly true :-)}\color{black}

			\smaller
				Si dice che la 6.2 sia 10 volte più lenta della 4.5. La verità è che in molti casi le prestazioni sono simili alle versioni precedenti. I \href{http://typo3.org/about/typo3-the-cms/system-requirements/}{requisiti minimi} per TYPO3 CMS non sono cambiati. Tuttavia, a causa dei cambiamenti architetturali e l'uso di tecnologie moderne, gli amministratori di sistema dovrebbero prendere in considerazione un aggiornamento dell'hardware (da considerare che TYPO3 4.5 è stato rilasciato nel gennaio 2011, più di 3 anni fa).
			\normalsize

	\end{itemize}

\end{frame}

% ------------------------------------------------------------------------------

