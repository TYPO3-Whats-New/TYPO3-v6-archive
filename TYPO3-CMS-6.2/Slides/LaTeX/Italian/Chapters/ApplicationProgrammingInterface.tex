% ------------------------------------------------------------------------------
% TYPO3 CMS 6.2 LTS - What's New - Chapter "Application Programming Interface" (Italian Version)
%
% @author Roberto Torresani <roberto.torresani@typo3.org>
% @license	Creative Commons BY-NC-SA 3.0
% @link		http://typo3.org/download/release-notes/whats-new/
% @language	Italian
% ------------------------------------------------------------------------------
% Chapter: Application Programming Interface
% ------------------------------------------------------------------------------

\section{Interfaccia di programmazione dell'applicazione}
\begin{frame}[fragile]
	\frametitle{Interfaccia di programmazione dell'applicazione}

	\begin{center}\huge{Capitolo 7:}\end{center}
	\begin{center}\huge{\color{typo3darkgrey}\textbf{Interfaccia di programmazione dell'applicazione (API)}}\end{center}

\end{frame}

% ------------------------------------------------------------------------------
% Hook: tsfe::checkEnableFields
% ------------------------------------------------------------------------------
% http://forge.typo3.org/issues/48981

\begin{frame}[fragile]
	\frametitle{Interfaccia di programmazione dell'applicazione}
	\framesubtitle{Hook: \texttt{tsfe::checkEnableFields}}

	\begin{itemize}
		\item In TYPO3 < 6.2, "\emph{estendere alle sottopagine}" non poteva essere utilizzato nelle estensioni che forniscono regole aggiuntive per la visibilità delle pagine\newline
			\small(l'elenco dei campi da scegliere è inserita nel core\texttt{tsfe::checkEnableFields()})\normalsize

		\item In TYPO3 >= 6.2, un nuovo hook permette alle estensioni di fornire delle regole aggiuntive per la visibilità delle pagine quando la pagina madre ha selezionato "estendere alle sottopagine".
		\item Classe:\newline
			\smaller
				\texttt{\textbackslash
					TYPO3\textbackslash
					CMS\textbackslash
					Frontend\textbackslash
					Controller\textbackslash
					TypoScriptFrontendController}\normalsize

			\lstset{
				basicstyle=\smaller\ttfamily
			}

			\begin{lstlisting}
				$GLOBALS['TYPO3_CONF_VARS']['SC_OPTIONS']
				  ['tslib/class.tslib_fe.php']['hook_checkEnableFields']
			\end{lstlisting}

	\end{itemize}

\end{frame}

% ------------------------------------------------------------------------------
% Hook: checkFlexFormValue in DataHandler
% ------------------------------------------------------------------------------
% http://forge.typo3.org/issues/49699

\begin{frame}[fragile]
	\frametitle{Interfaccia di programmazione dell'applicazione}
	\framesubtitle{Hook: \texttt{checkFlexFormValue} in DataHandler}

	\begin{itemize}
		\item In TYPO3 < 6.2, quando si aggiorna un valore di un Flexform, non c'era nessun controllo se un valore esistente nel database veniva cancellato.
		\item Quest diventava un problema, ad esempio durante il salvataggio delle azioni dei controller (Extbase) nei Flexform: le vecchie azioni che non devono più essere presenti, vanno rimosse manualmente

		\item In TYPO3 >= 6.2, un nuovo hook permette di aggiustare i vecchi dati del Flexform prima di essere unito a quello nuovo.
		\item Classe:\newline
			\smaller
				\texttt{\textbackslash
					TYPO3\textbackslash
					CMS\textbackslash
					Core\textbackslash
					DataHandling\textbackslash
					DataHandler}\normalsize

			\lstset{
				basicstyle=\smaller\ttfamily
			}

			\begin{lstlisting}
				$GLOBALS['TYPO3_CONF_VARS']['SC_OPTIONS']
				  ['t3lib/class.t3lib_tcemain.php']['checkFlexFormValue']
			\end{lstlisting}

		\item Metodo:\newline
			\smaller
				\texttt{checkFlexFormValue\_beforeMerge()}

	\end{itemize}

\end{frame}

% ------------------------------------------------------------------------------
% Hook to customize header in module "Web > Page"
% ------------------------------------------------------------------------------
% http://forge.typo3.org/issues/52579

\begin{frame}[fragile]
	\frametitle{Interfaccia di programmazione dell'applicazione}
	\framesubtitle{Hook per personalizzare le intestazioni}

	\begin{itemize}
		\item In TYPO3 >= 6.2, un nuovo hook permette di modificare le intestazioni delle pagine nel modulo di pagina (Module: "Web > Page")
		\item Questo hook è chiamato prima che il contenuto della pagina sia renderizzato
		\item Classe:\newline
			\smaller
				\texttt{\textbackslash
					TYPO3\textbackslash
					CMS\textbackslash
					Backend\textbackslash
					Controller\textbackslash
					PageLayoutController}\normalsize

			\lstset{
				basicstyle=\smaller\ttfamily
			}

			\begin{lstlisting}
				$GLOBALS['TYPO3_CONF_VARS']['SC_OPTIONS']
				  ['cms/layout/db_layout.php']['drawHeaderHook']
			\end{lstlisting}

		\item Metodo:\newline
			\smaller
				\texttt{callUserFunction()}

	\end{itemize}

\end{frame}

% ------------------------------------------------------------------------------
% IRRE: Provide default values for created records
% ------------------------------------------------------------------------------
% http://forge.typo3.org/issues/46124

\begin{frame}[fragile]
	\frametitle{Interfaccia di programmazione dell'applicazione}
	\framesubtitle{IRRE: valori di default per i record creati}

	\begin{itemize}
		\item Una nuova opzione TCA permette di configurare i campi "inline"
		\item La chiave \texttt{foreign\_record\_defaults} permette di impostare (default) il valore dei nuovi record creati

			\begin{lstlisting}
				'config' => array(
				  'type' => 'inline',
				  'foreign_table' => 'tt_content',
				  'foreign_record_defaults' => array(
				    'CType' => 'image'
				  ),
				)
			\end{lstlisting}

			\small
				Esempio sopra: l'elemento \texttt{tt\_content} che viene creato in questo campo IRRE sarà di default di tipo \textbf{contenuto di tipo immagine}. L'editore può impostare un altro tipo prima di salvare.
			\normalsize

	\end{itemize}

\end{frame}

% ------------------------------------------------------------------------------
% Workspaces
% ------------------------------------------------------------------------------
% http://forge.typo3.org/issues/46124

\begin{frame}[fragile]
	\frametitle{Interfaccia di programmazione dell'applicazione}
	\framesubtitle{Workspaces (1)}

	\begin{itemize}
		\item In TYPO3 < 6.2, il modulo "Workspaces" può essere esteso solo sovrascrivendo i componenti PHP e JavaScript.
		\item In TYPO3 >= 6.2, è possibile estendere la definizione e il comportamente delle colonne visualizzate nel modulo.
		\item Qualche esempio nelle slide seguenti...
	\end{itemize}

\end{frame}

% ------------------------------------------------------------------------------
% Workspaces
% ------------------------------------------------------------------------------
% http://forge.typo3.org/issues/46124

\begin{frame}[fragile]
	\frametitle{Interfaccia di programmazione dell'applicazione}
	\framesubtitle{Workspaces (2)}

	\lstset{
		basicstyle=\tiny\ttfamily
	}

	Esempio (file \texttt{ext\_localconf.php}):

	\begin{lstlisting}
		$GLOBALS['TYPO3_CONF_VARS']['SC_OPTIONS']
		  ['t3lib/class.t3lib_tcemain.php']['processCmdmapClass']['workspaces_logger'] =
		  'Vendor\\WorkspacesLogger\\Hook\\DataHandlerHook';
	\end{lstlisting}

	Esempio (file \texttt{ext\_tables.php}):

	\begin{lstlisting}
		\TYPO3\CMS\Workspaces\Service\AdditionalColumnService::getInstance()->register(
		  'WorkspacesLogger_StageChange',
		  'Vendor\\WorkspacesLogger\\DataProvider'
		);

		\TYPO3\CMS\Workspaces\Service\AdditionalResourceService::getInstance()->addJavaScriptResource(
		  'WorkspacesLogger',
		  'EXT:myextension/Resources/Public/JavaScript/StageChange.js'
		);
	\end{lstlisting}

\end{frame}

% ------------------------------------------------------------------------------
% Workspaces
% ------------------------------------------------------------------------------
% http://forge.typo3.org/issues/46124

\begin{frame}[fragile]
	\frametitle{Interfaccia di programmazione dell'applicazione}
	\framesubtitle{Workspaces (3)}

	Esempio (file \texttt{Vendor\textbackslash
		WorkspacesLogger\textbackslash
		Hook\textbackslash
		DataHandlerHook}):

	\lstset{
		basicstyle=\tiny\ttfamily
	}

	\begin{lstlisting}
		<?php
		namespace Vendor\WorkspacesLogger\Hook;
		use TYPO3\CMS\Core\SingletonInterface;

		class DataHandlerHook implements SingletonInterface {

		  const TABLE_Name = 'tx_workspaceslogger_event';
		  const EVENT_SetStage = 91;

		  /**
		   * hook that is called when no prepared command was found
		   */
		  public function processCmdmap($command, $table, $id, $value, &$commandIsProcessed,
		    \TYPO3\CMS\Core\DataHandling\DataHandler $tcemainObj) {
		    ...
		    $action = (string) $value['action'];
		    if ($command === 'version' && $action === 'setStage' && $commandIsProcessed) {
		      ...
		    }
		  }
		}
	\end{lstlisting}

\end{frame}

% ------------------------------------------------------------------------------
% PSR-3 compatible Logger
% ------------------------------------------------------------------------------
% http://forge.typo3.org/issues/48880

\begin{frame}[fragile]
	\frametitle{Interfaccia di programmazione dell'applicazione}
	\framesubtitle{Strumento di log compatibile PSR-3}

	\begin{itemize}
		\item In TYPO3 CMS 6.2 le API di log sono compatibili PSR-3
		\item PSR-3 mira a stabilire uno standard dei log di PHP (standard of the PHP Framework Interop Group)

		\item L'obiettivo principale di PSR-3 è
			"\emph{permettere alle librerie di ricevere un oggetto LoggerInterface e scrivere il log in esso in un modo semplice e universale.}"

		\item L'interfaccia dei log contiene metodi sintetici di log come\newline
			\texttt{debug()}, \texttt{warning()}, \texttt{notice()}, \texttt{alert()}, \texttt{error()}, etc.

		\item Altre risorse:\newline
			\url{http://www.php-fig.org/psr/3/}

	\end{itemize}

\end{frame}

% ------------------------------------------------------------------------------
% API to CSRF protect Ajax calls in Backend
% (slide added in March 2014)
% ------------------------------------------------------------------------------
% http://forge.typo3.org/issues/56345
% http://forge.typo3.org/issues/56347 (documentation)

\begin{frame}[fragile]
	\frametitle{Interfaccia di programmazione dell'applicazione}
	\framesubtitle{Protezione chiamate Ajax CSRF}

	\lstset{
		basicstyle=\tiny\ttfamily
	}

	\begin{itemize}
		\item Le chiamate Ajax nel backend di TYPO3 possono essere protette da CSRF (\textit{cross-site request forgery}) registrando i loro "handler"

			\begin{lstlisting}
				\TYPO3\CMS\Core\Utility\ExtensionManagementUtility::registerAjaxHandler(
				  'TxMyExt::process',
				  '\Vendor\MyExt\AjaxHandler->process'
				);
			\end{lstlisting}

		\item L'URL per un ID Ajax ID contiene un token di protezione CSRF, che può essere verificato nel "dispatcher" \texttt{ajax.php}

			\begin{lstlisting}
				$ajaxUrl = \TYPO3\CMS\Core\Utility\BackendUtility::getAjaxUrl('TxMyExt::process');
			\end{lstlisting}

		\item Queste impostazioni possono essere accessibili nel contesto Javascript della pagina

			\begin{lstlisting}
				var ajaxUrl = TYPO3.settings.MyExt.ajaxUrl;
			\end{lstlisting}

	\end{itemize}

\end{frame}

% ------------------------------------------------------------------------------
% Miscellaneous
% ------------------------------------------------------------------------------
% http://forge.typo3.org/issues/49144 (MathUtility: Add canBeInterpretedAsFloat)
% http://forge.typo3.org/issues/52707 (Introduce a PHP Enumeration type)
% http://forge.typo3.org/issues/52762 (Add type converter for core types like Enumeration)

\begin{frame}[fragile]
	\frametitle{Interfaccia di programmazione dell'applicazione}
	\framesubtitle{Varie}

	\begin{itemize}
		\item Nuovo metodo \texttt{canBeInterpretedAsFloat()} nella classe: \texttt{MathUtility}\newline
			\small(Questo è analogo a: \texttt{canBeInterpretedAsInteger()})\normalsize
		\item Nuovo tipo di enumerazione (senza relazioni a moduli PHP di terze parti):\newline
			\texttt{\textbackslash
				TYPO3\textbackslash
				CMS\textbackslash
				Core\textbackslash
				Type\textbackslash
				Enumeration}\newline

			Ad esempio usato in:\newline
			\texttt{\textbackslash
				TYPO3\textbackslash
				CMS\textbackslash
				Core\textbackslash
				Versioning\textbackslash
				VersionState}\newline

			...e poi come:\newline
			\texttt{new VersionState(VersionState::DEFAULT\_STATE);}

	\end{itemize}

\end{frame}

% ------------------------------------------------------------------------------

