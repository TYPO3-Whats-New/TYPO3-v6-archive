% ------------------------------------------------------------------------------
% TYPO3 CMS 6.2 LTS - What's New - Chapter "Package Management" (Italian Version)
%
% @author Roberto Torresani <roberto.torresani@typo3.org>
% @license	Creative Commons BY-NC-SA 3.0
% @link		http://typo3.org/download/release-notes/whats-new/
% @language	Italian
% ------------------------------------------------------------------------------
% Chapter: Package Management
% ------------------------------------------------------------------------------

\section{Gestione dei pacchetti}
\begin{frame}[fragile]
	\frametitle{Gestione dei pacchetti}

	\begin{center}\huge{Capitolo 5:}\end{center}
	\begin{center}\huge{\color{typo3darkgrey}\textbf{Gestione dei pacchetti}}\end{center}

\end{frame}

% ------------------------------------------------------------------------------
% Package Manager
% ------------------------------------------------------------------------------
% http://wiki.typo3.org/Blueprints/Packagemanager
% http://forge.typo3.org/issues/47018
% http://forge.typo3.org/issues/52737

\begin{frame}[fragile]
	\frametitle{Gestione dei pacchetti}
	\framesubtitle{Manager dei pacchetti}

	\begin{itemize}
		\item Il \textbf{Manger dei pacchetti} di TYPO3 Flow è stato portato su TYPO3 CMS
		\item Lo sviluppo/implementazione è iniziata con lo sviluppo di TYPO3 CMS 6.1
		\item Questo progetto mira ad uniformare il formato dei pacchetti
		\item Le estensioni in TYPO3 CMS sono solo dei "pacchetti" di tipo speciale
		\item Gli obiettivi principali del progetto:

			\begin{itemize}
				\item API adatte per la gestione dei pacchetti
				\item Namespace del fornitore di supporto
				\item Supporto ai pacchetti con composer
				\item Supporto ai pacchetti Flow
				\item Resviluppo dell'autoloader
			\end{itemize}

	\end{itemize}

\end{frame}

% ------------------------------------------------------------------------------
% Package Manager Integration
% ------------------------------------------------------------------------------

\begin{frame}[fragile]
	\frametitle{Gestione dei pacchetti}
	\framesubtitle{Integrazione della gestione dei pacchetti}

	\begin{itemize}
		\item Rimozione di \texttt{\$TYPO3\_CONF['EXT']['extListArray']} dal file\newline
			\smaller\texttt{typo3conf/LocalConfiguration.php}\normalsize

		\item I contenuti di \small\texttt{typo3conf/LocalConfiguration.php} sono copiati in\normalsize\newline
			\smaller\texttt{typo3conf/LocalConfiguration.beforePackageStatesMigration.php}\normalsize

		\item Il file \texttt{typo3conf/PackageStates.php} contiene:

			\begin{itemize}
				\item stato del pacchetto (attivo/disattivo)
				\item posizione dell'estensione nel filesystem
			\end{itemize}

		\item Sono automaticamente individuate le estensioni nelle directory:

			\begin{itemize}
				\item \texttt{typo3/sysext/}
				\item \texttt{typo3/ext/}
				\item \texttt{typo3/contrib/}
				\item \texttt{typo3conf/ext/}
				\item \texttt{Packages/} \emph{(ricorsiva)}
			\end{itemize}

	\end{itemize}

\end{frame}

% ------------------------------------------------------------------------------
% Package Manager Integration
% ------------------------------------------------------------------------------

\begin{frame}[fragile]
	\frametitle{Gestione dei pacchetti}
	\framesubtitle{Integrazione della gestione dei pacchetti}

	\begin{itemize}

		\item Due nuovi (aggiuntivi) file nella directory delle estensioni:

			\begin{itemize}
				\item \texttt{composer.json}
				\item \texttt{Classes/Package.php}
			\end{itemize}

		\item Se l'estensione lo richiede, la proprietà \texttt{protected \$protected}\newline
			può essere impostata in \texttt{composer.json}

		\item Se il file \texttt{PackageStates.php} manca, esso viene (ri)creato,\newline
			contenente tutte le estensioni, che hanno la proprietà sopra impostata a \texttt{TRUE}

		\item Autoloader gestisce la sua cache di backend

		\item Maggiori informazioni:\newline
			\url{http://wiki.typo3.org/Blueprints/Packagemanager}

	\end{itemize}

\end{frame}

% ------------------------------------------------------------------------------
% Examples
% ------------------------------------------------------------------------------

\begin{frame}[fragile]
	\frametitle{Gestione dei pacchetti}
	\framesubtitle{Integrazione della gestione dei pacchetti}

	Esempio: \texttt{typo3conf/PackageManager.php}

	\lstset{
		basicstyle=\tiny\ttfamily
		% basicstyle=\fontsize{5}{7}\selectfont\ttfamily
	}

	\begin{lstlisting}
		return array ('packages' =>
		    array (
		      'core' =>
		        array (
		          'manifestPath' => '',
		          'composerName' => 'typo3/cms/core',
		          'state' => 'active',
		          'packagePath' => 'typo3/sysext/core/',
		          'classesPath' => 'Classes/',
		        ),
		      'workspaces' =>
		        array (
		          'manifestPath' => '',
		          'composerName' => 'typo3/cms/workspaces',
		          'state' => 'inactive',
		          'packagePath' => 'typo3/sysext/workspaces/',
		          'classesPath' => 'Classes/',
		        ),
		      ...
		    ),
		    'version' => 4,
		);
	\end{lstlisting}

\end{frame}

% ------------------------------------------------------------------------------
% Examples
% ------------------------------------------------------------------------------

\begin{frame}[fragile]
	\frametitle{Gestione dei pacchetti}
	\framesubtitle{Integrazione della gestione dei pacchetti}

	Esempio: \texttt{composer.json}

	\lstset{
		basicstyle=\tiny\ttfamily
	}

        \begin{lstlisting}
                {
                  "name": "typo3/cms-indexed-search",
                  "type": "typo3-cms-framework",
                  "description": "TYPO3 Core",
                  "homepage": "http://typo3.org",
                  "license": ["GPL-2.0+"],
                  "version": "6.2.0",
                  "require": {
                    "typo3/cms-core": "*"
                  },
                  "replace": {
                    "indexed_search": "*"
                  }
                }
        \end{lstlisting}

\end{frame}

% ------------------------------------------------------------------------------
% Miscellaneous
% (slide added in March 2014)
% ------------------------------------------------------------------------------

\begin{frame}[fragile]
	\frametitle{Gestione dei pacchetti}
	\framesubtitle{Integrazione della gestione dei pacchetti}

	\lstset{
		basicstyle=\smaller\ttfamily
	}

	\begin{itemize}
		\item I pacchetti possono essere attivati anche in fase di esecuzione utilizzando la chiave:
		\begin{lstlisting}
			$GLOBALS['TYPO3_CONF_VARS']['EXT']['runtimeActivatedPackages'] =
			  array( '{packageKey}' );
		\end{lstlisting}

		\item Questa chiave è attivata immediamente dopo l'inizializzazione della gestione dei pacchetti

	\end{itemize}

\end{frame}

% ------------------------------------------------------------------------------

