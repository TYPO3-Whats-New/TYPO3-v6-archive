% ------------------------------------------------------------------------------
% TYPO3 CMS 6.2 LTS - What's New - Chapter "Extbase & Fluid" (Italian Version)
%
% @author Roberto Torresani <roberto.torresani@typo3.org>
% @license	Creative Commons BY-NC-SA 3.0
% @link		http://typo3.org/download/release-notes/whats-new/
% @language	Italian
% ------------------------------------------------------------------------------
% Chapter: Extbase & Fluid
% ------------------------------------------------------------------------------

\section{Extbase \& Fluid}
\begin{frame}[fragile]
	\frametitle{Extbase \& Fluid}

	\begin{center}\huge{Capitolo 8:}\end{center}
	\begin{center}\huge{\color{typo3darkgrey}\textbf{Extbase \& Fluid}}\end{center}

\end{frame}

% ------------------------------------------------------------------------------
% ObjectManager->getScope()
% ------------------------------------------------------------------------------
% http://forge.typo3.org/issues/48766

\begin{frame}[fragile]
	\frametitle{Extbase \& Fluid}
	\framesubtitle{ObjectManager->getScope()}

	\lstset{
		basicstyle=\tiny\ttfamily
	}

	\begin{itemize}
		\item Il metodo \texttt{ObjectManager->getScope()} determina,\newline
			se una classe è di tipo \textbf{prototype} o \textbf{singleton}

			\begin{lstlisting}
				/**
				 * @var \TYPO3\CMS\Extbase\Object\ObjectManagerInterface
				 * @inject
				 */
				protected $objectManager;

				$this->objectManager->getScope($propertyTargetClassName) === \TYPO3\CMS
				\Extbase\Object\Container\Container::SCOPE_PROTOTYPE

				$this->objectManager->getScope($propertyTargetClassName) === \TYPO3\CMS
				\Extbase\Object\Container\Container::SCOPE_SINGLETON
			\end{lstlisting}

	\end{itemize}

\end{frame}

% ------------------------------------------------------------------------------
% Page type for URIs with format
% ------------------------------------------------------------------------------
% http://forge.typo3.org/issues/27498

\begin{frame}[fragile]
	\frametitle{Extbase \& Fluid}
	\framesubtitle{Tipo di pagina per URIs}

	\lstset{
		basicstyle=\tiny\ttfamily
	}

	\begin{itemize}
		\item L'attributo per il tipo di pagina personalizzato non è più necessario nel link, quando è visualizzato un formato speciale

			\smaller\textbf{TYPO3 < 6.2:}\normalsize
			\begin{lstlisting}
				<f:link.action arguments="{blog: blog}" pageType="{settings.plaintextPageType}"
				  format="txt">[plaintext]</f:link.action></li>
			\end{lstlisting}

		\item Nuova opzione TypoScript \texttt{formatToPageTypeMapping} permette un'assegnazione globale:

			\begin{lstlisting}
				plugin.tx_myextension {
				  view.formatToPageTypeMapping {
				    txt = 99
				    pdf = 123
				  }
				}
			\end{lstlisting}

			\smaller\textbf{TYPO3 >= 6.2:}\normalsize
			\begin{lstlisting}
				<f:link.action arguments="{blog: blog}"
				  format="txt">[plaintext]</f:link.action></li>
			\end{lstlisting}

	\end{itemize}

\end{frame}

% ------------------------------------------------------------------------------
% Object Type Converter
% ------------------------------------------------------------------------------
% http://forge.typo3.org/issues/48548

\begin{frame}[fragile]
	\frametitle{Extbase \& Fluid}
	\framesubtitle{Conversione del tipo di oggetto (1)}

	\begin{itemize}
		\item Array sorgente di mappatura ad oggetti non persistenti
		\item Utile se si ha bisogno di oggetti transizionali costruiti da parametri della richiesta
		\item Qualche esempio nelle slide seguenti...

	\end{itemize}

\end{frame}

% ------------------------------------------------------------------------------
% Object Type Converter
% ------------------------------------------------------------------------------
% http://forge.typo3.org/issues/48548

\begin{frame}[fragile]
	\frametitle{Extbase \& Fluid}
	\framesubtitle{Conversione del tipo di oggetto (2)}

	\lstset{
		basicstyle=\tiny\ttfamily
	}

	\smaller\textbf{Richiesta GET}\normalsize
	\begin{lstlisting}
		http://example.com/index.php?id=299
		  &tx_myextension[action]=list
		  &tx_myextension[controller]=Entity
		  &tx_myextension[demand][title]=foo
		  &tx_myextension[demand][relation]=1
	\end{lstlisting}

	\smaller\textbf{Entity controller: \texttt{initializeListAction()}}\normalsize
	\begin{lstlisting}
		use [Vendor]\myextension\Domain\Dto\Demand;
		public function initializeListAction() {
		  /**
		   * @var PropertyMappingConfiguration $demandConfiguration
		   */
		  $demandConfiguration = $this->arguments['demand']->getPropertyMappingConfiguration();
		  $demandConfiguration->allowAllProperties()->forProperty('relation')->allowAllProperties()->
		    setTypeConverterOption(
		      'TYPO3\\CMS\\Extbase\\Property\\TypeConverter\\PersistentObjectConverter',
		      PersistentObjectConverter::CONFIGURATION_CREATION_ALLOWED,
		      TRUE
		  );
		}
	\end{lstlisting}

\end{frame}

% ------------------------------------------------------------------------------
% Object Type Converter
% ------------------------------------------------------------------------------
% http://forge.typo3.org/issues/48548

\begin{frame}[fragile]
	\frametitle{Extbase \& Fluid}
	\framesubtitle{Conversione del tipo di oggetto (3)}

	\lstset{
		basicstyle=\tiny\ttfamily
	}

	\smaller\textbf{Entity controller: \texttt{listAction()}}\normalsize
	\begin{lstlisting}
		use [Vendor]\myextension\Domain\Dto\Demand;
		/**
		 * @var PropertyMappingConfiguration $demandConfiguration
		 */
		public function listAction(Demand $demand = NULL) {
		  $entities = $this->entityRepository->findAll();
		  $this->view->assign('entities', $entities);
		}
	\end{lstlisting}

	\smaller\textbf{Model: \texttt{[Vendor]\textbackslash myextension\textbackslash Domain\textbackslash Dto\textbackslash Demand.php}}\normalsize
	\begin{lstlisting}
		namespace [Vendor]\myextension\Domain\Dto;
		use [Vendor]\myextension\Domain\Model\Relation;
		class Demand {
		  protected $relation;
		  /**
		   * @param \TYPO3Friends\MapperExample\Domain\Model\Relation $relation
		   */
		  public function setRelation($relation) {
		    $this->relation = $relation;
		  }
		}
	\end{lstlisting}

\end{frame}

% ------------------------------------------------------------------------------
% Chaining of set* functions
% ------------------------------------------------------------------------------
% http://forge.typo3.org/issues/47684

\begin{frame}[fragile]
	\frametitle{Extbase \& Fluid}
	\framesubtitle{Concatenamento di funzioni set*}

	\lstset{
		basicstyle=\tiny\ttfamily
	}

	\begin{itemize}
		\item I metodi \texttt{set*} possono essere \emph{concatenati} all'interno delle API di QuerySettings
		\item Sono state introdotte nuove opzioni con TYPO3 CMS 6.0:\newline
			\texttt{setIncludeDeleted} e \texttt{setIgnoreEnableFields}

			\begin{lstlisting}
				$query->getQuerySettings()
				  ->setRespectStoragePage(FALSE)
				  ->setRespectSysLanguage(FALSE)
				  ->setIgnoreEnableFields(TRUE)
				  ->setIncludeDeleted(TRUE);
			\end{lstlisting}
	\end{itemize}

\end{frame}

% ------------------------------------------------------------------------------
% returnRawQueryResult as argument
% ------------------------------------------------------------------------------
% http://forge.typo3.org/issues/51145

\begin{frame}[fragile]
	\frametitle{Extbase \& Fluid}
	\framesubtitle{returnRawQueryResult come argomento}

	\lstset{
		basicstyle=\smaller\ttfamily
	}

	\begin{itemize}
		\item Il risultato di una query Raw non è più il metodo centrale,\newline
			ma come parametro di un metodo: \texttt{execute()}
			\newline

			\smaller\textbf{TYPO3 < 6.2:}\normalsize\newline
			\lstinline!$query->getQuerySettings()->setReturnRawQueryResult(TRUE);!
			\newline

			\smaller\textbf{TYPO3 >= 6.2:}\normalsize\newline
			\lstinline!$query->execute(TRUE);!

	\end{itemize}

\end{frame}

% ------------------------------------------------------------------------------
% Recursive validation
% ------------------------------------------------------------------------------
% http://forge.typo3.org/issues/6893

\begin{frame}[fragile]
	\frametitle{Extbase \& Fluid}
	\framesubtitle{Validazione recursiva}

	\begin{itemize}
		\item Ora Extbase utilizza la validazione recursiva (come in TYPO3 Flow)
		\item Questo significa che quando gli oggetti nidificati sono creati da Property-Mapper,
			gli oggetti all'interno di una struttura, così come l'oggetto esterno sono validati\newline
			(in TYPO3 CMS < 6.2, solo l'oggetto esterno era validato)
		\item Inoltre, ora i validatori consentono valori vuoti
	\end{itemize}

	\breakingchange

	\smaller\begin{center} Per avere una proprietà obbligatoria, va \underline{aggiunto} esplicitamente \textbf{NotEmptyValidator}!\end{center}\normalsize

\end{frame}

% ------------------------------------------------------------------------------
% Application Context
% ------------------------------------------------------------------------------
% http://forge.typo3.org/issues/49988

\begin{frame}[fragile]
	\frametitle{Extbase \& Fluid}
	\framesubtitle{Contesto dell'applicazione}

	\begin{itemize}
		\item Accesso al contesto dell'applicazione corrente in Extbase\newline
			(impostare una variabile d'ambiente \texttt{TYPO3\_CONTEXT} o nello "Strumento d'installazione")\newline

			\lstinline!\TYPO3\CMS\Core\Core\Bootstrap::getInstance()->getContext();!
			\lstinline!\TYPO3\CMS\Core\Utility\GeneralUtility::getContext();!

	\end{itemize}

\end{frame}

% ------------------------------------------------------------------------------
% ViewHelper: Image
% ------------------------------------------------------------------------------
% http://forge.typo3.org/issues/47552

\begin{frame}[fragile]
	\frametitle{Extbase \& Fluid}
	\framesubtitle{ViewHelper: Image}

	\begin{itemize}
		\item Fluid ViewHelper \textbf{image} hanno l'opzione \texttt{title}\newline

			\smaller\textbf{Esempio:}\normalsize\newline
			\lstinline!<f:image src="background.jpg" alt="Text" />!
			\newline

			\smaller\textbf{TYPO3 < 6.2:}\normalsize\newline
			\lstinline!<img src="background.jpg" alt="Text" title="Text" />!
			\newline

			\smaller\textbf{TYPO3 >= 6.2:}\normalsize\newline
			\lstinline!<img src="background.jpg" alt="Text" />!

	\end{itemize}

\end{frame}

% ------------------------------------------------------------------------------
% ViewHelper: textfield and textarea
% ------------------------------------------------------------------------------
% http://forge.typo3.org/issues/45960
% http://forge.typo3.org/issues/48689 (Added autofocus attribute to textfield and textarea)

\begin{frame}[fragile]
	\frametitle{Extbase \& Fluid}
	\framesubtitle{ViewHelper: textfield e textarea}

	\begin{itemize}
		\item Parametri \texttt{autofocus} e \texttt{placeholder} (parametri per HTML5) nei Fluid ViewHelpers \textbf{form.textarea} e \textbf{form.textfield}\newline

			\smaller\textbf{Esempio ("placeholder"):}\normalsize
			\begin{lstlisting}
				<f:form.textfield
				  id="powermail_field_{field.marker}"
				  ...
				  placeholder="{field.title -> vh:string.RawAndRemoveXss()}"
				  ...
				  name="field[{field.uid}]"
				  required="{field.mandatory}" />
			\end{lstlisting}

	\end{itemize}

\end{frame}

% ------------------------------------------------------------------------------
% ViewHelper: switch
% ------------------------------------------------------------------------------
% http://forge.typo3.org/issues/48653

\begin{frame}[fragile]
	\frametitle{Extbase \& Fluid}
	\framesubtitle{ViewHelper: switch}

	\lstset{
		basicstyle=\smaller\ttfamily
	}

	\begin{itemize}
		\item Nuovo Fluid ViewHelper \textbf{switch} visualizza un contenuto in base ad un valore dato o un'espressione
		\item Si comporta in modo simile al metodo \texttt{switch()} di PHP

			\begin{lstlisting}
				<f:switch expression="{person.gender}">
				  <f:case value="male">Mr.</f:case>
				  <f:case value="female">Mrs.</f:case>
				</f:switch>
			\end{lstlisting}

		\item \textbf{\underline{Nota:}} un eccessivo uso di questo ViewHelper è un indicatore di cattivo design! L'esempio sopra potrebbe essere fatto con l'uso dei partials "\texttt{title.male.html}" e "\texttt{title.female.html}" e il seguente:

			\begin{lstlisting}
				<f:render partial="title.{person.gender}" />
			\end{lstlisting}

	\end{itemize}

\end{frame}

% ------------------------------------------------------------------------------
% ViewHelper: fileSize
% ------------------------------------------------------------------------------
% http://forge.typo3.org/issues/49139

\begin{frame}[fragile]
	\frametitle{Extbase \& Fluid}
	\framesubtitle{ViewHelper: fileSize}

	\lstset{
		basicstyle=\smaller\ttfamily
	}

	\begin{itemize}
		\item Converte la dimensioni di un file (intero) in una stringa leggibile\newline

			\smaller\textbf{Esempio:} fileSize = 1263616:\normalsize
			\begin{lstlisting}
			{fileSize -> f:format.bytes()}
			\end{lstlisting}
			\smaller Output: "1234 KB"\normalsize\newline
			\begin{lstlisting}
			{fileSize -> f:format.bytes(
			  decimals: 2,
			  decimalSeparator: '.',
			  thousandsSeparator: ','
			)}
			\end{lstlisting}
			\smaller Output: "1,234.00 KB"\normalsize

	\end{itemize}

\end{frame}

% ------------------------------------------------------------------------------
% ViewHelper: format.date
% (slide added in March 2014)
% ------------------------------------------------------------------------------

\begin{frame}[fragile]
        \frametitle{Extbase \& Fluid}
        \framesubtitle{ViewHelper: format.date}

        \lstset{
                basicstyle=\smaller\ttfamily
        }

        \begin{itemize}
                \item Il valore di default del ViewHelper \textbf{format.date} è il valore configurato nello "strumento di installazione"

                        \lstinline!$GLOBALS['TYPO3_CONF_VARS']['SYS']['ddmmyy']!

                \item Se questo valore non è impostato, è utilizzato "\texttt{Y-m-d}" (anno, mese, giorno)

        \end{itemize}

\end{frame}

% ------------------------------------------------------------------------------
% ViewHelper: Backend Container
% ------------------------------------------------------------------------------
% http://forge.typo3.org/issues/49749

\begin{frame}[fragile]
	\frametitle{Extbase \& Fluid}
	\framesubtitle{ViewHelper Backend Container}

	\lstset{
		basicstyle=\smaller\ttfamily
	}

	\begin{itemize}
		\item Fluid ViewHelper backend container (\texttt{be.container}) rielaborato:\newline
			\smaller\texttt{typo3/sysext/fluid/Classes/ViewHelpers/Be/ContainerViewHelper.php}\normalsize\newline

			\smaller\textbf{Deprecato:}\normalsize
			\begin{itemize}
				\item \texttt{\$addCssFile} (use \texttt{\$includeCssFiles} instead)
				\item \texttt{\$addJsFile} (use \texttt{\$includeJsFiles} instead)
			\end{itemize}

			\smaller\textbf{Nuovo:}\normalsize
			\begin{itemize}
				\item \texttt{\$loadJQuery}
				\item \texttt{\$includeCssFiles}
				\item \texttt{\$includeJsFiles}
				\item \texttt{\$addJsInlineLabels}
			\end{itemize}

	\end{itemize}

\end{frame}

% ------------------------------------------------------------------------------
% ViewHelper: button.icon
% ------------------------------------------------------------------------------

\begin{frame}[fragile]
	\frametitle{Extbase \& Fluid}
	\framesubtitle{ViewHelper: button.icon}

	\lstset{
		basicstyle=\smaller\ttfamily
	}

	\begin{itemize}
		\item Fluid ViewHelper \textbf{button.icon} consolidato (era un "esperimento")
		\item Crea un bottone a icona (opzionalmente con un link)

			\begin{lstlisting}
				<f:be.buttons.icon uri="{f:uri.action(action:'new')}"
				  icon="actions-document-new" title="Create new Foo" />

				<f:be.buttons.icon
				  icon="actions-document-new" title="Create new Foo" />
			\end{lstlisting}

		\item L'attributo \texttt{icon} accetta più di 310 valori!\newline

			\smaller
				Cerca:\newline
				\texttt{\$GLOBALS['TBE\_STYLES']['spriteIconApi']['coreSpriteImageNames']}\newline
				...nel file:\newline
				\texttt{typo3/systext/core/ext\_tables.php}
			\normalsize

	\end{itemize}

\end{frame}

% ------------------------------------------------------------------------------
% Option addQueryStringMethod
% ------------------------------------------------------------------------------
% http://forge.typo3.org/issues/11441
% http://forge.typo3.org/issues/35281

\begin{frame}[fragile]
	\frametitle{Extbase \& Fluid}
	\framesubtitle{Opzione addQueryStringMethod}

	\begin{itemize}
		\item L'opzione \texttt{addQueryString} supporta solo argomenti \textbf{GET}\newline
			\small(che sono poi aggiunti al link generato)\normalsize
		\item gli argomenti \textbf{POST} (usati dai Widget) non funzionano con questa opzione
		\item La nuova opzione \texttt{addQueryStringMethod} gestisce questo problema e permette di definire quale metodo deve essere preso in considerazione:\newline
			GET (default), POST, GET/POST or POST/GET
		\item Molti Fluid ViewHelpers utilizzano questa nuova opzione:

			\begin{itemize}\smaller
				\item \texttt{link.action}
				\item \texttt{link.page}
				\item \texttt{uri.action}
				\item \texttt{uri.page}
				\item \texttt{widget.link}
				\item \texttt{widget.uri}
				\item \texttt{widget.paginate}
			\end{itemize}

	\end{itemize}

\end{frame}

% ------------------------------------------------------------------------------
% Fluid: Fallback path for templates
% ------------------------------------------------------------------------------

\begin{frame}[fragile]
	\frametitle{Extbase \& Fluid}
	\framesubtitle{Fluid: Fallback Path per i template}

	\lstset{
		basicstyle=\tiny\ttfamily
	}

	\begin{itemize}
		\item Fluid gestisce "fallback" path per i templates, partials e layouts:\newline
			\smaller\texttt{templateRootPaths}, \texttt{partialRootPaths}, \texttt{layoutRootPaths}\normalsize
		\item Prima l'indice più alto, poi gli indici più bassi, fino a trovare il template

			\begin{lstlisting}
				plugin.tx_myextension {
				  view {
				    templateRootPath = EXT:myextension/Resources/Private/Templates/
				  }
				}

				plugin.tx_myextension {
				  view {
				    templateRootPath >
				    templateRootPaths {
				      10 = fileadmin/myextension/Templates/
				      20 = EXT:myextension/Resources/Private/Templates/
				    }
				  }
				}
			\end{lstlisting}

	\end{itemize}

\end{frame}

% ------------------------------------------------------------------------------

