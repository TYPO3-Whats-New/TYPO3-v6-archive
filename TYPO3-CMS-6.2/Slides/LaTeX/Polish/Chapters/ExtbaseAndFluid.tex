% ------------------------------------------------------------------------------
% TYPO3 CMS 6.2 LTS - What's New - Chapter "Extbase & Fluid" (English Version)
%
% @author	Michael Schams <schams.net>
% @license	Creative Commons BY-NC-SA 3.0
% @link		http://typo3.org/download/release-notes/whats-new/
% @language	English
% ------------------------------------------------------------------------------
% Chapter: Extbase & Fluid
% ------------------------------------------------------------------------------
\section{Extbase \& Fluid}
\begin{frame}[fragile]
	\frametitle{Extbase \& Fluid}

	\begin{center}\huge{Rozdział 8:}\end{center}
	\begin{center}\huge{\color{typo3darkgrey}\textbf{Extbase \& Fluid}}\end{center}

\end{frame}

% ------------------------------------------------------------------------------
% ObjectManager->getScope()
% ------------------------------------------------------------------------------
% http://forge.typo3.org/issues/48766

\begin{frame}[fragile]
	\frametitle{Extbase \& Fluid}
	\framesubtitle{ObjectManager->getScope()}

	\lstset{
		basicstyle=\tiny\ttfamily
	}

	\begin{itemize}
		\item Metoda \texttt{ObjectManager->getScope()} determinuje,\newline
			czy klasa jest typu \textbf{prototype} czy \textbf{singleton}

			\begin{lstlisting}
				/**
				 * @var \TYPO3\CMS\Extbase\Object\ObjectManagerInterface
				 * @inject
				 */
				protected $objectManager;

				$this->objectManager->getScope($propertyTargetClassName) === \TYPO3\CMS
				\Extbase\Object\Container\Container::SCOPE_PROTOTYPE

				$this->objectManager->getScope($propertyTargetClassName) === \TYPO3\CMS
				\Extbase\Object\Container\Container::SCOPE_SINGLETON
			\end{lstlisting}

	\end{itemize}

\end{frame}

% ------------------------------------------------------------------------------
% Page type for URIs with format
% ------------------------------------------------------------------------------
% http://forge.typo3.org/issues/27498

\begin{frame}[fragile]
	\frametitle{Extbase \& Fluid}
	\framesubtitle{Typ Strony Dla URI}

	\lstset{
		basicstyle=\tiny\ttfamily
	}

	\begin{itemize}
		\item Niestandardowy typ atrybutu strony nie jest dalej wymagany w linkach, przy renderowaniu specjalnego formatu

			\smaller\textbf{TYPO3 < 6.2:}\normalsize
			\begin{lstlisting}
				<f:link.action arguments="{blog: blog}" pageType="{settings.plaintextPageType}"
				  format="txt">[plaintext]</f:link.action></li>
			\end{lstlisting}

		\item Nowa opcja TypoScript \texttt{formatToPageTypeMapping} pozwala na globalne mapowanie:

			\begin{lstlisting}
				plugin.tx_myextension {
				  view.formatToPageTypeMapping {
				    txt = 99
				    pdf = 123
				  }
				}
			\end{lstlisting}

			\smaller\textbf{TYPO3 >= 6.2:}\normalsize
			\begin{lstlisting}
				<f:link.action arguments="{blog: blog}"
				  format="txt">[plaintext]</f:link.action></li>
			\end{lstlisting}

	\end{itemize}

\end{frame}

% ------------------------------------------------------------------------------
% Object Type Converter
% ------------------------------------------------------------------------------
% http://forge.typo3.org/issues/48548

\begin{frame}[fragile]
	\frametitle{Extbase \& Fluid}
	\framesubtitle{Konwerter typu obiektu (1)}

	\begin{itemize}
		\item Mapuje tablice danych źródłowych na obiekty nie trwałe
		\item Przydatne gdy musisz zbudować obiekt przejściowy z argumentów otrzymanych w odpowiedzi
		\item Kilka przykładów na kolejnych slajdach...

	\end{itemize}

\end{frame}

% ------------------------------------------------------------------------------
% Object Type Converter
% ------------------------------------------------------------------------------
% http://forge.typo3.org/issues/48548

\begin{frame}[fragile]
	\frametitle{Extbase \& Fluid}
	\framesubtitle{Konwerter typu obiektu (2)}

	\lstset{
		basicstyle=\tiny\ttfamily
	}

	\smaller\textbf{GET request}\normalsize
	\begin{lstlisting}
		http://example.com/index.php?id=299
		  &tx_myextension[action]=list
		  &tx_myextension[controller]=Entity
		  &tx_myextension[demand][title]=foo
		  &tx_myextension[demand][relation]=1
	\end{lstlisting}

	\smaller\textbf{Entity controller: \texttt{initializeListAction()}}\normalsize
	\begin{lstlisting}
		use [Vendor]\myextension\Domain\Dto\Demand;
		public function initializeListAction() {
		  /**
		   * @var PropertyMappingConfiguration $demandConfiguration
		   */
		  $demandConfiguration = $this->arguments['demand']->getPropertyMappingConfiguration();
		  $demandConfiguration->allowAllProperties()->forProperty('relation')->allowAllProperties()->
		    setTypeConverterOption(
		      'TYPO3\\CMS\\Extbase\\Property\\TypeConverter\\PersistentObjectConverter',
		      PersistentObjectConverter::CONFIGURATION_CREATION_ALLOWED,
		      TRUE
		  );
		}
	\end{lstlisting}

\end{frame}

% ------------------------------------------------------------------------------
% Object Type Converter
% ------------------------------------------------------------------------------
% http://forge.typo3.org/issues/48548

\begin{frame}[fragile]
	\frametitle{Extbase \& Fluid}
	\framesubtitle{Konwerter typu obiektu (3)}

	\lstset{
		basicstyle=\tiny\ttfamily
	}

	\smaller\textbf{Entity controller: \texttt{listAction()}}\normalsize
	\begin{lstlisting}
		use [Vendor]\myextension\Domain\Dto\Demand;
		/**
		 * @var PropertyMappingConfiguration $demandConfiguration
		 */
		public function listAction(Demand $demand = NULL) {
		  $entities = $this->entityRepository->findAll();
		  $this->view->assign('entities', $entities);
		}
	\end{lstlisting}

	\smaller\textbf{Model: \texttt{[Vendor]\textbackslash myextension\textbackslash Domain\textbackslash Dto\textbackslash Demand.php}}\normalsize
	\begin{lstlisting}
		namespace [Vendor]\myextension\Domain\Dto;
		use [Vendor]\myextension\Domain\Model\Relation;
		class Demand {
		  protected $relation;
		  /**
		   * @param \TYPO3Friends\MapperExample\Domain\Model\Relation $relation
		   */
		  public function setRelation($relation) {
		    $this->relation = $relation;
		  }
		}
	\end{lstlisting}

\end{frame}

% ------------------------------------------------------------------------------
% Chaining of set* functions
% ------------------------------------------------------------------------------
% http://forge.typo3.org/issues/47684

\begin{frame}[fragile]
	\frametitle{Extbase \& Fluid}
	\framesubtitle{Chaining Of set* Functions}

	\lstset{
		basicstyle=\tiny\ttfamily
	}

	\begin{itemize}
		\item \texttt Metoda {set*} może być \emph{łańcuchem} w ciągu QuerySettings API
		\item Dołącza nowe opcje przedstawione wraz z TYPO3 CMS 6.0:\newline
			\texttt{setIncludeDeleted} i \texttt{setIgnoreEnableFields}

			\begin{lstlisting}
				$query->getQuerySettings()
				  ->setRespectStoragePage(FALSE)
				  ->setRespectSysLanguage(FALSE)
				  ->setIgnoreEnableFields(TRUE)
				  ->setIncludeDeleted(TRUE);
			\end{lstlisting}
	\end{itemize}

\end{frame}

% ------------------------------------------------------------------------------
% returnRawQueryResult as argument
% ------------------------------------------------------------------------------
% http://forge.typo3.org/issues/51145

\begin{frame}[fragile]
	\frametitle{Extbase \& Fluid}
	\framesubtitle{returnRawQueryResult jako argument}

	\lstset{
		basicstyle=\smaller\ttfamily
	}

	\begin{itemize}
		\item Czysty wynik zapytania nie jest dalej metodą centralną,\newline
			ale jako argument metody: \texttt{execute()}
			\newline

			\smaller\textbf{TYPO3 < 6.2:}\normalsize\newline
			\lstinline!$query->getQuerySettings()->setReturnRawQueryResult(TRUE);!
			\newline

			\smaller\textbf{TYPO3 >= 6.2:}\normalsize\newline
			\lstinline!$query->execute(TRUE);!

	\end{itemize}

\end{frame}

% ------------------------------------------------------------------------------
% Recursive validation
% ------------------------------------------------------------------------------
% http://forge.typo3.org/issues/6893

\begin{frame}[fragile]
	\frametitle{Extbase \& Fluid}
	\framesubtitle{Walidacja Rekurencyjna}

	\begin{itemize}
		\item Extbase używa teraz rekurencyjnej walidacji (znanej z TYPO3 Flow)
		\item To oznacza że, gdy zagnieżdżone obiekty są tworzone przez Property-Mapper, obiekty wewnątrz właściwości, jak i na zewnątrz obiektu są
		walidowane\newline
			(w TYPO3 CMS < 6.2, tylko zewnętrzne obiekty są walidowane)
		\item Dodatkowko, walidator pozwala teraz na puste wartości
	\end{itemize}

	\breakingchange

	\smaller\begin{center} By właściwość była wymagana \underline{musisz} koniecznie dodać \textbf{NotEmptyValidator}!\end{center}\normalsize

\end{frame}

% ------------------------------------------------------------------------------
% Application Context
% ------------------------------------------------------------------------------
% http://forge.typo3.org/issues/49988

\begin{frame}[fragile]
	\frametitle{Extbase \& Fluid}
	\framesubtitle{Kontekst Aplikacji}

	\begin{itemize}
		\item Dostęp do aktualnego kontekstu aplikacji w Extbase\newline
			(ustaw zmienną środowiskową \texttt{TYPO3\_CONTEXT} lub w Instalatorze)\newline

			\lstinline!\TYPO3\CMS\Core\Core\Bootstrap::getInstance()->getContext();!
			\lstinline!\TYPO3\CMS\Core\Utility\GeneralUtility::getContext();!

	\end{itemize}

\end{frame}

% ------------------------------------------------------------------------------
% ViewHelper: Image
% ------------------------------------------------------------------------------
% http://forge.typo3.org/issues/47552

\begin{frame}[fragile]
	\frametitle{Extbase \& Fluid}
	\framesubtitle{ViewHelper: image}

	\begin{itemize}
		\item Fluid ViewHelper \textbf{image} z opcjonalnym atrybutem \texttt{title} 

			\smaller\textbf{Przykład:}\normalsize\newline
			\lstinline!<f:image src="background.jpg" alt="Text" />!
			\newline

			\smaller\textbf{TYPO3 < 6.2:}\normalsize\newline
			\lstinline!<img src="background.jpg" alt="Text" title="Text" />!
			\newline

			\smaller\textbf{TYPO3 >= 6.2:}\normalsize\newline
			\lstinline!<img src="background.jpg" alt="Text" />!

	\end{itemize}

\end{frame}

% ------------------------------------------------------------------------------
% ViewHelper: textfield and textarea
% ------------------------------------------------------------------------------
% http://forge.typo3.org/issues/45960
% http://forge.typo3.org/issues/48689 (Added autofocus attribute to textfield and textarea)

\begin{frame}[fragile]
	\frametitle{Extbase \& Fluid}
	\framesubtitle{ViewHelpers: textfield and textarea}

	\begin{itemize}
		\item Argumenty \texttt{autofocus} i \texttt{placeholder} (argument walidacji w HTML5) dla Fluid ViewHelpers \textbf{form.textarea} i \textbf{form.textfield}\newline

			\smaller\textbf{przykład ("placeholder"):}\normalsize
			\begin{lstlisting}
				<f:form.textfield
				  id="powermail_field_{field.marker}"
				  ...
				  placeholder="{field.title -> vh:string.RawAndRemoveXss()}"
				  ...
				  name="field[{field.uid}]"
				  required="{field.mandatory}" />
			\end{lstlisting}

	\end{itemize}

\end{frame}

% ------------------------------------------------------------------------------
% ViewHelper: switch
% ------------------------------------------------------------------------------
% http://forge.typo3.org/issues/48653

\begin{frame}[fragile]
	\frametitle{Extbase \& Fluid}
	\framesubtitle{ViewHelper: switch}

	\lstset{
		basicstyle=\smaller\ttfamily
	}

	\begin{itemize}
		\item Nowy Fluid ViewHelper \textbf{switch} renderuje kontent polegając na podanych wartościach lub wyrażeniach
		\item Zachowania podobne do wyrażenia \texttt{switch()} w PHP

			\begin{lstlisting}
				<f:switch expression="{person.gender}">
				  <f:case value="male">Mr.</f:case>
				  <f:case value="female">Mrs.</f:case>
				  <f:case default="TRUE">Mrs. or Mr.</f:case>
				</f:switch>
			\end{lstlisting}

		\item \textbf{\underline{Uwaga:}} nadmierne używanie ViewHelper-a może zapoczątkować błędny design! Podobny rezultat może być osiągnięty używając podszablonów "\texttt{title.male.html}" oraz "\texttt{title.female.html}" i następujące:

			\begin{lstlisting}
				<f:render partial="title.{person.gender}" />
			\end{lstlisting}

	\end{itemize}

\end{frame}

% ------------------------------------------------------------------------------
% ViewHelper: fileSize
% ------------------------------------------------------------------------------
% http://forge.typo3.org/issues/49139

\begin{frame}[fragile]
	\frametitle{Extbase \& Fluid}
	\framesubtitle{ViewHelper: fileSize}

	\begin{itemize}
		\item Konwertuje wielkość pliku (stała) do zrozumiałego przez człowieka string-a\newline
	\end{itemize}

	\begin{columns}[T]

		\begin{column}{.5\textwidth}
			\advance\leftskip+0.8cm
			\smaller
				\textbf{Przykład 1} (fileSize = 1263616):\newline
				\texttt{fileSize -> f:format.bytes()}\newline
				\newline
				Output: "1234 KB"
			\normalsize
		\end{column}
		\begin{column}{.5\textwidth}
			\smaller
				\textbf{Przykład 2} (fileSize = 1263616):\newline
				\texttt{fileSize -> f:format.bytes(}\newline
				\texttt{
				decimals: 2,\newline
				decimalSeparator: '.',\newline
				thousandsSeparator: ','\newline
				)}\newline
				\newline
				Output: "1,234.00 KB"
			\normalsize
		\end{column}

	\end{columns}

\end{frame}

% ------------------------------------------------------------------------------
% ViewHelper: format.date
% (slide added in March 2014)
% ------------------------------------------------------------------------------

\begin{frame}[fragile]
	\frametitle{Extbase \& Fluid}
	\framesubtitle{ViewHelper: format.date}

	\lstset{
		basicstyle=\smaller\ttfamily
	}

	\begin{itemize}
		\item Domyślna wartość ViewHelper-a \textbf{format.date} jest konfigurowalna z instalatora

			\lstinline!$GLOBALS['TYPO3_CONF_VARS']['SYS']['ddmmyy']!

		\item Jeśli ta zmienna nie jest ustawiona, "\texttt{Y-m-d}" to używane jest (year, month, day)

	\end{itemize}

\end{frame}

% ------------------------------------------------------------------------------
% ViewHelper: Backend Container
% ------------------------------------------------------------------------------
% http://forge.typo3.org/issues/49749

\begin{frame}[fragile]
	\frametitle{Extbase \& Fluid}
	\framesubtitle{ViewHelper: Backend Container}

	\lstset{
		basicstyle=\smaller\ttfamily
	}

	\begin{itemize}
		\item Fluid ViewHelper kontener panelu administracyjnego (\texttt{be.container}) zmienione:\newline
			\smaller\texttt{typo3/sysext/fluid/Classes/ViewHelpers/Be/ContainerViewHelper.php}\normalsize\newline

			\smaller\textbf{Nieaktualne:}\normalsize
			\begin{itemize}
				\item \texttt{\$addCssFile} (use \texttt{\$includeCssFiles} instead)
				\item \texttt{\$addJsFile} (use \texttt{\$includeJsFiles} instead)
			\end{itemize}

			\smaller\textbf{Nowe:}\normalsize
			\begin{itemize}
				\item \texttt{\$loadJQuery}
				\item \texttt{\$includeCssFiles}
				\item \texttt{\$includeJsFiles}
				\item \texttt{\$addJsInlineLabels}
			\end{itemize}

	\end{itemize}

\end{frame}

% ------------------------------------------------------------------------------
% ViewHelper: button.icon
% ------------------------------------------------------------------------------

\begin{frame}[fragile]
	\frametitle{Extbase \& Fluid}
	\framesubtitle{ViewHelper: button.icon}

	\lstset{
		basicstyle=\smaller\ttfamily
	}

	\begin{itemize}
		\item Stworzenie Fluid ViewHelper \textbf{button.icon} (było "eksperymentem")
		\item Tworzy ikone przycisku (opcjonalnie z linkiem)

			\begin{lstlisting}
				<f:be.buttons.icon uri="{f:uri.action(action:'new')}"
				  icon="actions-document-new" title="Create new Foo" />

				<f:be.buttons.icon
				  icon="actions-document-new" title="Create new Foo" />
			\end{lstlisting}

		\item Atrybut \texttt{icon} akceptuje ponad 310 wartości!\newline

			\smaller
				Szukaj:\newline
				\texttt{\$GLOBALS['TBE\_STYLES']['spriteIconApi']['coreSpriteImageNames']}\newline
				...w pliku:\newline
				\texttt{typo3/systext/core/ext\_tables.php}
			\normalsize

	\end{itemize}

\end{frame}

% ------------------------------------------------------------------------------
% Option addQueryStringMethod
% ------------------------------------------------------------------------------
% http://forge.typo3.org/issues/11441
% http://forge.typo3.org/issues/35281

\begin{frame}[fragile]
	\frametitle{Extbase \& Fluid}
	\framesubtitle{Opcja addQueryStringMethod}

	\begin{itemize}
		\item Opcja \texttt{addQueryString} wspiera \textbf tylko argumenty-{GET}\newline
			\small(które później są dodawane do wygenerowanego linku)\normalsize
		\item \textbf Argument-{POST} (używany przez widżety) nie funkcjonuje z tą opcją
		\item Nowa opcja \texttt{addQueryStringMethod} rozwiązuje ten problem i pozwala zdefiniować, która metoda powinna być brana pod uwagę:\newline
			GET (domyślnie), POST, GET/POST lub POST/GET
		\item Kilka Fluid ViewHelper-ów wspiera tą nową opcje:

			\begin{itemize}\smaller
				\item \texttt{link.action}
				\item \texttt{link.page}
				\item \texttt{uri.action}
				\item \texttt{uri.page}
				\item \texttt{widget.link}
				\item \texttt{widget.uri}
				\item \texttt{widget.paginate}
			\end{itemize}

	\end{itemize}

\end{frame}

% ------------------------------------------------------------------------------
% Fluid: Fallback path for templates
% ------------------------------------------------------------------------------

\begin{frame}[fragile]
	\frametitle{Extbase \& Fluid}
	\framesubtitle{Fluid: Fallback Path For Templates}

	\lstset{
		basicstyle=\tiny\ttfamily
	}

	\begin{itemize}
		\item Fluid wspiera teraz ścieżki "fallback" w szablonach, podstronach i layout-ach :\newline
			\smaller\texttt{templateRootPaths}, \texttt{partialRootPaths}, \texttt{layoutRootPaths}\normalsize
		\item Najpierw najwyższy index, następnie iteracja przez niskie indexy, dopóki szablon nie zostanie odnaleziony

			\begin{lstlisting}
				plugin.tx_myextension {
				  view {
				    templateRootPath = EXT:myextension/Resources/Private/Templates/
				  }
				}
				plugin.tx_myextension {
				  view {
				    templateRootPath >
				    templateRootPaths {
				      10 = fileadmin/myextension/Templates/
				      20 = EXT:myextension/Resources/Private/Templates/
				    }
				  }
				}
			\end{lstlisting}

	\end{itemize}

\end{frame}

% ------------------------------------------------------------------------------

