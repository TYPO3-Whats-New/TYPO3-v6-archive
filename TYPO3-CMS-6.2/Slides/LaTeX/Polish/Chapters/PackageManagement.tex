% ------------------------------------------------------------------------------
% TYPO3 CMS 6.2 LTS - What's New - Chapter "Package Management" (English Version)
%
% @author	Michael Schams <schams.net>
% @license	Creative Commons BY-NC-SA 3.0
% @link		http://typo3.org/download/release-notes/whats-new/
% @language	English
% ------------------------------------------------------------------------------
% Chapter: Package Management
% ------------------------------------------------------------------------------
\section{Zarządzanie pakietami}
\begin{frame}[fragile]
	\frametitle{Zarządzanie pakietami}

	\begin{center}\huge{Rozdział 5:}\end{center}
	\begin{center}\huge{\color{typo3darkgrey}\textbf{Zarządzanie pakietami}}\end{center}

\end{frame}

% ------------------------------------------------------------------------------
% Package Manager
% ------------------------------------------------------------------------------
% http://wiki.typo3.org/Blueprints/Packagemanager
% http://forge.typo3.org/issues/47018
% http://forge.typo3.org/issues/52737

	\begin{frame}[fragile]
	\frametitle{Zarządzanie pakietami}
	\framesubtitle{Menadżer pakietów}

	\begin{itemize}
		\item \textbf{Zarządzanie pakietami} z TYPO3 Flow przeniesione do TYPO3 CMS
		\item Rozwój rozpoczęty podczas pracy nad TYPO3 CMS 6.1
		\item Ten projekt dąży do harmonizacji formatów pakietów
		\item Rozszerzenia w TYPO3 CMS to tylko szczególny rodzaj pakietów
		\item Główne cele projektu:

			\begin{itemize}
				\item Właściwe API do zarządzania pakietami
				\item Wsparcie dla nazw firm dostawców
				\item Wsparcie dla pakietów Composer'a
				\item Wsparcie dla pakietów Flow
				\item Refaktoryzacja Autoloader'a
			\end{itemize}

	\end{itemize}

\end{frame}

% ------------------------------------------------------------------------------
% Package Manager Integration
% ------------------------------------------------------------------------------

\begin{frame}[fragile]
	\frametitle{Zarządzanie pakietami}
	\framesubtitle{Integracja Menadżera Pakietów}
	\begin{itemize}
		\item Usunięcie \texttt{\$TYPO3\_CONF['EXT']['extListArray']} z pliku\newline
			\smaller\texttt{typo3conf/LocalConfiguration.php}\normalsize

		\item Plik \small\texttt{typo3conf/LocalConfiguration.php} skopiowany do\normalsize\newline
			\smaller\texttt{typo3conf/LocalConfiguration.beforePackageStatesMigration.php}\normalsize

		\item Plik \texttt{typo3conf/PackageStates.php} zawiera:

			\begin{itemize}
				\item status pakietu (aktywny/nieaktywny)
				\item lokalizacja rozszerzenia w systemie plików
			\end{itemize}

		\item Rozszerzenia w następujących katalogach są wykrywane automatycznie:

			\begin{itemize}
				\item \texttt{typo3/sysext/}
				\item \texttt{typo3/ext/}
				\item \texttt{typo3/contrib/}
				\item \texttt{typo3conf/ext/}
				\item \texttt{Packages/} \emph{(rekurencyjnie)}
			\end{itemize}

	\end{itemize}

\end{frame}

% ------------------------------------------------------------------------------
% Package Manager Integration
% ------------------------------------------------------------------------------

\begin{frame}[fragile]
	\frametitle{Zarządzanie pakietami}
	\framesubtitle{Integracja Menadżera Pakietów}

	\begin{itemize}

		\item Dwa nowe (dodatkowe) pliki w katalogu rozszerzeń:

			\begin{itemize}
				\item \texttt{composer.json}
				\item \texttt{Classes/Package.php}
			\end{itemize}

		\item Jeżeli wymagane jest rozszerzenie, flaga \texttt{protected} ma być ustawiona w pliku \texttt{composer.json}

		\item Jeżeli plik \texttt{PackageStates.php} nie istnieje, zostanie (ponownie) stworzony,
			wraz z wszystkimi rozszerzeniami, których wartość zostanie ustawiona na \texttt{TRUE}

		\item Autoloader otrzymuje własny, zbuforowany panel administracyjny

		\item Więcej informacji:\newline
			\url{http://wiki.typo3.org/Blueprints/Packagemanager}

	\end{itemize}

\end{frame}

% ------------------------------------------------------------------------------
% Examples
% ------------------------------------------------------------------------------

\begin{frame}[fragile]
	\frametitle{Zarządzanie pakietami}
	\framesubtitle{Integracja Menadżera Pakietów}

	Przykład: \texttt{typo3conf/PackageManager.php}

	\lstset{
		basicstyle=\tiny\ttfamily
		% basicstyle=\fontsize{5}{7}\selectfont\ttfamily
	}

	\begin{lstlisting}
		return array ('packages' =>
		    array (
		      'core' =>
		        array (
		          'manifestPath' => '',
		          'composerName' => 'typo3/cms/core',
		          'state' => 'active',
		          'packagePath' => 'typo3/sysext/core/',
		          'classesPath' => 'Classes/',
		        ),
		      'workspaces' =>
		        array (
		          'manifestPath' => '',
		          'composerName' => 'typo3/cms/workspaces',
		          'state' => 'inactive',
		          'packagePath' => 'typo3/sysext/workspaces/',
		          'classesPath' => 'Classes/',
		        ),
		      ...
		    ),
		    'version' => 4,
		);
	\end{lstlisting}

\end{frame}

% ------------------------------------------------------------------------------
% Examples
% ------------------------------------------------------------------------------

\begin{frame}[fragile]
	\frametitle{Zarządzanie pakietami}
	\framesubtitle{Integracja Menadżera Pakietów}

	Przykład: \texttt{composer.json}

	\lstset{
		basicstyle=\tiny\ttfamily
	}

	\begin{lstlisting}
		{
		  "name": "typo3/cms-indexed-search",
		  "type": "typo3-cms-framework",
		  "description": "TYPO3 Core",
		  "homepage": "http://typo3.org",
		  "license": ["GPL-2.0+"],
		  "version": "6.2.0",
		  "require": {
		    "typo3/cms-core": "*"
		  },
		  "replace": {
		    "indexed_search": "*"
		  }
		}
	\end{lstlisting}

\end{frame}

% ------------------------------------------------------------------------------
% Miscellaneous
% (slide added in March 2014)
% ------------------------------------------------------------------------------

\begin{frame}[fragile]
	\frametitle{Zarządzanie pakietami}
	\framesubtitle{Integracja Menadżera Pakietów}

	\lstset{
		basicstyle=\smaller\ttfamily
	}

	\begin{itemize}
		\item Pakiety mogą być także aktywowane w czasie wykonywania przy użyciu klucza:
			\smaller\texttt{\$GLOBALS['TYPO3\_CONF\_VARS']['EXT']['runtimeActivatedPackages'] = array(}\space\textit{packageKey}\space\texttt{);}\normalsize

		\item Ten klucz jest aktywowany natychmiast po inicjalizacji Menadżera Pakietów

	\end{itemize}

\end{frame}

% ------------------------------------------------------------------------------

