% ------------------------------------------------------------------------------
% TYPO3 CMS 6.2 LTS - What's New - Chapter "Application Programming Interface" (Polish Version)
%
% @author	Tymoteusz Motylewski <t.motylewski@macopedia.pl>
% @license	Creative Commons BY-NC-SA 3.0
% @link		http://typo3.org/download/release-notes/whats-new/
% @language	Polish
% ------------------------------------------------------------------------------
% Chapter: Application Programming Interface
% ------------------------------------------------------------------------------
\section{Interfejs programowania aplikacji}
\begin{frame}[fragile]
	\frametitle{Interfejs programowania aplikacji}

	\begin{center}\huge{Rozdział 7:}\end{center}
	\begin{center}\huge{\color{typo3darkgrey}\textbf{Interfejs programowania aplikacji (API)}}\end{center}

\end{frame}

% ------------------------------------------------------------------------------
% Hook: tsfe::checkEnableFields
% ------------------------------------------------------------------------------
% http://forge.typo3.org/issues/48981

\begin{frame}[fragile]
	\frametitle{Interfejs programowania aplikacji}
	\framesubtitle{Hak: \texttt{tsfe::checkEnableFields}}

	\begin{itemize}
		\item W TYPO3 < 6.2, "\emph{extend to subpages}" nie może być używany w swoim własnym rozszerzeniu posiadającym definicje widoczności strony\newline
			\small(lista pól do sprawdzenia jest zakodowana w \texttt{tsfe::checkEnableFields()})\normalsize

		\item W TYPO3 >= 6.2, nowy hak zezwala rozszerzeniom dostarczyć reguły widoczności strony w przypadku gdy strony rodzice posiadają włączony "extend to subpages" .
		\item Klasa:\newline
			\smaller
				\texttt{\textbackslash
					TYPO3\textbackslash
					CMS\textbackslash
					Frontend\textbackslash
					Controller\textbackslash
					TypoScriptFrontendController}\normalsize

			\lstset{
				basicstyle=\smaller\ttfamily
			}

			\begin{lstlisting}
				$GLOBALS['TYPO3_CONF_VARS']['SC_OPTIONS']
				  ['tslib/class.tslib_fe.php']['hook_checkEnableFields']
			\end{lstlisting}

	\end{itemize}

\end{frame}

% ------------------------------------------------------------------------------
% Hook: checkFlexFormValue in DataHandler
% ------------------------------------------------------------------------------
% http://forge.typo3.org/issues/49699

\begin{frame}[fragile]
	\frametitle{Interfejs programowania aplikacji}
	\framesubtitle{Hak: \texttt{checkFlexFormValue} w DataHandler-ze}

	\begin{itemize}
		\item W TYPO3 < 6.2, gdy aktualizujemy wartości Flexform-y, nie jest sprawdzane czy istniejąca wartość w bazie nie została usunięta
		\item To staje się problemem np. kiedy zapisujemy akcje kontrolera umożliwiającego przełączanie (Extbase) w Flexform-ie: stare akcje, które mogą być już nie używane muszą zostać usunięte ręcznie

		\item W TYPO3 >= 6.2, nowy hak pozwala na dostosowanie starej Flexform-y przed tym zanim zostanie scalona z nowa
		\item Klasa:\newline
			\smaller
				\texttt{\textbackslash
					TYPO3\textbackslash
					CMS\textbackslash
					Core\textbackslash
					DataHandling\textbackslash
					DataHandler}\normalsize

			\lstset{
				basicstyle=\smaller\ttfamily
			}

			\begin{lstlisting}
				$GLOBALS['TYPO3_CONF_VARS']['SC_OPTIONS']
				  ['t3lib/class.t3lib_tcemain.php']['checkFlexFormValue']
			\end{lstlisting}

		\item Metoda:\newline
			\smaller
				\texttt{checkFlexFormValue\_beforeMerge()}

	\end{itemize}

\end{frame}

% ------------------------------------------------------------------------------
% Hook to customize header in module "Web > Page"
% ------------------------------------------------------------------------------
% http://forge.typo3.org/issues/52579

\begin{frame}[fragile]
	\frametitle{Interfejs programowania aplikacji}
	\framesubtitle{Hak do modyfikownia nagłówka}

	\begin{itemize}
		\item Nowy hak w TYPO3 >= 6.2, pozwala modyfikować nagłówek strony w module strony (Module: "Web > Page")
		\item Ten hak jest wywoływany przed wyrenderowaniem zawartości strony
		\item Klasa:\newline
			\smaller
				\texttt{\textbackslash
					TYPO3\textbackslash
					CMS\textbackslash
					Backend\textbackslash
					Controller\textbackslash
					PageLayoutController}\normalsize

			\lstset{
				basicstyle=\smaller\ttfamily
			}

			\begin{lstlisting}
				$GLOBALS['TYPO3_CONF_VARS']['SC_OPTIONS']
				  ['cms/layout/db_layout.php']['drawHeaderHook']
			\end{lstlisting}

		\item Metoda:\newline
			\smaller
				\texttt{callUserFunction()}

	\end{itemize}

\end{frame}

% ------------------------------------------------------------------------------
% IRRE: Provide default values for created records
% ------------------------------------------------------------------------------
% http://forge.typo3.org/issues/46124

\begin{frame}[fragile]
	\frametitle{Interfejs programowania aplikacji}
	\framesubtitle{IRRE: domyślne wartości przy tworzeniu rekordów}

	\begin{itemize}
		\item Nowa opcja TCA pozwala konfigurować pola "inline"
		\item Klucz \texttt{foreign\_record\_defaults} pozwala na ustawienie (domyślnie) wartości w nowo utworzonych rekordach

			\begin{lstlisting}
				'config' => array(
				  'type' => 'inline',
				  'foreign_table' => 'tt_content',
				  'foreign_record_defaults' => array(
				    'CType' => 'image'
				  ),
				)
			\end{lstlisting}

			\small
				Przykład powyżej: \texttt{tt\_content} elementy, które są stworzone dla pola IRRE będą domyślnie jako\textbf{image content elements}. Edytor może ustawić to jako inny typ przed zapisaniem.
			\normalsize
	\end{itemize}

\end{frame}

% ------------------------------------------------------------------------------
% Workspaces
% ------------------------------------------------------------------------------
% http://forge.typo3.org/issues/46124

\begin{frame}[fragile]
	\frametitle{Interfejs programowania aplikacji}
	\framesubtitle{Obszary robocze (1)}

	\begin{itemize}
		\item W TYPO3 < 6.2, moduł "Workspaces" może być rozszerzony jedynie przez nadpisanie komponentów PHP i JavaScript
		\item W TYPO3 >= 6.2, możliwe jest rozszerzanie definicji i zachowania wyswietlanych kolumn w module
		\item Klika przykładów na kolejnych slajdach...
	\end{itemize}

\end{frame}

% ------------------------------------------------------------------------------
% Workspaces
% ------------------------------------------------------------------------------
% http://forge.typo3.org/issues/46124

\begin{frame}[fragile]
	\frametitle{Interfejs programowania aplikacji}
	\framesubtitle{Obszary robocze (2)}

	\lstset{
		basicstyle=\tiny\ttfamily
	}

	Przykład (plik \texttt{ext\_localconf.php}):

	\begin{lstlisting}
		$GLOBALS['TYPO3_CONF_VARS']['SC_OPTIONS']
		  ['t3lib/class.t3lib_tcemain.php']['processCmdmapClass']['workspaces_logger'] =
		  'Vendor\\WorkspacesLogger\\Hook\\DataHandlerHook';
	\end{lstlisting}

	Przykład (plik \texttt{ext\_tables.php}):

	\begin{lstlisting}
		\TYPO3\CMS\Workspaces\Service\AdditionalColumnService::getInstance()->register(
		  'WorkspacesLogger_StageChange',
		  'Vendor\\WorkspacesLogger\\DataProvider'
		);

		\TYPO3\CMS\Workspaces\Service\AdditionalResourceService::getInstance()->addJavaScriptResource(
		  'WorkspacesLogger',
		  'EXT:myextension/Resources/Public/JavaScript/StageChange.js'
		);
	\end{lstlisting}

\end{frame}

% ------------------------------------------------------------------------------
% Workspaces
% ------------------------------------------------------------------------------
% http://forge.typo3.org/issues/46124

\begin{frame}[fragile]
	\frametitle{Interfejs programowania aplikacji}
	\framesubtitle{Obszary robocze (3)}

	Przykład (plik \texttt{Vendor\textbackslash
		WorkspacesLogger\textbackslash
		Hook\textbackslash
		DataHandlerHook}):

	\lstset{
		basicstyle=\tiny\ttfamily
	}

	\begin{lstlisting}
		<?php
		namespace Vendor\WorkspacesLogger\Hook;
		use TYPO3\CMS\Core\SingletonInterface;

		class DataHandlerHook implements SingletonInterface {

		  const TABLE_Name = 'tx_workspaceslogger_event';
		  const EVENT_SetStage = 91;

		  /**
		   * hook that is called when no prepared command was found
		   */
		  public function processCmdmap($command, $table, $id, $value, &$commandIsProcessed,
		    \TYPO3\CMS\Core\DataHandling\DataHandler $tcemainObj) {
		    ...
		    $action = (string) $value['action'];
		    if ($command === 'version' && $action === 'setStage' && $commandIsProcessed) {
		      ...
		    }
		  }
		}
	\end{lstlisting}

\end{frame}

% ------------------------------------------------------------------------------
% PSR-3 compatible Logger
% ------------------------------------------------------------------------------
% http://forge.typo3.org/issues/48880

\begin{frame}[fragile]
	\frametitle{Interfejs programowania aplikacji}
	\framesubtitle{Kompatybilny logger PSR-3}

	\begin{itemize}
		\item CMS TYPO3 6.2 logging API jest teraz kompatybilne z PSR-3
		\item PSR-3 ma na celu ustalenie standardu logowania w PHP (standard framework-a PHP Interop Group)

		\item Głównym celem PSR-3 jest
			"\emph{ pozwalanie bibliotekom na odbieranie obiektów LoggerInterface i wpisywanie w nich logów w prosty i uniwersalny sposób.}"

		\item Logger interface zawiera skrócone metody log-u takie jak\newline
			\texttt{debug()}, \texttt{warning()}, \texttt{notice()}, \texttt{alert()}, \texttt{error()}, etc.

		\item Więcej informacji:\newline
			\url{http://www.php-fig.org/psr/3/}

	\end{itemize}

\end{frame}

% ------------------------------------------------------------------------------
% API to CSRF protect Ajax calls in Backend
% (slide added in March 2014)
% ------------------------------------------------------------------------------
% http://forge.typo3.org/issues/56345
% http://forge.typo3.org/issues/56347 (documentation)

\begin{frame}[fragile]
	\frametitle{Interfejs programowania aplikacji}
	\framesubtitle{CSRF Ochrona zapytań Ajax}

	\lstset{
		basicstyle=\tiny\ttfamily
	}

	\begin{itemize}
		\item Zapytania Ajax w panelu administracyjnym TYPO3 mogą być chronione przed CSRF (\textit{cross-site request forgery}) przez rejestrowanie pól przetrzymujących dane

			\begin{lstlisting}
				\TYPO3\CMS\Core\Utility\ExtensionManagementUtility::registerAjaxHandler(
				  'TxMyExt::process',
				  '\Vendor\MyExt\AjaxHandler->process'
				);
			\end{lstlisting}

		\item URL podanego identyfikatora Ajax jest chroniony tokenem CSRF, który jest sprawdzany w dispatcher-rze \texttt{ajax.php}

			\begin{lstlisting}
				$ajaxUrl = \TYPO3\CMS\Core\Utility\BackendUtility::getAjaxUrl('TxMyExt::process');
			\end{lstlisting}

		\item Następnie do tych ustawień możemy się dostać JavaScript-em w kontekście strony

			\begin{lstlisting}
				var ajaxUrl = TYPO3.settings.MyExt.ajaxUrl;
			\end{lstlisting}

	\end{itemize}

\end{frame}

% ------------------------------------------------------------------------------
% Miscellaneous
% ------------------------------------------------------------------------------
% http://forge.typo3.org/issues/49144 (MathUtility: Add canBeInterpretedAsFloat)
% http://forge.typo3.org/issues/52707 (Introduce a PHP Enumeration type)
% http://forge.typo3.org/issues/52762 (Add type converter for core types like Enumeration)

\begin{frame}[fragile]
	\frametitle{Interfejs programowania aplikacji}
	\framesubtitle{Różne}

	\begin{itemize}
		\item Nowa metoda \texttt{canBeInterpretedAsFloat()} w klasie: \texttt{MathUtility}\newline
			\small(Jest to analogiczne do: \texttt{canBeInterpretedAsInteger()})\normalsize
		\item Nowy typ enumeracji (bez relacji z 3-cią częścią modułu PHP):\newline
			\texttt{\textbackslash
				TYPO3\textbackslash
				CMS\textbackslash
				Core\textbackslash
				Type\textbackslash
				Enumeration}\newline

			Dla przykładu użyte w:\newline
			\texttt{\textbackslash
				TYPO3\textbackslash
				CMS\textbackslash
				Core\textbackslash
				Versioning\textbackslash
				VersionState}\newline

			...a nastepnie jako:\newline
			\texttt{new VersionState(VersionState::DEFAULT\_STATE);}

	\end{itemize}

\end{frame}

% ------------------------------------------------------------------------------

