% ------------------------------------------------------------------------------
% TYPO3 CMS 6.2 LTS - What's New - Chapter "Application Programming Interface" (Russian Version)
%
% @author	Andrey Aksenov <aksenovaa@bk.ru>
% @license	Creative Commons BY-NC-SA 3.0
% @link		http://typo3.org/download/release-notes/whats-new/
% @language	Russian
% ------------------------------------------------------------------------------
% Chapter: Application Programming Interface
% ------------------------------------------------------------------------------

\section{Application Programming Interface}
\begin{frame}[fragile]
	\frametitle{Application Programming Interface}

	\begin{center}\huge{Глава 7:}\end{center}
	\begin{center}\huge{\color{typo3darkgrey}\textbf{Application Programming Interface (API)}}\end{center}

\end{frame}

% ------------------------------------------------------------------------------
% Hook: tsfe::checkEnableFields
% ------------------------------------------------------------------------------
% http://forge.typo3.org/issues/48981

\begin{frame}[fragile]
	\frametitle{Application Programming Interface}
	\framesubtitle{Hook: \texttt{tsfe::checkEnableFields}}

	\begin{itemize}
		\item В TYPO3 < 6.2, "\emph{Включая подстраницы}" невозможно использовать в расширениях, обеспечивающих дополнительные правила для видимости страниц\newline
			\small(список полей жестко определён в \texttt{tsfe::checkEnableFields()})\normalsize

		\item В TYPO3 >= 6.2, новая уловка (hook) допускает создание дополнительных правил для видимости страниц в расширениях, при условии включения флага "Включая подстраницы" на родительской странице.
		\item Class:\newline
			\smaller
				\texttt{\textbackslash
					TYPO3\textbackslash
					CMS\textbackslash
					Frontend\textbackslash
					Controller\textbackslash
					TypoScriptFrontendController}\normalsize

			\lstset{
				basicstyle=\smaller\ttfamily
			}

			\begin{lstlisting}
				$GLOBALS['TYPO3_CONF_VARS']['SC_OPTIONS']
				  ['tslib/class.tslib_fe.php']['hook_checkEnableFields']
			\end{lstlisting}

	\end{itemize}

\end{frame}

% ------------------------------------------------------------------------------
% Hook: checkFlexFormValue in DataHandler
% ------------------------------------------------------------------------------
% http://forge.typo3.org/issues/49699

\begin{frame}[fragile]
	\frametitle{Application Programming Interface}
	\framesubtitle{Hook: \texttt{checkFlexFormValue} в DataHandler}

	\begin{itemize}
		\item В TYPO3 < 6.2, при обновлении значений во Flexform, нет проверки фактического удаления значения из базы данных
		\item Это становиться проблемой при, например, сохранении переключаемых действий контроллера - switchable controller actions (Extbase) во Flexform: старое действие, которого больше может и не быть, должно быть удалено вручную

		\item В TYPO3 >= 6.2, новая уловка (hook) позволяет проверить старые данные во Flexform прямо перед обновлением новых значений
		\item Class:\newline
			\smaller
				\texttt{\textbackslash
					TYPO3\textbackslash
					CMS\textbackslash
					Core\textbackslash
					DataHandling\textbackslash
					DataHandler}\normalsize

			\lstset{
				basicstyle=\smaller\ttfamily
			}

			\begin{lstlisting}
				$GLOBALS['TYPO3_CONF_VARS']['SC_OPTIONS']
				  ['t3lib/class.t3lib_tcemain.php']['checkFlexFormValue']
			\end{lstlisting}

		\item Method:\newline
			\smaller
				\texttt{checkFlexFormValue\_beforeMerge()}

	\end{itemize}

\end{frame}

% ------------------------------------------------------------------------------
% Hook to customize header in module "Web > Page"
% ------------------------------------------------------------------------------
% http://forge.typo3.org/issues/52579

\begin{frame}[fragile]
	\frametitle{Application Programming Interface}
	\framesubtitle{Hook для настройки header}

	\begin{itemize}
		\item В TYPO3 >= 6.2, новая уловка (hook) позволяет изменять заголовок страницы в модуле Страница (Модуль: "Веб > Страница")
		\item Он вызывается перед формированием страницы
		\item Класс:\newline
			\smaller
				\texttt{\textbackslash
					TYPO3\textbackslash
					CMS\textbackslash
					Backend\textbackslash
					Controller\textbackslash
					PageLayoutController}\normalsize

			\lstset{
				basicstyle=\smaller\ttfamily
			}

			\begin{lstlisting}
				$GLOBALS['TYPO3_CONF_VARS']['SC_OPTIONS']
				  ['cms/layout/db_layout.php']['drawHeaderHook']
			\end{lstlisting}

		\item Method:\newline
			\smaller
				\texttt{callUserFunction()}

	\end{itemize}

\end{frame}

% ------------------------------------------------------------------------------
% IRRE: Provide default values for created records
% ------------------------------------------------------------------------------
% http://forge.typo3.org/issues/46124

\begin{frame}[fragile]
	\frametitle{Application Programming Interface}
	\framesubtitle{IRRE: значения по умолчанию для создаваемых записей}

	\begin{itemize}
		\item Новый параметр TCA позволяет настраивать поля "inline"
		\item Ключ \texttt{foreign\_record\_defaults} позволяет установить значения (по умолчанию) во вновь создаваемых записях

			\begin{lstlisting}
				'config' => array(
				  'type' => 'inline',
				  'foreign_table' => 'tt_content',
				  'foreign_record_defaults' => array(
				    'CType' => 'image'
				  ),
				)
			\end{lstlisting}

			\small
				Пример: вновь создаваемые \texttt{tt\_content} элементы указанного поля IRRE по умолчанию будут \textbf{Изображениями}. Редактор сможет установить для них другой тип перед сохранением.
			\normalsize
	\end{itemize}

\end{frame}

% ------------------------------------------------------------------------------
% Workspaces
% ------------------------------------------------------------------------------
% http://forge.typo3.org/issues/46124

\begin{frame}[fragile]
	\frametitle{Application Programming Interface}
	\framesubtitle{Рабочие области (1)}

	\begin{itemize}
		\item В TYPO3 < 6.2, модуль "Workspaces" можно дополнить только переназначив компоненты PHP и JavaScript
		\item В TYPO3 >= 6.2, теперь можно дополнить определение и поведение выводимых в модуле столбцов
		\item Несколько примеров на последующих слайдах...
	\end{itemize}

\end{frame}

% ------------------------------------------------------------------------------
% Workspaces
% ------------------------------------------------------------------------------
% http://forge.typo3.org/issues/46124

\begin{frame}[fragile]
	\frametitle{Application Programming Interface}
	\framesubtitle{Рабочие области (2)}

	\lstset{
		basicstyle=\tiny\ttfamily
	}

	Пример (файл \texttt{ext\_localconf.php}):

	\begin{lstlisting}
		$GLOBALS['TYPO3_CONF_VARS']['SC_OPTIONS']
		  ['t3lib/class.t3lib_tcemain.php']['processCmdmapClass']['workspaces_logger'] =
		  'Vendor\\WorkspacesLogger\\Hook\\DataHandlerHook';
	\end{lstlisting}

	Example (file \texttt{ext\_tables.php}):

	\begin{lstlisting}
		\TYPO3\CMS\Workspaces\Service\AdditionalColumnService::getInstance()->register(
		  'WorkspacesLogger_StageChange',
		  'Vendor\\WorkspacesLogger\\DataProvider'
		);

		\TYPO3\CMS\Workspaces\Service\AdditionalResourceService::getInstance()->addJavaScriptResource(
		  'WorkspacesLogger',
		  'EXT:myextension/Resources/Public/JavaScript/StageChange.js'
		);
	\end{lstlisting}

\end{frame}

% ------------------------------------------------------------------------------
% Workspaces
% ------------------------------------------------------------------------------
% http://forge.typo3.org/issues/46124

\begin{frame}[fragile]
	\frametitle{Application Programming Interface}
	\framesubtitle{Рабочие области (3)}

	Пример (файл \texttt{Vendor\textbackslash
		WorkspacesLogger\textbackslash
		Hook\textbackslash
		DataHandlerHook}):

	\lstset{
		basicstyle=\tiny\ttfamily
	}

	\begin{lstlisting}
		<?php
		namespace Vendor\WorkspacesLogger\Hook;
		use TYPO3\CMS\Core\SingletonInterface;

		class DataHandlerHook implements SingletonInterface {

		  const TABLE_Name = 'tx_workspaceslogger_event';
		  const EVENT_SetStage = 91;

		  /**
		   * hook that is called when no prepared command was found
		   */
		  public function processCmdmap($command, $table, $id, $value, &$commandIsProcessed,
		    \TYPO3\CMS\Core\DataHandling\DataHandler $tcemainObj) {
		    ...
		    $action = (string) $value['action'];
		    if ($command === 'version' && $action === 'setStage' && $commandIsProcessed) {
		      ...
		    }
		  }
		}
	\end{lstlisting}

\end{frame}

% ------------------------------------------------------------------------------
% PSR-3 compatible Logger
% ------------------------------------------------------------------------------
% http://forge.typo3.org/issues/48880

\begin{frame}[fragile]
	\frametitle{Application Programming Interface}
	\framesubtitle{PSR-3 совместимое журналирование}

	\begin{itemize}
		\item API журналирования TYPO3 CMS 6.2 теперь совместимо с PSR-3
		\item PSR-3 призвано стать стандартов журналирования в PHP (стандарт PHP Framework Interop Group)

		\item Основные цели PSR-3:
			"\emph{возможность получать объекты LoggerInterface библиотекам и упрощенная стандартизированная запись в них.}"

		\item Интерфейс журналирования содержит сокращенные методы ведения журнала, вроде\newline
			\texttt{debug()}, \texttt{warning()}, \texttt{notice()}, \texttt{alert()}, \texttt{error()}, и т. д.

		\item Дополнительные ресурсы:\newline
			\url{http://www.php-fig.org/psr/3/}

	\end{itemize}

\end{frame}

% ------------------------------------------------------------------------------
% API to CSRF protect Ajax calls in Backend
% (slide added in March 2014)
% ------------------------------------------------------------------------------
% http://forge.typo3.org/issues/56345
% http://forge.typo3.org/issues/56347 (documentation)

\begin{frame}[fragile]
	\frametitle{Application Programming Interface}
	\framesubtitle{CSRF Защищённые вызовы Ajax}

	\lstset{
		basicstyle=\tiny\ttfamily
	}

	\begin{itemize}
		\item Вызовы Ajax во внутреннем интерфейсе TYPO3 можно защитить против CSRF (\textit{cross-site request forgery}) посредством регистрации свои обработчиков

			\begin{lstlisting}
				\TYPO3\CMS\Core\Utility\ExtensionManagementUtility::registerAjaxHandler(
				  'TxMyExt::process',
				  '\Vendor\MyExt\AjaxHandler->process'
				);
			\end{lstlisting}

		\item URL для данного Ajax ID содержит маркер защиты CSRF, который проверяется в \texttt{ajax.php}
		dispatcher

			\begin{lstlisting}
				$ajaxUrl = \TYPO3\CMS\Core\Utility\BackendUtility::getAjaxUrl('TxMyExt::process');
			\end{lstlisting}

		\item Эти параметры затем могут быть доступны в контексте JavaScript страницы

			\begin{lstlisting}
				var ajaxUrl = TYPO3.settings.MyExt.ajaxUrl;
			\end{lstlisting}

	\end{itemize}

\end{frame}

% ------------------------------------------------------------------------------
% Miscellaneous
% ------------------------------------------------------------------------------
% http://forge.typo3.org/issues/49144 (MathUtility: Add canBeInterpretedAsFloat)
% http://forge.typo3.org/issues/52707 (Introduce a PHP Enumeration type)
% http://forge.typo3.org/issues/52762 (Add type converter for core types like Enumeration)

\begin{frame}[fragile]
	\frametitle{Application Programming Interface}
	\framesubtitle{Дополнительно}

	\begin{itemize}
		\item Новый метод \texttt{canBeInterpretedAsFloat()} в классе: \texttt{MathUtility}\newline
			\small(аналог: \texttt{canBeInterpretedAsInteger()})\normalsize
		\item Новый тип перечисления (не связанный со второстепенными модулями PHP):\newline
			\texttt{\textbackslash
				TYPO3\textbackslash
				CMS\textbackslash
				Core\textbackslash
				Type\textbackslash
				Enumeration}\newline

			Пример использования в:\newline
			\texttt{\textbackslash
				TYPO3\textbackslash
				CMS\textbackslash
				Core\textbackslash
				Versioning\textbackslash
				VersionState}\newline

			...а затем как:\newline
			\texttt{new VersionState(VersionState::DEFAULT\_STATE);}

	\end{itemize}

\end{frame}

% ------------------------------------------------------------------------------

