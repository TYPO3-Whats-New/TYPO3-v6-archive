% ------------------------------------------------------------------------------
% TYPO3 CMS 6.2 LTS - What's New - Chapter "TypoScript" (German Version)
%
% @author	Michael Schams <schams.net>
% @license	Creative Commons BY-NC-SA 3.0
% @link		http://typo3.org/download/release-notes/whats-new/
% @language	German
% ------------------------------------------------------------------------------
% Chapter: TypoScript
% ------------------------------------------------------------------------------

\section{TSconfig \& TypoScript}
\begin{frame}[fragile]
	\frametitle{TSconfig \& TypoScript}

	\begin{center}\huge{Kapitel 4:}\end{center}
	\begin{center}\huge{\color{typo3darkgrey}\textbf{TSconfig \& TypoScript}}\end{center}

\end{frame}

% ------------------------------------------------------------------------------
% Include TypoScript
% ------------------------------------------------------------------------------
% http://forge.typo3.org/issues/34621

\begin{frame}[fragile]
	\frametitle{TSconfig \& TypoScript}
	\framesubtitle{TypoScript inkludieren}

	\begin{itemize}
		\item Sämtliche Dateien eines Verzeichnisses (rekursiv) inkludieren:

			\lstinline!<INCLUDE_TYPOSCRIPT: source="DIR:directory">!
			\lstinline!<INCLUDE_TYPOSCRIPT: source="DIR:EXT:myextension/res/setup">!

		\item Reihenfolge der Inkludierung:\newline
			alphabetisch, zuerst Dateien, dann Verzeichnisse
		\item Mit der Option \texttt{extensions="..."} werden nur Dateien mit dieser Endung berücksichtigt

			\lstinline!<INCLUDE_TYPOSCRIPT: source="DIR:directory" extensions="ts">!

		\item Standardmäßig werden nur ts-, t3-, t3s-, t3c-, txt-Dateien inkludiert
		\item Diese Liste ist im Install Tool konfigurierbar:\newline
			\texttt{\$TYPO3\_CONF\_VARS['SYS']['tsfile\_ext']}
	\end{itemize}

\end{frame}

% ------------------------------------------------------------------------------
% Include TypoScript
% ------------------------------------------------------------------------------
% http://forge.typo3.org/issues/52018

\begin{frame}[fragile]
	\frametitle{TSconfig \& TypoScript}
	\framesubtitle{TypoScript inkludieren}

	\begin{itemize}
		\item \texttt{INCLUDE\_TYPOSCRIPT} unterstützt nun auch relative Pfade
		\item Pfad der ersten Inkludierung \textbf{muss} absolut sein
		\item \texttt{./} verweist auf das aktuelle Verzeichnis der letzten Inkludierung
		\item \texttt{../} verweist auf das Elternverzeichnis des letzten Inkludierung
		\item Zum Beispiel:

			\lstinline!<INCLUDE_TYPOSCRIPT: source="FILE:directory/typoscript/setup.ts">!
			\lstinline!<INCLUDE_TYPOSCRIPT: source="FILE:./filename.ts">!
			\lstinline!<INCLUDE_TYPOSCRIPT: source="FILE:../filename.ts">!
			\lstinline!<INCLUDE_TYPOSCRIPT: source="FILE:../directory/filename.ts">!

	\end{itemize}

\end{frame}

% ------------------------------------------------------------------------------
% stdWrap for strPad
% ------------------------------------------------------------------------------
% http://forge.typo3.org/issues/43604

\begin{frame}[fragile]
	\frametitle{TSconfig \& TypoScript}
	\framesubtitle{strPad}

	\begin{itemize}
		\item Option \texttt{stdWrap} wurde zu \texttt{strPad} Eigenschaften hinzugefügt

			\begin{lstlisting}
				page = PAGE
				page.10 = TEXT
				page.10 {
				  value = Hello World!
				  strPad {
				    length = 5
				    length {
				      current = 1
				      setCurrent.data = TSFE:page|uid
				      setCurrent.wrap = | + 80
				      prioriCalc = 1
				    }
				    padWith = .
				  }
				}
			\end{lstlisting}

	\end{itemize}

\end{frame}

% ------------------------------------------------------------------------------
% stdWrap for _DEFAULT_PI_VARS
% ------------------------------------------------------------------------------
% http://forge.typo3.org/issues/22045
% http://forge.typo3.org/issues/49314

\begin{frame}[fragile]
	\frametitle{TSconfig \& TypoScript}
	\framesubtitle{\_DEFAULT\_PI\_VARS}

	\begin{itemize}
		\item \texttt{stdWrap} wurde zu \texttt{\_DEFAULT\_PI\_VARS} hinzugefügt
		\item \texttt{\_DEFAULT\_PI\_VARS} dienen dazu, Default-Werte für piVars (zu einer Extension gehörende GET/POST-Variablen) zu setzen

		\item TYPO3 vor Version 6.2:
			\begin{lstlisting}
				plugin.tt_news._DEFAULT_PI_VARS {
				  year = 2013
				}
			\end{lstlisting}

		\item TYPO3 ab Version 6.2:
			\begin{lstlisting}
				plugin.tt_news._DEFAULT_PI_VARS {
				  year.stdWrap.data = date:Y
				}
			\end{lstlisting}

	\end{itemize}

\end{frame}

% ------------------------------------------------------------------------------
% Debug Register and Page
% ------------------------------------------------------------------------------
% http://forge.typo3.org/issues/49478

\begin{frame}[fragile]
	\frametitle{TSconfig \& TypoScript}
	\framesubtitle{Debug Ausgaben}

	\begin{columns}[T]

		\begin{column}{.6\textwidth}
			\begin{itemize}
				\item Debug Register und Page per TypoScript ausgeben:

					\begin{lstlisting}
						GLOBALS['TSFE']->register
						GLOBALS['TSFE']->page
					\end{lstlisting}

				\item Zum Beispiel:

					\begin{lstlisting}
						10 = LOAD_REGISTER
						10.variable = value
					\end{lstlisting}

					\begin{lstlisting}
						20 = TEXT
						20.data = debug:register
					\end{lstlisting}

					\begin{lstlisting}
						30 = TEXT
						30.data = debug:page
					\end{lstlisting}

			\end{itemize}
		\end{column}

		\begin{column}{.4\textwidth}
			\begin{figure}\vspace*{-0.4cm}
				\includegraphics[width=0.6\linewidth]{Images/TypoScript/DebugRegisterAndPage.png}
			\end{figure}
		\end{column}

	\end{columns}

\end{frame}

% ------------------------------------------------------------------------------
% File Links: "register:titleText" and "register:altText"
% ------------------------------------------------------------------------------
% http://forge.typo3.org/issues/44182

\begin{frame}[fragile]
	\frametitle{TSconfig \& TypoScript}
	\framesubtitle{Filelinks}

	\begin{itemize}
		\item Filelinks bieten eine Beschreibung, einen Titel und einen alternativen Titel für jede Datei.
			Auf sämtliche drei Werte kann nun per Register zugegriffen werden:

			\begin{lstlisting}
				register:description
				register:titleText
				register:altText
			\end{lstlisting}

		\item Zum Beispiel:

			\begin{lstlisting}
				# filelinks
				tt_content.uploads.20 {
				  # link description instead of filename
				  labelStdWrap.data = register:description
				  # output alternative text
				  itemRendering.20.data = register:titleText
				}
			\end{lstlisting}

	\end{itemize}

\end{frame}

% ------------------------------------------------------------------------------
% stdWrap replacement: add optionSplit-support
% ------------------------------------------------------------------------------
% http://forge.typo3.org/issues/42287

\begin{frame}[fragile]
	\frametitle{TSconfig \& TypoScript}
	\framesubtitle{stdWrap function: replacement (1)}

	\begin{itemize}
		\item Option \texttt{replace} der \texttt{stdWrap}-Funktion \texttt{replacement}\newline
			unterstützt nun \texttt{optionSplit}

		\item Beispiel 1:

			\begin{lstlisting}
				10 = TEXT
				10.value = TYPO3_inspires_people_to_share
				10.replacement.10 {
				  search = _
				  replace = 1 || 2 || 3
				  useOptionSplitReplace = 1
				}
			\end{lstlisting}

			Ausgabe:\newline
				\texttt{TYPO31inspires2people3to3share}

	\end{itemize}

\end{frame}

% ------------------------------------------------------------------------------
% stdWrap replacement: add optionSplit-support
% ------------------------------------------------------------------------------
% http://forge.typo3.org/issues/42287

\begin{frame}[fragile]
	\frametitle{TSconfig \& TypoScript}
	\framesubtitle{stdWrap function: replacement (2)}

	\begin{itemize}
		\item Option \texttt{replace} der \texttt{stdWrap}-Funktion \texttt{replacement}\newline
			unterstützt nun \texttt{optionSplit}

		\item Beispiel 2:

			\begin{lstlisting}
				10 = TEXT
				10.value = TYPO3 inspires people to share
				10.replacement.10 {
				  search = #(TYPO3|people|share)#i
				  replace = ${1} CMS || all ${1} || collaborate and ${1}
				  useOptionSplitReplace = 1
				  useRegExp = 1
				}
			\end{lstlisting}

			Ausgabe:\newline
				\texttt{TYPO3 CMS inspires all people to collaborate and share}

	\end{itemize}

\end{frame}

% ------------------------------------------------------------------------------
% Register values in FilesContentObject
% ------------------------------------------------------------------------------
% http://forge.typo3.org/issues/49480

\begin{frame}[fragile]
	\frametitle{TSconfig \& TypoScript}
	\framesubtitle{cObject FILE}

	\begin{itemize}
		\item cObject FILES wurde um die Register \texttt{FILE\_NUM\_CURRENT} und \texttt{FILES\_COUNT} erweitert

		\item Zum Beispiel:

			\lstset{
				basicstyle=\tiny\ttfamily
			}

			\begin{lstlisting}
				10 = FILES
				10 {
				  references {
				    table = tt_news
				    uid.field = uid
				    fieldName = media
				  }
				  renderObj = COA
				  renderObj {
				    10 = TEXT
				    10.value = Renders first file twice
				    10.if.isFalse.data = register:FILE_NUM_CURRENT
				    20 = TEXT
				    20.value = file {register:FILE_NUM_CURRENT} of {register:FILES_COUNT}
				    20.insertData = 1
				  }
				}
			\end{lstlisting}

	\end{itemize}

\end{frame}

% ------------------------------------------------------------------------------
% Category Menu In TypoScript
% ------------------------------------------------------------------------------
% http://forge.typo3.org/issues/51161

\begin{frame}[fragile]
	\frametitle{TSconfig \& TypoScript}
	\framesubtitle{Menü-Typ: Categories}

	\begin{itemize}
		\item Es gibt nun die Möglichkeit, ein Kategorien-Menü in TypoScript zu erstellen

		\item Beispiel:

			\lstset{
				basicstyle=\tiny\ttfamily
			}

			\begin{lstlisting}
				page.20 = HMENU
				page.20 {
				  special = categories
				  special {
				    # comma-separated list of categories
				    value = 1
				    # sort by title (stdWrap)
				    sorting = title
				    # sorting "asc" or "desc" (stdWrap)
				    order = desc
				    1 = TMENU
				    1.NO {
				      allWrap = <li> | </li>
				    }
				  }
				}
			\end{lstlisting}

	\end{itemize}

\end{frame}

% ------------------------------------------------------------------------------
% Category Menu In TypoScript
% ------------------------------------------------------------------------------
% http://forge.typo3.org/issues/51782

\begin{frame}[fragile]
	\frametitle{TSconfig \& TypoScript}
	\framesubtitle{CSS and JavaScript files}

	\begin{itemize}
		\item \texttt{splitChar} kann nun auf \texttt{allWrap} Eigenschaften angewandt werden
		\item "wrap" funktioniert nun wie die \texttt{stdWrap.wrap} Methode
		\item Standard \texttt{splitChar}-Zeichen ist das Pipe-Symbol: \texttt{|}
		\item Diese Erweiterung gilt für:

			\begin{itemize}
				\item \texttt{includeCSS}
				\item \texttt{includeJSlibs}
				\item \texttt{includeJSFooterlibs}
				\item \texttt{includeJS}
				\item \texttt{includeJSFooter}
			\end{itemize}

	\end{itemize}

\end{frame}


% ------------------------------------------------------------------------------
% Conditions: userFunc Accepts Multiple Arguments
% ------------------------------------------------------------------------------
% http://forge.typo3.org/issues/47159

\begin{frame}[fragile]
	\frametitle{TSconfig \& TypoScript}
	\framesubtitle{Conditions}

	\begin{itemize}
		\item Condition \texttt{userFunc} erlaubt die Angabe von mehreren Argumenten

		\item TYPO3 vor Version 6.2:
			\begin{lstlisting}
				[userFunc = user_function(argument1)]
			\end{lstlisting}

		\item TYPO3 ab Version 6.2:
			\begin{lstlisting}
				[userFunc = user_function(argument1, argument2, ...)]
			\end{lstlisting}

		\item Zum Beispiel:

			\lstinline![userFunc = user_match(checkSubnet, 192.168)]!

			\begin{lstlisting}
				function user_match($command, $subnet) {
				  switch($command) {
				    case 'checkSubnet':
				      if (strstr(getenv('REMOTE_ADDR'), $subnet)) { ... }
				  }
				}
			\end{lstlisting}

	\end{itemize}

\end{frame}

% ------------------------------------------------------------------------------
% Conditions: Application Context
% ------------------------------------------------------------------------------
% http://forge.typo3.org/issues/39441
% http://forge.typo3.org/issues/50132

\begin{frame}[fragile]
	\frametitle{TSconfig \& TypoScript}
	\framesubtitle{Conditions}

	\begin{itemize}
		\item Application Context kann nun auch per Condition abgefragt werden
		\item Wildcards "\texttt{+}", "\texttt{*}", sowie reguläre Ausdrücke werden unterstützt
		\item Zum Beispiel:

			\lstset{
				basicstyle=\tiny\ttfamily
			}

			\begin{lstlisting}
				[applicationContext = Development/Debugging, Development/Profiling]
				  # TYPO3 site in development stage
				[global]

				[applicationContext = Production*]
				  # TYPO3 site in production stage
				  # for example "Production/Live" or "Production/Staging"
				[global]

				[applicationContext = /^TestServer\d+$/]
				  # TYPO3 site on TestServer1 or TestServer2 or TestServer3, etc.
				[global]
			\end{lstlisting}

	\end{itemize}

\end{frame}

% ------------------------------------------------------------------------------
% Conditions: IP Supports Keyword devIP
% ------------------------------------------------------------------------------
% http://forge.typo3.org/issues/50092

\begin{frame}[fragile]
	\frametitle{TSconfig \& TypoScript}
	\framesubtitle{Conditions}

	\begin{itemize}

		\item Mittels \texttt{devIP} kann auf die im Install Tool eingetragene \texttt{devIpMask} geprüft werden
		\item Zum Beispiel:

			\begin{lstlisting}
				[IP = devIP]
				  page.10 = TEXT
				  page.10.value = Hello Developer!
				[global]
			\end{lstlisting}

	\end{itemize}

\end{frame}

% ------------------------------------------------------------------------------
% Records Without Default Translation
% ------------------------------------------------------------------------------
% http://forge.typo3.org/issues/24005

\begin{frame}[fragile]
	\frametitle{TSconfig \& TypoScript}
	\framesubtitle{Datensätze ohne Standard Sprache}

	\begin{itemize}

		\item Neue Option \texttt{includeRecordsWithoutDefaultTranslation}
			ermittelt Datensätze, die keine Entsprechung in der Default-Sprache haben,
			aber deren \texttt{languageField} der aktuellen Sprache entspricht

		\item Zum Beispiel:

			\begin{lstlisting}
				pageContent = CONTENT
				pageContent {
				  table = tt_content
				  select.includeRecordsWithoutDefaultTranslation = 1
				  ...
				}
			\end{lstlisting}

	\end{itemize}

\end{frame}

% ------------------------------------------------------------------------------
% cObject FILES: begin/maxItems options
% ------------------------------------------------------------------------------
% http://forge.typo3.org/issues/52632

\begin{frame}[fragile]
	\frametitle{TSconfig \& TypoScript}
	\framesubtitle{cObject FILES}

	\begin{itemize}

		\item cObject FILES verfügt nun über die Eigenschaften \texttt{begin} und \texttt{maxItems}

		\item Zum Beispiel:

			\lstset{
				basicstyle=\tiny\ttfamily
			}

			\begin{lstlisting}
				page.10 = FILES
				page.10 {
				  references {
				    table = pages
				    uid.data = page:uid
				    fieldName = media
				  }

				  # retrieve up to 5 files, beginning at the first (0):
				  begin = 0
				  maxItems = 5

				  renderObj = TEXT
				  renderObj {
				    data = file:current:size
				    wrap = <p>File size:<strong>|</strong></p>
				  }
				}
			\end{lstlisting}

	\end{itemize}

\end{frame}

% ------------------------------------------------------------------------------
% Exclude doktypes From Pagetree
% ------------------------------------------------------------------------------
% http://forge.typo3.org/issues/49279
% http://forge.typo3.org/issues/49356

\begin{frame}[fragile]
	\frametitle{TSconfig \& TypoScript}
	\framesubtitle{Seitenbaum manipulieren}

	\begin{itemize}

		\item Bestimmte Doktypes können vom Seitenbaum ausgeschlossen werden
		\item Jenes erfolgt über die Einstellung \texttt{excludeDoktypes} in der UserTSconfig und ist somit Benutzer und Gruppen-spezifisch
		\item Zum Beispiel:

			\begin{lstlisting}
				# exclude "folder" pages
				options.pageTree.excludeDoktypes = 254

				# exclude "folder" and "standard" pages
				options.pageTree.excludeDoktypes = 254,1
			\end{lstlisting}

	\end{itemize}

\end{frame}

% ------------------------------------------------------------------------------
% Hide Modules In Backend
% ------------------------------------------------------------------------------

\begin{frame}[fragile]
	\frametitle{TSconfig \& TypoScript}
	\framesubtitle{Module im BE ausblenden}

	\begin{itemize}

		\item Man kann nun einzelne Module im Backend ausblenden
		\item Der Zugriff darauf wird allerdings nicht eingeschränkt\newline
			\small(hierfür verwendet man nach wie vor die Berechtigungseinstellungen für BE-Benutzer und -Gruppen)\normalsize
		\item Zum Beispiel:

			\lstinline!options.hideModules = file, help!
			\lstinline!options.hideModules.web := addToList(func,info)!
			\lstinline!options.hideModules.system = BelogLog!

	\end{itemize}

\end{frame}

% ------------------------------------------------------------------------------
% Alternative Domain For Preview
% ------------------------------------------------------------------------------
% http://forge.typo3.org/issues/30889

\begin{frame}[fragile]
	\frametitle{TSconfig \& TypoScript}
	\framesubtitle{Domain für Vorschau}

	\begin{itemize}

		\item Im PageTSconfig ist es nun möglich, eine Domain anzugeben, die für die Anzeige verwendet wird
		\item Jenes ist beispielsweise für Multi-Domain-Sites nützlich
		\item Zum Beispiel:

			\lstinline!TCEMAIN.viewDomain = example.com!

	\end{itemize}

\end{frame}

% ------------------------------------------------------------------------------
% Conditions in Backend Layouts
% ------------------------------------------------------------------------------
% http://forge.typo3.org/issues/47588

\begin{frame}[fragile]
	\frametitle{TSconfig \& TypoScript}
	\framesubtitle{Conditions in Backend-Layouts}

	\begin{itemize}

		\item Innerhalb von Backend-Layouts können nun auch Conditions verwendet werden

			\lstset{
				basicstyle=\tiny\ttfamily
			}

			\begin{lstlisting}
				backend_layout {
				  colCount = 2
				  rowCount = 1
				  rows {
				    1 {
				      columns {
				        1.name = Main
				        1.colPos = 0
				        2.name = Right
				        2.colPos = 1
				      }
				    }
				  }
				}

				[PIDupinRootline = 123]
				  # remove right column in branch of page ID 123
				  backend_layout.rows.1.columns.2 >
				[global]
			\end{lstlisting}

	\end{itemize}

\end{frame}

% ------------------------------------------------------------------------------
% Miscellaneous
% ------------------------------------------------------------------------------
% http://forge.typo3.org/issues/34597 (showForgotPassword)
% http://forge.typo3.org/issues/50138 (showForgotPassword)
% http://forge.typo3.org/issues/19732 (Content-Length)
% http://forge.typo3.org/issues/16386 (Indexed Search)

\begin{frame}[fragile]
	\frametitle{TSconfig \& TypoScript}
	\framesubtitle{Diverses}

	\begin{itemize}

		\item Der "Passwort vergessen" Link kann nun über die Option
			\texttt{showForgotPassword} gezielt ein- und ausgeschaltet werden.\newline
			Jenes ist sinnvoll, wenn sich mehrere Login-Formulare auf einer Seite befinden.

		\item HTTP Antwort beinhaltet standardmäßig den Header \texttt{Content-length}

			\begin{itemize}
				\item Jenes läßt sich durch folgende Option beeinflussen:\newline
					\texttt{config.enableContentLengthHeader}
				\item Beschleunigt das Rendering, wenn "Pipelining" im Apache aktiviert ist
			\end{itemize}

		\item Ergebnisliste der EXT:indexed\_search hat nun \texttt{stdWrap}-Eigenschaften\newline
			(Option: \texttt{plugin.tx\_indexedsearch.resultlist\_stdWrap})

	\end{itemize}

\end{frame}

% ------------------------------------------------------------------------------

