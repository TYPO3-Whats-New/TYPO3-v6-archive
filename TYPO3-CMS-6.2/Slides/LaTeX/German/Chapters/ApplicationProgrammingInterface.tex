% ------------------------------------------------------------------------------
% TYPO3 CMS 6.2 LTS - What's New - Chapter "Application Programming Interface" (German Version)
%
% @author	Michael Schams <schams.net>
% @license	Creative Commons BY-NC-SA 3.0
% @link		http://typo3.org/download/release-notes/whats-new/
% @language	German
% ------------------------------------------------------------------------------
% Chapter: Application Programming Interface
% ------------------------------------------------------------------------------

\section{Application Programming Interface}
\begin{frame}[fragile]
	\frametitle{Application Programming Interface}

	\begin{center}\huge{Kapitel 7:}\end{center}
	\begin{center}\huge{\color{typo3darkgrey}\textbf{Application Programming Interface (API)}}\end{center}

\end{frame}

% ------------------------------------------------------------------------------
% Hook: tsfe::checkEnableFields
% ------------------------------------------------------------------------------
% http://forge.typo3.org/issues/48981

\begin{frame}[fragile]
	\frametitle{Application Programming Interface}
	\framesubtitle{Hook: \texttt{tsfe::checkEnableFields}}

	\begin{itemize}
		\item In TYPO3 CMS vor 6.2 kann das Feature "auf Unterseiten ausdehnen" (\emph{extend to subpages}) nicht in eigenen Extensions verwendet werden, um zusätzliche Regeln für die Sichtbarkeit von Seiten bereitzustellen\newline
			\smaller(Liste der zu prüfenden Felder ist hard-coded in \texttt{tsfe::checkEnableFields()})\normalsize
		\item In TYPO3 CMS ab 6.2 kann durch einen neuen Hook diese Liste beliebig erweitert werden

		\item Klasse:\newline
			\smaller
				\texttt{\textbackslash
					TYPO3\textbackslash
					CMS\textbackslash
					Frontend\textbackslash
					Controller\textbackslash
					TypoScriptFrontendController}\normalsize

			\lstset{
				basicstyle=\smaller\ttfamily
			}

			\begin{lstlisting}
				$GLOBALS['TYPO3_CONF_VARS']['SC_OPTIONS']
				  ['tslib/class.tslib_fe.php']['hook_checkEnableFields']
			\end{lstlisting}

	\end{itemize}

\end{frame}

% ------------------------------------------------------------------------------
% Hook: checkFlexFormValue in DataHandler
% ------------------------------------------------------------------------------
% http://forge.typo3.org/issues/49699

\begin{frame}[fragile]
	\frametitle{Application Programming Interface}
	\framesubtitle{Hook: \texttt{checkFlexFormValue} im DataHandler}

	\begin{itemize}
		\item In TYPO3 CMS vor 6.2 wird vor dem Speichern von FlexForm-Daten nicht geprüft, ob der Wert selbst bereits aus der Datenbank gelöscht wurde
		\item Jenes ist problematisch, wenn SwitchableControllerActions (Extbase) verwendet werden: alte Actions, die es nicht mehr gibt, müssen manuell aus der Datenbank entfernt werden
		\item In TYPO3 CMS ab 6.2 können mit einem neuen Hook FlexForm-Daten vor dem Speichern verändert werden
		\item Klasse:\newline
			\smaller
				\texttt{\textbackslash
					TYPO3\textbackslash
					CMS\textbackslash
					Core\textbackslash
					DataHandling\textbackslash
					DataHandler}\normalsize

			\lstset{
				basicstyle=\smaller\ttfamily
			}

			\begin{lstlisting}
				$GLOBALS['TYPO3_CONF_VARS']['SC_OPTIONS']
				  ['t3lib/class.t3lib_tcemain.php']['checkFlexFormValue']
			\end{lstlisting}

		\item Methode:\newline
			\smaller
				\texttt{checkFlexFormValue\_beforeMerge()}

	\end{itemize}

\end{frame}

% ------------------------------------------------------------------------------
% Hook to customize header in module "Web > Page"
% ------------------------------------------------------------------------------
% http://forge.typo3.org/issues/52579

\begin{frame}[fragile]
	\frametitle{Application Programming Interface}
	\framesubtitle{Hook für den Header im PageLayoutController}

	\begin{itemize}
		\item Mit diesem Hook kann der Header einer Seite im Modul "Web > Page" erweitert werden
		\item Der Hook wird ausgeführt, bevor der Seiteninhalt gerendert wird
		\item Klasse:\newline
			\smaller
				\texttt{\textbackslash
					TYPO3\textbackslash
					CMS\textbackslash
					Backend\textbackslash
					Controller\textbackslash
					PageLayoutController}\normalsize

			\lstset{
				basicstyle=\smaller\ttfamily
			}

			\begin{lstlisting}
				$GLOBALS['TYPO3_CONF_VARS']['SC_OPTIONS']
				  ['cms/layout/db_layout.php']['drawHeaderHook']
			\end{lstlisting}

		\item Methode:\newline
			\smaller
				\texttt{callUserFunction()}

	\end{itemize}

\end{frame}

% ------------------------------------------------------------------------------
% IRRE: Provide default values for created records
% ------------------------------------------------------------------------------
% http://forge.typo3.org/issues/46124

\begin{frame}[fragile]
	\frametitle{Application Programming Interface}
	\framesubtitle{IRRE: Defaults für erzeugte Werte}

	\begin{itemize}
		\item Für IRRE Datensätze gibt es eine neue TCA-Option für "inline" Felder
		\item Über \texttt{foreign\_record\_defaults} können Default-Werte für erzeugte Datensätzen gesetzt werden

			\begin{lstlisting}
				'config' => array(
				  'type' => 'inline',
				  'foreign_table' => 'tt_content',
				  'foreign_record_defaults' => array(
				    'CType' => 'image'
				  ),
				)
			\end{lstlisting}

			\small
				Beispiel oben: \texttt{tt\_content}-Elemente, die für dieses IRRE Feld erstellt wurden, sind standarmäßig \textbf{image}-Inhaltselemente (jedoch können Redakteure diese natürlich vor dem Speichern ändern)
			\normalsize

	\end{itemize}

\end{frame}

% ------------------------------------------------------------------------------
% Workspaces
% ------------------------------------------------------------------------------
% http://forge.typo3.org/issues/46124

\begin{frame}[fragile]
	\frametitle{Application Programming Interface}
	\framesubtitle{Arbeitsumgebung (1)}

	\begin{itemize}
		\item In TYPO3 CMS vor 6.2 kann das Modul Arbeitsumgebung ("Workspaces") nur durch das Überschreiben von PHP und JavaScript Komponenten erweitert werden
		\item In TYPO3 CMS ab 6.2 ist es nun möglich, die Definitionen und das Verhalten der dargestellten Spalten des Moduls zu manipulieren
		\item Einige Beispiele auf den folgenden Slides...
	\end{itemize}

\end{frame}

% ------------------------------------------------------------------------------
% Workspaces
% ------------------------------------------------------------------------------
% http://forge.typo3.org/issues/46124

\begin{frame}[fragile]
	\frametitle{Application Programming Interface}
	\framesubtitle{Arbeitsumgebung (2)}

	\lstset{
		basicstyle=\tiny\ttfamily
	}

	Beispiel (Datei \texttt{ext\_localconf.php}):

	\begin{lstlisting}
		$GLOBALS['TYPO3_CONF_VARS']['SC_OPTIONS']
		  ['t3lib/class.t3lib_tcemain.php']['processCmdmapClass']['workspaces_logger'] =
		  'Vendor\\WorkspacesLogger\\Hook\\DataHandlerHook';
	\end{lstlisting}

	Beispiel (Datei \texttt{ext\_tables.php}):

	\begin{lstlisting}
		\TYPO3\CMS\Workspaces\Service\AdditionalColumnService::getInstance()->register(
		  'WorkspacesLogger_StageChange',
		  'Vendor\\WorkspacesLogger\\DataProvider'
		);

		\TYPO3\CMS\Workspaces\Service\AdditionalResourceService::getInstance()->addJavaScriptResource(
		  'WorkspacesLogger',
		  'EXT:myextension/Resources/Public/JavaScript/StageChange.js'
		);
	\end{lstlisting}

\end{frame}

% ------------------------------------------------------------------------------
% Workspaces
% ------------------------------------------------------------------------------
% http://forge.typo3.org/issues/46124

\begin{frame}[fragile]
	\frametitle{Application Programming Interface}
	\framesubtitle{Arbeitsumgebung (3)}

	Beispiel (Datei \texttt{Vendor\textbackslash
		WorkspacesLogger\textbackslash
		Hook\textbackslash
		DataHandlerHook}):

	\lstset{
		basicstyle=\tiny\ttfamily
	}

	\begin{lstlisting}
		<?php
		namespace Vendor\WorkspacesLogger\Hook;
		use TYPO3\CMS\Core\SingletonInterface;

		class DataHandlerHook implements SingletonInterface {

		  const TABLE_Name = 'tx_workspaceslogger_event';
		  const EVENT_SetStage = 91;

		  /**
		   * hook that is called when no prepared command was found
		   */
		  public function processCmdmap($command, $table, $id, $value, &$commandIsProcessed,
		    \TYPO3\CMS\Core\DataHandling\DataHandler $tcemainObj) {
		    ...
		    $action = (string) $value['action'];
		    if ($command === 'version' && $action === 'setStage' && $commandIsProcessed) {
		      ...
		    }
		  }
		}
	\end{lstlisting}

\end{frame}

% ------------------------------------------------------------------------------
% PSR-3 compatible Logger
% ------------------------------------------------------------------------------
% http://forge.typo3.org/issues/48880

\begin{frame}[fragile]
	\frametitle{Application Programming Interface}
	\framesubtitle{PSR-3 kompatibles Logging}

	\begin{itemize}
		\item Die TYPO3 CMS 6.2 Logging API ist nun PSR-3 kompatibel
		\item PSR-3 soll einen Standard für das Logging in PHP definieren\newline
			(PHP Framework Interop Group)

		\item Das Hauptziel von PSR-3 ist es,
			"\emph{dass Bibliotheken ein LoggerInterface Objekt erhalten und in einer einfachen und universellen Weise Protokoll-Einträge in dieses schreiben}"

		\item LoggerInterface enthält Methoden, wie beispielsweise\newline
			\texttt{debug()}, \texttt{warning()}, \texttt{notice()}, \texttt{alert()}, \texttt{error()}, etc.

		\item Weitere Informationen:\newline
			\url{http://www.php-fig.org/psr/3/}

	\end{itemize}

\end{frame}

% ------------------------------------------------------------------------------
% API to CSRF protect Ajax calls in Backend
% ------------------------------------------------------------------------------
% http://forge.typo3.org/issues/56345
% http://forge.typo3.org/issues/56347 (documentation)

\begin{frame}[fragile]
	\frametitle{Application Programming Interface}
	\framesubtitle{CSRF-geschützte Ajax Calls}

	\lstset{
		basicstyle=\tiny\ttfamily
	}

	\begin{itemize}
		\item Ajax Calls im TYPO3 Backend können nun wirksam gegen CSRF (\textit{Cross-Site Request Forgery}) geschützt werden, indem deren Handler registriert wird

			\begin{lstlisting}
				\TYPO3\CMS\Core\Utility\ExtensionManagementUtility::registerAjaxHandler(
				  'TxMyExt::process',
				  '\Vendor\MyExt\AjaxHandler->process'
				);
			\end{lstlisting}

		\item URL für die Ajax ID beinhaltet einen CSRF Token, der im \texttt{ajax.php}-Dispatcher überprüft wird

			\begin{lstlisting}
				$ajaxUrl = \TYPO3\CMS\Core\Utility\BackendUtility::getAjaxUrl('TxMyExt::process');
			\end{lstlisting}

		\item Die Variable kann im JavaScript Kontext auf der Seite verwendet werden

			\begin{lstlisting}
				var ajaxUrl = TYPO3.settings.MyExt.ajaxUrl;
			\end{lstlisting}

	\end{itemize}

\end{frame}

% ------------------------------------------------------------------------------
% Miscellaneous
% ------------------------------------------------------------------------------
% http://forge.typo3.org/issues/49144 (MathUtility: Add canBeInterpretedAsFloat)
% http://forge.typo3.org/issues/52707 (Introduce a PHP Enumeration type)
% http://forge.typo3.org/issues/52762 (Add type converter for core types like Enumeration)

\begin{frame}[fragile]
	\frametitle{Application Programming Interface}
	\framesubtitle{Diverses}

	\begin{itemize}
		\item Neue Methode \texttt{canBeInterpretedAsFloat()}\newline
			in der Klasse \texttt{MathUtility}\newline
			\small(analog zu: \texttt{canBeInterpretedAsInteger()})\normalsize
		\item Neuer Enumeration-Type (ohne die Verwendung von Modulen von Drittanbietern):
			\texttt{\textbackslash
				TYPO3\textbackslash
				CMS\textbackslash
				Core\textbackslash
				Type\textbackslash
				Enumeration}\newline

			Zum Beispiel verwendet in:\newline
			\small
				\texttt{\textbackslash
					TYPO3\textbackslash
					CMS\textbackslash
					Core\textbackslash
					Versioning\textbackslash
					VersionState}
			\normalsize\newline

			...und dann im Code mittels:\newline
			\small
				\texttt{new VersionState(VersionState::DEFAULT\_STATE);}
			
			\normalsize

	\end{itemize}

\end{frame}

% ------------------------------------------------------------------------------

