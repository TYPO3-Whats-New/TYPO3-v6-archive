% ------------------------------------------------------------------------------
% TYPO3 CMS 6.2 LTS - What's New - Chapter "Extbase & Fluid" (German Version)
%
% @author	Michael Schams <schams.net>
% @license	Creative Commons BY-NC-SA 3.0
% @link		http://typo3.org/download/release-notes/whats-new/
% @language	German
% ------------------------------------------------------------------------------
% Chapter: Extbase & Fluid
% ------------------------------------------------------------------------------

\section{Extbase \& Fluid}
\begin{frame}[fragile]
	\frametitle{Extbase \& Fluid}

	\begin{center}\huge{Kapitel 8:}\end{center}
	\begin{center}\huge{\color{typo3darkgrey}\textbf{Extbase \& Fluid}}\end{center}

\end{frame}

% ------------------------------------------------------------------------------
% ObjectManager->getScope()
% ------------------------------------------------------------------------------
% http://forge.typo3.org/issues/48766

\begin{frame}[fragile]
	\frametitle{Extbase \& Fluid}
	\framesubtitle{ObjectManager->getScope()}

	\lstset{
		basicstyle=\tiny\ttfamily
	}

	\begin{itemize}
		\item Über die Methode \texttt{ObjectManager->getScope()} kann nun überprüft werden, ob eine Klasse vom Typ \textbf{Prototype} oder \textbf{Singleton} ist

			\begin{lstlisting}
				/**
				 * @var \TYPO3\CMS\Extbase\Object\ObjectManagerInterface
				 * @inject
				 */
				protected $objectManager;

				$this->objectManager->getScope($propertyTargetClassName) ===
				  \TYPO3\CMS\Extbase\Object\Container\Container::SCOPE_PROTOTYPE

				$this->objectManager->getScope($propertyTargetClassName) ===
				  \TYPO3\CMS\Extbase\Object\Container\Container::SCOPE_SINGLETON
			\end{lstlisting}

	\end{itemize}

\end{frame}

% ------------------------------------------------------------------------------
% Page type for URIs with format
% ------------------------------------------------------------------------------
% http://forge.typo3.org/issues/27498

\begin{frame}[fragile]
	\frametitle{Extbase \& Fluid}
	\framesubtitle{Ermittlung des PageTypes in URIs}

	\lstset{
		basicstyle=\tiny\ttfamily
	}

	\begin{itemize}
		\item Das Attribut \texttt{pageType} ist nicht mehr erforderlich, wenn spezielle Seitentypen gerendert werden

			\smaller\textbf{TYPO3 vor Version 6.2:}\normalsize
			\begin{lstlisting}
				<f:link.action arguments="{blog: blog}" pageType="{settings.plaintextPageType}"
				  format="txt">[plaintext]</f:link.action></li>
			\end{lstlisting}

		\item Über eine neue TypoScript Option \texttt{formatToPageTypeMapping} kann ein globales Mapping erreicht werden

			\begin{lstlisting}
				plugin.tx_myextension {
				  view.formatToPageTypeMapping {
				    txt = 99
				    pdf = 123
				  }
				}
			\end{lstlisting}

			\smaller\textbf{TYPO3 ab Version 6.2:}\normalsize
			\begin{lstlisting}
				<f:link.action arguments="{blog: blog}"
				  format="txt">[plaintext]</f:link.action></li>
			\end{lstlisting}

	\end{itemize}

\end{frame}

% ------------------------------------------------------------------------------
% Object Type Converter
% ------------------------------------------------------------------------------
% http://forge.typo3.org/issues/48548

\begin{frame}[fragile]
	\frametitle{Extbase \& Fluid}
	\framesubtitle{Object Type Converter (1)}

	\begin{itemize}
		\item Der in TYPO3 Flow eingeführte \textbf{Object Type Converter} wurde nach Extbase portiert
		\item Damit ist es möglich, Arrays in nicht-persistente Objekte zu konvertieren
		\item Einige Beispiele auf den folgenden Slides...
	\end{itemize}

\end{frame}

% ------------------------------------------------------------------------------
% Object Type Converter
% ------------------------------------------------------------------------------
% http://forge.typo3.org/issues/48548

\begin{frame}[fragile]
	\frametitle{Extbase \& Fluid}
	\framesubtitle{Object Type Converter (2)}

	\lstset{
		basicstyle=\tiny\ttfamily
	}

	\smaller\textbf{GET request}\normalsize
	\begin{lstlisting}
		http://example.com/index.php?id=299
		  &tx_myextension[action]=list
		  &tx_myextension[controller]=Entity
		  &tx_myextension[demand][title]=foo
		  &tx_myextension[demand][relation]=1
	\end{lstlisting}

	\smaller\textbf{Entity controller: \texttt{initializeListAction()}}\normalsize
	\begin{lstlisting}
		use [Vendor]\myextension\Domain\Dto\Demand;
		public function initializeListAction() {
		  /**
		   * @var PropertyMappingConfiguration $demandConfiguration
		   */
		  $demandConfiguration = $this->arguments['demand']->getPropertyMappingConfiguration();
		  $demandConfiguration->allowAllProperties()->forProperty('relation')->allowAllProperties()->
		    setTypeConverterOption(
		      'TYPO3\\CMS\\Extbase\\Property\\TypeConverter\\PersistentObjectConverter',
		      PersistentObjectConverter::CONFIGURATION_CREATION_ALLOWED,
		      TRUE
		  );
		}
	\end{lstlisting}

\end{frame}

% ------------------------------------------------------------------------------
% Object Type Converter
% ------------------------------------------------------------------------------
% http://forge.typo3.org/issues/48548

\begin{frame}[fragile]
	\frametitle{Extbase \& Fluid}
	\framesubtitle{Object Type Converter (3)}

	\lstset{
		basicstyle=\tiny\ttfamily
	}

	\smaller\textbf{Entity controller: \texttt{listAction()}}\normalsize
	\begin{lstlisting}
		use [Vendor]\myextension\Domain\Dto\Demand;
		/**
		 * @var PropertyMappingConfiguration $demandConfiguration
		 */
		public function listAction(Demand $demand = NULL) {
		  $entities = $this->entityRepository->findAll();
		  $this->view->assign('entities', $entities);
		}
	\end{lstlisting}

	\smaller\textbf{Model: \texttt{[Vendor]\textbackslash myextension\textbackslash Domain\textbackslash Dto\textbackslash Demand.php}}\normalsize
	\begin{lstlisting}
		namespace [Vendor]\myextension\Domain\Dto;
		use [Vendor]\myextension\Domain\Model\Relation;
		class Demand {
		  protected $relation;
		  /**
		   * @param \TYPO3Friends\MapperExample\Domain\Model\Relation $relation
		   */
		  public function setRelation($relation) {
		    $this->relation = $relation;
		  }
		}
	\end{lstlisting}

\end{frame}

% ------------------------------------------------------------------------------
% Chaining of set* functions
% ------------------------------------------------------------------------------
% http://forge.typo3.org/issues/47684

\begin{frame}[fragile]
	\frametitle{Extbase \& Fluid}
	\framesubtitle{Verketten von QuerySettings}

	\lstset{
		basicstyle=\tiny\ttfamily
	}

	\begin{itemize}
		\item Das Chaining (Verketten) von QuerySettings ist nun auch mit den Optionen
			\texttt{setIncludeDeleted} und \texttt{setIgnoreEnableFields} möglich,
			die mit TYPO3 CMS 6.0 eingeführt wurden:

			\begin{lstlisting}
				$query->getQuerySettings()
				  ->setRespectStoragePage(FALSE)
				  ->setRespectSysLanguage(FALSE)
				  ->setIgnoreEnableFields(TRUE)
				  ->setIncludeDeleted(TRUE);
			\end{lstlisting}
	\end{itemize}

\end{frame}

% ------------------------------------------------------------------------------
% returnRawQueryResult as argument
% ------------------------------------------------------------------------------
% http://forge.typo3.org/issues/51145

\begin{frame}[fragile]
	\frametitle{Extbase \& Fluid}
	\framesubtitle{RawQueryResults per Query}

	\lstset{
		basicstyle=\smaller\ttfamily
	}

	\begin{itemize}
		\item Bisher war es möglich, mittels folgendem Statement dafür zu sorgen,
			dass keine Objekte vom QueryManager rekonstruiert werden, sondern
			dass das Query-Ergebnis "roh" zurückgegeben wird:\newline
			\lstinline!$query->getQuerySettings()->setReturnRawQueryResult(TRUE);!
			\newline
		\item Diese Möglichkeit gibt es nicht mehr zentral, sondern man gibt dies
			per Query im \texttt{execute-}Statement mittels \texttt{TRUE} an:\newline
			\lstinline!$query->execute(TRUE);!

	\end{itemize}

\end{frame}

% ------------------------------------------------------------------------------
% Recursive validation
% ------------------------------------------------------------------------------
% http://forge.typo3.org/issues/6893

\begin{frame}[fragile]
	\frametitle{Extbase \& Fluid}
	\framesubtitle{Rekursive Validierung}

	\begin{itemize}
		\item Extbase verwendet nun die von TYPO3 Flow portierte sogenannte
			"\textbf{Rekursive Validierung}"
		\item Dies bedeutet, dass bei Erzeugung von verschachtelten Objekten
			(Objekt-Baum) durch den Property-Mapper auch die inneren Objekte
			validiert werden und nicht nur das äußere Objekt, wie bisher
		\item Zudem ist es nun möglich, leere (optionale) Werte zuzulassen.
			Will man daher explizit erreichen, dass eine Eigenschaft angegeben
			werden muss, so muss der \texttt{NotEmptyValidator} verwendet werden

	\end{itemize}

	\breakingchange

\end{frame}

% ------------------------------------------------------------------------------
% Application Context
% ------------------------------------------------------------------------------
% http://forge.typo3.org/issues/49988

\begin{frame}[fragile]
	\frametitle{Extbase \& Fluid}
	\framesubtitle{Application Context}

	\begin{itemize}
		\item Der aktuelle Application Context (wird als Umgebungsvariable
			\texttt{TYPO3\_CONTEXT} oder im Install Tool gesetzt) kann wie folgt
			abgefragt werden:\newline\newline
			\lstinline!\TYPO3\CMS\Core\Core\Bootstrap::getInstance()->getContext();!
			\lstinline!\TYPO3\CMS\Core\Utility\GeneralUtility::getContext();!

	\end{itemize}

\end{frame}

% ------------------------------------------------------------------------------
% ViewHelper: Image
% ------------------------------------------------------------------------------
% http://forge.typo3.org/issues/47552

\begin{frame}[fragile]
	\frametitle{Extbase \& Fluid}
	\framesubtitle{ViewHelper: Image}

	\begin{itemize}
		\item Im Fluid ViewHelper \textbf{image} ist das Attribut \texttt{title} nun  optional\newline

			\smaller\textbf{Beispiel:}\normalsize\newline
			\lstinline!<f:image src="background.jpg" alt="Text" />!
			\newline

			\smaller\textbf{TYPO3 vor Version 6.2:}\normalsize\newline
			\lstinline!<img src="background.jpg" alt="Text" title="Text" />!
			\newline

			\smaller\textbf{TYPO3 ab Version6.2:}\normalsize\newline
			\lstinline!<img src="background.jpg" alt="Text" />!

	\end{itemize}

\end{frame}

% ------------------------------------------------------------------------------
% ViewHelper: textfield and textarea
% ------------------------------------------------------------------------------
% http://forge.typo3.org/issues/45960
% http://forge.typo3.org/issues/48689 (Added autofocus attribute to textfield and textarea)

\begin{frame}[fragile]
	\frametitle{Extbase \& Fluid}
	\framesubtitle{ViewHelper: textfield und textarea}

	\begin{itemize}
		\item Die beiden ViewHelper \textbf{form.textfield} und \textbf{form.textarea}
			können nun mit den Argumenten \texttt{autofocus} und \texttt{placeholder}
			ausgestattet werden
			\newline\newline
			\smaller\textbf{Beispiel ("placeholder"):}\normalsize
			\begin{lstlisting}
				<f:form.textfield
				  id="powermail_field_{field.marker}"
				  ...
				  placeholder="{field.title -> vh:string.RawAndRemoveXss()}"
				  ...
				  name="field[{field.uid}]"
				  required="{field.mandatory}" />
			\end{lstlisting}

	\end{itemize}

\end{frame}

% ------------------------------------------------------------------------------
% ViewHelper: switch
% ------------------------------------------------------------------------------
% http://forge.typo3.org/issues/48653

\begin{frame}[fragile]
	\frametitle{Extbase \& Fluid}
	\framesubtitle{ViewHelper: switch}

	\lstset{
		basicstyle=\smaller\ttfamily
	}

	\begin{itemize}
		\item Es gibt einen neuen Fluid ViewHelper \textbf{switch}, der der gleichnamigen PHP Funktion ähnelt:
			
			\begin{lstlisting}
				<f:switch expression="{person.gender}">
				  <f:case value="male">Mr.</f:case>
				  <f:case value="female">Mrs.</f:case>
				  <f:case default="TRUE">Mrs. or Mr.</f:case>
				</f:switch>
			\end{lstlisting}

		\item \textbf{\underline{Hinweis:}} die exzessive Verwendung dieses ViewHelpers könnte auf eine schlechte Architektur hindeuten.
			So könnte das Beispiel oben mit den entsprechenden Partials "\texttt{title.male.html}" und "\texttt{title.female.html}" auch wie folgt gelöst werden:

			\begin{lstlisting}
				<f:render partial="title.{person.gender}" />
			\end{lstlisting}

	\end{itemize}

\end{frame}

% ------------------------------------------------------------------------------
% ViewHelper: fileSize
% ------------------------------------------------------------------------------
% http://forge.typo3.org/issues/49139

\begin{frame}[fragile]
	\frametitle{Extbase \& Fluid}
	\framesubtitle{ViewHelper: fileSize}

	\begin{itemize}
		\item Dieser ViewHelper konvertiert eine Byte-Angabe (in Form eines Integer-Wertes) in ein Menschen-lesbares Format
	\end{itemize}

	\begin{columns}[T]

		\begin{column}{.5\textwidth}
			\advance\leftskip+0.8cm
			\smaller
				\textbf{Beispiel 1} (fileSize = 1263616):\newline
				\texttt{fileSize -> f:format.bytes()}\newline
				\newline
				Ausgabe: "1234 KB"
			\normalsize
		\end{column}
		\begin{column}{.5\textwidth}
			\smaller
				\textbf{Beispiel 2} (fileSize = 1263616):\newline
				\texttt{fileSize -> f:format.bytes(}\newline
				\texttt{
				decimals: 2,\newline
				decimalSeparator: '.',\newline
				thousandsSeparator: ','\newline
				)}\newline
				\newline
				Ausgabe: "1,234.00 KB"
			\normalsize
		\end{column}

	\end{columns}

\end{frame}

% ------------------------------------------------------------------------------
% ViewHelper: format.date
% ------------------------------------------------------------------------------

\begin{frame}[fragile]
	\frametitle{Extbase \& Fluid}
	\framesubtitle{ViewHelper: format.date}

	\lstset{
		basicstyle=\smaller\ttfamily
	}

	\begin{itemize}
		\item Standardwert des ViewHelper \textbf{format.date} ist der im Install Tool konfigurierte Wert

			\lstinline!$GLOBALS['TYPO3_CONF_VARS']['SYS']['ddmmyy']!

		\item Ist kein Wert im Install Tool gesetzt, wird "\texttt{Y-m-d}" verwendet (Jahr, Monat, Tag)

	\end{itemize}

\end{frame}

% ------------------------------------------------------------------------------
% ViewHelper: Backend Container
% ------------------------------------------------------------------------------
% http://forge.typo3.org/issues/49749

\begin{frame}[fragile]
	\frametitle{Extbase \& Fluid}
	\framesubtitle{ViewHelper Backend Container}

	\lstset{
		basicstyle=\smaller\ttfamily
	}

	\begin{itemize}
		\item Der Fluid ViewHelper backend container (\texttt{be.container}) wurde überarbeitet:\newline
			\smaller\texttt{typo3/sysext/fluid/Classes/ViewHelpers/Be/ContainerViewHelper.php}\normalsize\newline

			\smaller\textbf{Überholt (deprecated):}\normalsize
			\begin{itemize}
				\item \texttt{\$addCssFile} (wird abgelöst von: \texttt{\$includeCssFiles})
				\item \texttt{\$addJsFile} (wird abgelöst von: \texttt{\$includeJsFiles})
			\end{itemize}

			\smaller\textbf{Neu:}\normalsize
			\begin{itemize}
				\item \texttt{\$loadJQuery}
				\item \texttt{\$includeCssFiles}
				\item \texttt{\$includeJsFiles}
				\item \texttt{\$addJsInlineLabels}
			\end{itemize}

	\end{itemize}

\end{frame}

% ------------------------------------------------------------------------------
% ViewHelper: button.icon
% ------------------------------------------------------------------------------

\begin{frame}[fragile]
	\frametitle{Extbase \& Fluid}
	\framesubtitle{ViewHelper: button.icon}

	\lstset{
		basicstyle=\smaller\ttfamily
	}

	\begin{itemize}
		\item Fluid ViewHelper \textbf{button.icon} ist nicht mehr "experimentell"
		\item Liefert ein Button-Icon zurück (ggf. mit Link)

			\begin{lstlisting}
				<f:be.buttons.icon
				  icon="actions-document-new" title="Create new Foo" />

				<f:be.buttons.icon uri="{f:uri.action(action:'new')}"
				  icon="actions-document-new" title="Create new Foo" />
			\end{lstlisting}

		\item Das Attribut \texttt{icon} unterstützt derzeit über 310 Werte!\newline

			\smaller
				Siehe:\newline
				\texttt{\$GLOBALS['TBE\_STYLES']['spriteIconApi']['coreSpriteImageNames']}\newline
				...in der Datei:\newline
				\texttt{typo3/systext/core/ext\_tables.php}
			\normalsize

	\end{itemize}

\end{frame}

% ------------------------------------------------------------------------------
% Option addQueryStringMethod
% ------------------------------------------------------------------------------
% http://forge.typo3.org/issues/11441
% http://forge.typo3.org/issues/35281

\begin{frame}[fragile]
	\frametitle{Extbase \& Fluid}
	\framesubtitle{Option addQueryStringMethod}

	\begin{itemize}
		\item Option \texttt{addQueryString} unterstützt lediglich \textbf{GET}-Argumente, die dann an den erzeugten Link angehängt werden
		\item \textbf{POST}-Argumente (wie sie z.B. bei Widgets vorkommen) funktionieren mit dieser Option allerdings nicht
		\item Über die neue Option \texttt{addQueryStringMethod} kann nun definiert werden, welche Parameter verwendet werden sollen:\newline
			GET (default), POST, GET/POST oder POST/GET
		\item Folgende ViewHelper wurden mit dieser Option ausgestattet:
	\end{itemize}

	\begin{columns}[T]

		\begin{column}{.5\textwidth}
			\begin{itemize}\vspace*{-0.4cm}\smaller
				\advance\leftskip+1cm
				\item \texttt{link.action}
				\item \texttt{link.page}
				\item \texttt{uri.action}
				\item \texttt{uri.page}
			\end{itemize}
		\end{column}

		\begin{column}{.5\textwidth}
			\begin{itemize}\vspace*{-0.4cm}\smaller
				\item \texttt{widget.link}
				\item \texttt{widget.uri}
				\item \texttt{widget.paginate}
			\end{itemize}
		\end{column}

	\end{columns}

\end{frame}

% ------------------------------------------------------------------------------
% Fluid: Fallback path for templates
% ------------------------------------------------------------------------------

\begin{frame}[fragile]
	\frametitle{Extbase \& Fluid}
	\framesubtitle{Fluid: Template Fallback Pfad}

	\lstset{
		basicstyle=\tiny\ttfamily
	}

	\begin{itemize}
		\item Fluid erlaubt nun "Fallback" Pfade für Templates, Partials und Layouts:\newline
			\smaller\texttt{templateRootPaths}, \texttt{partialRootPaths}, \texttt{layoutRootPaths}\normalsize
		\item Reihenfolge der Verzeichnisse, die geprüft werden: höchster Index zuerst, dann
			Verzeichnisse mit ensprechend kleinerem Index

			\begin{lstlisting}
				plugin.tx_myextension {
				  view {
				    templateRootPath = EXT:myextension/Resources/Private/Templates/
				  }
				}

				plugin.tx_myextension {
				  view {
				    templateRootPath >
				    templateRootPaths {
				      10 = fileadmin/myextension/Templates/
				      20 = EXT:myextension/Resources/Private/Templates/
				    }
				  }
				}
			\end{lstlisting}

	\end{itemize}

\end{frame}

% ------------------------------------------------------------------------------

