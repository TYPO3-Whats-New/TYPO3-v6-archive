% ------------------------------------------------------------------------------
% TYPO3 CMS 6.2 LTS - What's New - Chapter "In-Depth Changes" (Dutch Version)
%
% @author	Christiaan Wiesenekker <cwiesenekker@gmail.com>
% @author	Ric van Westhreenen <ric.vanwesthreenen@typo3.org>
% @license	Creative Commons BY-NC-SA 3.0
% @link		http://typo3.org/download/release-notes/whats-new/
% @language	Dutch
% ------------------------------------------------------------------------------
% Chapter: In-Depth Changes
% ------------------------------------------------------------------------------

\section{Diepgaande veranderingen}
\begin{frame}[fragile]
	\frametitle{Diepgaande veranderingen}

	\begin{center}\huge{Hoofdstuk 6:}\end{center}
	\begin{center}\huge{\color{typo3darkgrey}\textbf{Diepgaande veranderingen}}\end{center}

\end{frame}

% ------------------------------------------------------------------------------
% normalize.css
% ------------------------------------------------------------------------------
% http://forge.typo3.org/issues/47920

\begin{frame}[fragile]
	\frametitle{Diepgaande veranderingen}
	\framesubtitle{Normalize.css}

	\begin{itemize}
		\item Backend user interface maakt gebruik van \texttt{normalize.css},\newline
			welke ervoor zorgt dat browsers alle elementen concistenter renderd en in lijn is met de monderne standaarden
		\item Modern, HTML5-ready, alternatief naar de traditionele CSS reset
		\item Doelen van \texttt{normalize.css} zijn:

			\begin{itemize}
				\item Behouden van nuttige browser standaarden in plaats van ze te verwijderen
				\item Normaliseren van stylen voor een groot spectrum van HTML elementen
				\item Correct bugs en veel voorkomende browser inconsistenties
				\item Verbeterde usability met subtiele verbeteringen
				\item Leg de code uit door het gebruik van commentaar en gedetaileerde documentatie
			\end{itemize}

	\end{itemize}

\end{frame}

% ------------------------------------------------------------------------------
% displayCond options BIT and !BIT
% ------------------------------------------------------------------------------
% http://forge.typo3.org/issues/45514

\begin{frame}[fragile]
	\frametitle{Diepgaande veranderingen}
	\framesubtitle{TCA: displayCond opties BIT en !BIT}

	\lstset{
		basicstyle=\tiny\ttfamily
	}

	\begin{itemize}
		\item Controleer met een multi-value veld in \texttt{displayCond} (bitwise)\newline
			\texttt{BIT}: bit is ingesteld, \texttt{!BIT}: bit is \underline{niet} ingesteld
	\end{itemize}

	\begin{columns}[T]

		\begin{column}{.5\textwidth}

			\advance\leftskip+1cm
			Uitgaande van deze TCA:

			\lstset{xleftmargin=1cm}

			\begin{lstlisting}
				'content' => array(
				  'label' => '...',
				  'config' => array(
				    'type' => 'check',
				    'items' => array(
				      array('Content A', ''),
				      array('Content B', ''),
				      array('Content C', ''),
				    ),
				  )
				),
			\end{lstlisting}

		\end{column}
		\begin{column}{.5\textwidth}

			Voorbeelden:

			\begin{lstlisting}
				'content_a' => array(
				  'label' => '...',
				  'displayCond' => 'FIELD:content:BIT:1',
				  'config' => array(
				    'type' => 'text',
				  )
				),

				'content_b' => array(
				  'label' => '...',
				  'displayCond' => 'FIELD:content:!BIT:2',
				  'config' => array(
				    'type' => 'text',
				  )
				),
			\end{lstlisting}
		\end{column}

	\end{columns}

\end{frame}

% ------------------------------------------------------------------------------
% Automatic language updates for extensions
% ------------------------------------------------------------------------------
% http://forge.typo3.org/issues/43703

\begin{frame}[fragile]
	\frametitle{Diepgaande veranderingen}
	\framesubtitle{Taal Updates}

	\begin{itemize}
		% \item Extbase Command Controller allows to update translations of extensions for selected languages
		\item Extbase Command Controller staat taal updates toe voor Extenties:

			\begin{lstlisting}
				$GLOBALS['TYPO3_CONF_VARS']['SC_OPTIONS']['extbase']
				  ['commandControllers'][] =
				  'TYPO3\\CMS\\Lang\\Command\\LanguageCommandController';
			\end{lstlisting}

		\item Voorbeeld call:

			\lstinline!typo3/cli_dispatch.phpsh extbase language:update de,en,fr!

		\item Komma-gescheiden lijst van locales (bijv. \texttt{de,en,fr}) limiteerd de updates van deze talen
		\item Zonder dit argument worden alle talen die zijn ingesteld in de module "Talen" geupdated

	\end{itemize}

\end{frame}

% ------------------------------------------------------------------------------
% Migrate system extension manuals to reStructured Text
% ------------------------------------------------------------------------------
% http://forge.typo3.org/issues/50052

\begin{frame}[fragile]
	\frametitle{Diepgaande veranderingen}
	\framesubtitle{Systeem Extenties: ReST handleidingen}

	\begin{itemize}
		\item Alle systeem Extentie handleidingen zijn gemigreerd naar reStructuredText
		\item OpenOffice handleidingen worden niet langer gebruikt en zijn verwijderd
		\item ReST is een 'easy-to-read', what-you-see-is-what-you-get(wysiwyg) plaintext markup syntax en parser systeem
		\item ReST bestanden van systeem Extenties worden opgeslagen in:\newline
			\texttt{typo3/sysext/<extensionkey>/Documentation/*}

		\item Verdere informatie:

			\begin{itemize}
				\item \url{http://en.wikipedia.org/wiki/ReStructuredText}
				\item \url{http://wiki.typo3.org/ReST}
			\end{itemize}

	\end{itemize}

\end{frame}

% ------------------------------------------------------------------------------
% Support custom translation servers for extensions
% ------------------------------------------------------------------------------
% http://forge.typo3.org/issues/50052

\begin{frame}[fragile]
	\frametitle{Diepgaande veranderingen}
	\framesubtitle{Custom Vertaal Servers}

	\begin{itemize}
		\item Support van custom vertaal servers voor Extenties is geimplemented
		\item Door het gebruik van XLIFF en een nieuwe Signal/Slot,\newline
			wordt dit een absolute 'no-brainer' (see next slide for an example)
		\item Een mogelijke server vertaal oplossing: \textbf{Pootle}

			\begin{itemize}
				\item online translation management tool met translation interface
				\item geschreven in Python/Django
				\item in origine ontiwkkeld en uitgegeven door \url{translate.org.za}
				\item GNU GPL license
			\end{itemize}

	\end{itemize}

\end{frame}

% ------------------------------------------------------------------------------
% Support custom translation servers for extensions
% ------------------------------------------------------------------------------
% http://forge.typo3.org/issues/50052

\begin{frame}[fragile]
	\frametitle{Diepgaande veranderingen}
	\framesubtitle{Custom Vertaal Servers}

	Voorbeeld: \texttt{EXT:myextension/localconf.php}

	\lstset{
		basicstyle=\tiny\ttfamily
	}

	\begin{lstlisting}
		/**
		 * @var \TYPO3\CMS\Extbase\SignalSlot\Dispatcher $signalSlotDispatcher
		 */
		$signalSlotDispatcher =
		  \TYPO3\CMS\Core\Utility\GeneralUtility::makeInstance(
		    'TYPO3\\CMS\\Extbase\\SignalSlot\\Dispatcher');

		$signalSlotDispatcher->connect(
		  'TYPO3\\CMS\\Lang\\Service\\UpdateTranslationService',
		  'postProcessMirrorUrl',
		  'Company\\Extension\Slots\\CustomMirror',
		  'postProcessMirrorUrl'
		);
	\end{lstlisting}

\end{frame}

% ------------------------------------------------------------------------------
% Support custom translation servers for extensions
% ------------------------------------------------------------------------------
% http://forge.typo3.org/issues/50052

\begin{frame}[fragile]
	\frametitle{Diepgaande veranderingen}
	\framesubtitle{Custom Vertaal Servers}

	Voorbeeld: \texttt{EXT:myextension/Classes/Slots/CustomMirror.php}

	\lstset{
		basicstyle=\tiny\ttfamily
	}

	\begin{lstlisting}
		<?php
		namespace Company\Extensions\Slots;
		class CustomMirror {

		  /**
		   * @var string
		   */
		  protected static $extKey = 'myextension';

		  public function postProcessMirrorUrl($extensionKey, &$mirrorUrl) {
		    if ($extensionKey === self::$extKey) {
		      $mirrorUrl = 'http://example.com/typo3-packages/';
		    }
		  }

		}
	\end{lstlisting}

\end{frame}

% ------------------------------------------------------------------------------
% Support custom translation servers for extensions
% ------------------------------------------------------------------------------
% http://forge.typo3.org/issues/50052

\begin{frame}[fragile]
	\frametitle{Diepgaande veranderingen}
	\framesubtitle{Custom Vertaal Servers}

	Verwachtte bestand/mappen structuur op server:

	\begin{lstlisting}
		http://example.com/typo3-packages/
		 `-- <first-letter-of-extension-key>
		     `-- <second-letter-of-extension-key>
		         `-- <extension-key>-l10n
		             |-- <extension-key>-l10n-de.zip
		             |-- <extension-key>-l10n-fr.zip
		             |-- <extension-key>-l10n-it.zip
		             `-- <extension-key>-l10n.xml
	\end{lstlisting}

	Bijvoorbeeld:

	\begin{lstlisting}
		http://example.com/typo3-packages/m/y/myextension-l10n/myextension-l10n.xml
	\end{lstlisting}

\end{frame}

% ------------------------------------------------------------------------------
% Support custom translation servers for extensions
% ------------------------------------------------------------------------------
% http://forge.typo3.org/issues/50052

\begin{frame}[fragile]
	\frametitle{Diepgaande veranderingen}
	\framesubtitle{Custom Vertaal Servers}

	Voorbeeld: \texttt{<extension-key>-l10n.xml}

	\lstset{
		basicstyle=\tiny\ttfamily
	}

	\begin{lstlisting}
		<?xml version="1.0" standalone="yes" ?>
		  <TERlanguagePackIndex>
		    <meta>
		      <timestamp>1374841386</timestamp>
		      <date>2013-07-26 14:23:06</date>
		    </meta>
		    <languagePackIndex>
		    <languagepack language="de">
		      <md5>1cc7046c3b624ba1fb1ef565343b84a1</md5>
		    </languagepack>
		    <languagepack language="fr">
		     <md5>f00f73ae5c43cb68392e6c508b65de7a</md5>
		    </languagepack>
		    <languagepack language="it">
		     <md5>cd59530ce1ee0a38e6309544be6bcb3d</md5>
		    </languagepack>
		  </languagePackIndex>
		</TERlanguagePackIndex>
	\end{lstlisting}

\end{frame}

% ------------------------------------------------------------------------------
% Automatic import of t3d files for extensions
% ------------------------------------------------------------------------------
% http://forge.typo3.org/issues/51437

\begin{frame}[fragile]
	\frametitle{Diepgaande veranderingen}
	\framesubtitle{Automatische t3d Import}

	\begin{itemize}
		\item Extenties kunnen nu een automatisch een import initaliseren van \textbf{t3d packages} \newline
			bij installatie van een Extentie
		\item t3d bestanden bevatten dingen als data, relaties en bestanden. 
		\item Het T3D bestand moet \texttt{data.t3d} worden genoemd en geplaatst in:\newline
			\texttt{EXT:myextension/Initialisation/}

		\item De import gebeurt maar \underline{een} keer\newline
			(zelfs als de Extentie later word geherinstaleerd)

	\end{itemize}

\end{frame}

% ------------------------------------------------------------------------------
% Automatic import of files for extensions
% ------------------------------------------------------------------------------
% http://forge.typo3.org/issues/51446

\begin{frame}[fragile]
	\frametitle{Diepgaande veranderingen}
	\framesubtitle{Automatische bestands import}

	\begin{itemize}
		\item Extenties kunnen nu een automatisch een import initaliseren van \textbf{t3d packages} \newline
			bij installatie van een Extentie
		\item Bestanden worden gekopieerd naar:\newline
			\texttt{fileadmin/<extensionkey>/}
		\item Bestanden moeten worden geladen in:\newline
			\texttt{EXT:myextension/Initialisation/Files/...}

		\item De import gebeurt maar \underline{een} keer\newline
			(zelfs als de Extentie later word geherinstaleerd)

	\end{itemize}

\end{frame}

% ------------------------------------------------------------------------------
% Use an extension as repository
% ------------------------------------------------------------------------------
% http://forge.typo3.org/issues/51835

\begin{frame}[fragile]
	\frametitle{Diepgaande veranderingen}
	\framesubtitle{Gebruik een Extentie als Repository}

	\begin{itemize}
		\item Soms zijn Extenties afhaneklijk van custom versies van andere extenties of van Extenties die niet in de TER zijn te vinden
		\item Om dit probleem op te lossen kunnen extenisies worden uitgebracht met andere Extenties
		\item Deze moeten worden geplaatst in (unpacked):\newline
			\texttt{EXT:myextension/Initialisation/Extensions/...}

		\item Bij installatie worden deze gekopieerd naar:\newline
			\texttt{typo3conf/ext/}

		\item Hierna zijn de afhanekelijkheden van de Extentie opgelost.

	\end{itemize}

\end{frame}

% ------------------------------------------------------------------------------
% CLI command to install/uninstall extensions
% ------------------------------------------------------------------------------
% http://forge.typo3.org/issues/51629

\begin{frame}[fragile]
	\frametitle{Diepgaande veranderingen}
	\framesubtitle{Installeer/deinstalleer Extenties via CLI}

	\begin{itemize}
		\item Installeer/deinstalleer Extenties via CLI
		\item Voorbeeld:
			\lstinline!typo3/cli_dispatch.phpsh extbase extension:install <extensionkey>!
			\lstinline!typo3/cli_dispatch.phpsh extbase extension:uninstall <extensionkey>!

		\item NB: een backend gebruiker \textbf{\_cli\_lowlevel} is hiervoor verplicht
	\end{itemize}

\end{frame}

% ------------------------------------------------------------------------------
% Enable/disable cascading deletion of child elements
% ------------------------------------------------------------------------------
% http://forge.typo3.org/issues/50391

\begin{frame}[fragile]
	\frametitle{Diepgaande veranderingen}
	\framesubtitle{Trapsgewijs verwijderen van 'Child Elementen'}

	\begin{itemize}
		\item TCA ondersteund nu een instelling om  het trapsgewijs verwijderen van 'child elements' in- of uit te schakelen
		\item De relatie moet van het type "inline" zijn
		\item De standaard waarde is \texttt{TRUE} (verwijderen van "inline child records" staat aan)
		\item Voorbeeld (verwijderen van "inline child records"):

			\begin{lstlisting}
				...
				'type' => 'inline',
				'foreign_table' => ...,
				  'behaviour' => array(
				    'enableCascadingDelete' => 0
				  )
				  ...
				)
				...
			\end{lstlisting}

	\end{itemize}

\end{frame}

% ------------------------------------------------------------------------------
% Multiple category fields per table
% ------------------------------------------------------------------------------
% http://forge.typo3.org/issues/51921

\begin{frame}[fragile]
	\frametitle{Diepgaande veranderingen}
	\framesubtitle{Meerdere categorieen velden per tabel}

	\begin{itemize}
		\item In TYPO3 < 6.2 was het alleen mogelijk om \underline{een} \texttt{makeCategorizable()} call per tabel te maken
			(meerdere calls zouden eerdere categorien veld declaraties overschrijven)
		\item Sinds TYPO3 >= 6.2, meerdere categorie velden zijn mogelijk per tabel
		\item Voorbeeld:

			\begin{lstlisting}
				\TYPO3\CMS\Core\Utility\ExtensionManagementUtility::makeCategorizable(
				  $extensionKey,
				  $tableName,
				  $fieldName = 'categories',
				  $options = array(
				  	'label' => 'my category'
				  )
				);
			\end{lstlisting}

		\item Aangepaste labels voor elk categorie veld kunnen worden ingesteld in een array \texttt{\$options}

	\end{itemize}

\end{frame}

% ------------------------------------------------------------------------------
% Backend layout data providers
% ------------------------------------------------------------------------------
% http://forge.typo3.org/issues/37208

\begin{frame}[fragile]
	\frametitle{Diepgaande veranderingen}
	\framesubtitle{Backend Layout Data Providers}

	\begin{itemize}
		\item In TYPO3 < 6.2 werden backend layouts opgeslagen in de DB als normale records
		\item Sinds TYPO3 >= 6.2 zogenaamde \emph{data providers} kunnen worden gedefinieerd \newline
			\small(bijvoorbeeld om extenies het mogelijk te maken hun eigen backend layout uit te geven vanuit static files)\normalsize

		\item Data providers moeten de interface implementeren:\newline
			\smaller\texttt{
				TYPO3\textbackslash\textbackslash
				CMS\textbackslash\textbackslash
				Backend\textbackslash\textbackslash
				View\textbackslash\textbackslash
				BackendLayout\textbackslash\textbackslash
				DataProviderInterface}\normalsize

		\item en dit kan worden geregistereerd door:

			\begin{lstlisting}
				$GLOBALS['TYPO3_CONF_VARS']['SC_OPTIONS']
				  ['BackendLayoutDataProvider'][$_EXTKEY] = 'Classname';
			\end{lstlisting}


	\end{itemize}

\end{frame}

% ------------------------------------------------------------------------------
% Backend layout data providers
% ------------------------------------------------------------------------------
% http://forge.typo3.org/issues/37208

\begin{frame}[fragile]
	\frametitle{Diepgaande veranderingen}
	\framesubtitle{Backend Layout Data Providers}

	\begin{itemize}
		\item Nieuwe API functies voor het gebruiken van de backend layout data providers:

			\begin{lstlisting}
				'itemsProcFunc' => 'TYPO3\\CMS\\Backend\\View\\
				  BackendLayoutView->addBackendLayoutItems'
			\end{lstlisting}

			\begin{lstlisting}
				getBackendLayoutView()->getSelectedCombinedIdentifier($id);
				getBackendLayoutView()->getSelectedBackendLayout();
			\end{lstlisting}

		\item Nieuwe PageTSconfig optie om backend layouts te excluden:

			\begin{lstlisting}
				options.backendLayout.exclude = default_1, my_extension__headerLayout
			\end{lstlisting}

	\end{itemize}

\end{frame}

% ------------------------------------------------------------------------------
% Filter for multiple value selector
% ------------------------------------------------------------------------------
% http://forge.typo3.org/issues/49739

\begin{frame}[fragile]
	\frametitle{Diepgaande veranderingen}
	\framesubtitle{Multiple Value Selector (1)}

	\begin{itemize}
		\item Filter beschikbare items in een 'multi-select' element (d.m.v. TCA instellingen )
		\item Bijvoorbeeld: maak een text-veld beschikbaar voor het individueel filteren van woorden en het definieren van zoek worden die een gebruiker kan selecteren via een dropdown box

		\item Om dit te gebruiken, pas de TCA als volgt aan\newline
			(zoals in bestand \texttt{typo3conf/extTables.php}):

			\lstset{
				basicstyle=\tiny\ttfamily
			}

			\begin{lstlisting}
				$GLOBALS['TCA']['fe_users']['columns']['usergroup']['config']
				  ['enableMultiSelectFilterTextfield'] = TRUE;
				$GLOBALS['TCA']['fe_users']['columns']['usergroup']['config']
				  ['multiSelectFilterItems'] = array(

				  // no filter
				  array('', 'show all'),

				  // first value: filter, second value: label
				  array('News', 'News'),
				);
			\end{lstlisting}

	\end{itemize}

\end{frame}

% ------------------------------------------------------------------------------
% Filter for multiple value selector
% ------------------------------------------------------------------------------
% http://forge.typo3.org/issues/49739

\begin{frame}[fragile]
	\frametitle{Diepgaande veranderingen}
	\framesubtitle{Multiple Value Selector (2)}

	Het resultaat lijk dan hierop:

	\begin{figure}
		\includegraphics[width=0.9\linewidth,height=4cm]{Images/Dummy_1640x470.png}
	\end{figure}

\end{frame}

% ------------------------------------------------------------------------------
% Miscellaneous
% ------------------------------------------------------------------------------
% http://forge.typo3.org/issues/49037 (Custom record list in element browser)
% http://forge.typo3.org/issues/36505 (Increase size of be_groups.subgroup field)
% http://forge.typo3.org/issues/49270 (Merge extensions TS/Template)

\begin{frame}[fragile]
	\frametitle{Diepgaande veranderingen}
	\framesubtitle{Diversen}

	\begin{itemize}

		\item \textbf{Custom record lijst:}\newline
			\small
				Een 'custom record list' instantie kan in de element browser worden gebruikt om een de standaard element browser record lijst te overschrijven
			\normalsize

		\item \textbf{Meer subgroepen:}\newline
			\small
				Attribuut \texttt{subgroep} in DB tabel \texttt{be\_groups} veranderd van\texttt{varchar(250)} naar \texttt{text}, wat veel meer subgroepen toestaat (backend gebruikers/groepen)
			\normalsize

		\item \textbf{Extentie TS/Template samengevoegd:}\newline
			\small
				Technisch gezien, "WEB > Template" was verspreid onder meerdere Extenties (tstemplate\_ceditor, tstemplate\_info,
tstemplate\_objbrowser en tstemplate\_analyzer). Al deze Extenties zijn nu samengevoegd in een Extentie: "tstemplate"
			\normalsize

	\end{itemize}
	
\end{frame}

% ------------------------------------------------------------------------------
% Miscellaneous
% ------------------------------------------------------------------------------
% http://forge.typo3.org/issues/49721 (Add label_userFunc_options support to BackendUtility)
% http://forge.typo3.org/issues/50441 (Add a timestamp when downloading an extension)
% http://forge.typo3.org/issues/51352 (Force saltedpasswords for Backend)

\begin{frame}[fragile]
	\frametitle{Diepgaande veranderingen}
	\framesubtitle{Diversen}

	\begin{itemize}

		\item \textbf{label\_userFunc\_option:}\newline
			\small
				Support van \texttt{label\_userFunc\_options} toegevoegd aan \texttt{BackendUtility}
			\normalsize

		\item \textbf{Extension filename:}\newline
			\small
				Wanneer je een Extentie download via de Extentiemanager, de bestandsnaam bevat een timestamp (jaar, maand, dag en tijd):\newline
				\texttt{<extensionKey>\_<version>\_<timestamp>.zip}\newline
				\texttt{myextension\_1.0.0\_201312102359.zip}
			\normalsize

		\item \textbf{EXT:saltedpasswords:}\newline
			\small
				Extentie EXT:saltedpasswords is een verplichte systeem Extentie en staat standaard geinstalleerd vanaf nu.
				Dit zorgt ervoor dat salted hashes worden geforceerd voor backend authenticatie. De InstallTool controleerd deze instellingen en neemt hem over wanneer dit nodig is.
			\normalsize

	\end{itemize}
	
\end{frame}

% ------------------------------------------------------------------------------
% Miscellaneous
% ------------------------------------------------------------------------------
% http://forge.typo3.org/issues/51138 (Allow SignalSlots to modify arguments)
% http://forge.typo3.org/issues/31996 (Transfer query parameters in preview)

\begin{frame}[fragile]
	\frametitle{Diepgaande veranderingen}
	\framesubtitle{Diversen}

	\begin{itemize}

		\item \textbf{SignalSlots om argumenten aan te passen:}\newline
			\small
				Arugementen doorgegeven aan de SignalSlots dispatcher kunnen worden aangepast en de dispatcher geeft de (aangepaste) argumenten terug zoals die zijn ontvangen om de 'chaining' intact te houden.
			\normalsize

		\item \textbf{Workspace preview:}\newline
			\small
				Query parameters worden doorgegeven aan de workspace preview. Dit was een problmeen in TYPO3 < 6.2, daar werkte het doorgeven van aangepaste parameters niet goed in extenties.
			\normalsize

	\end{itemize}
	
\end{frame}

% ------------------------------------------------------------------------------
% Miscellaneous
% ------------------------------------------------------------------------------
% http://forge.typo3.org/issues/14730 (Support for proxy NTLM authentication)
% http://forge.typo3.org/issues/49667 (Enable double-resolution icons in SpriteGenerator)

\begin{frame}[fragile]
	\frametitle{Diepgaande veranderingen}
	\framesubtitle{Diversen}

	\begin{itemize}

		\item \textbf{Double-resolution icons:}\newline
			\small
				SpriteManager ondersteunt hoge resolutie iconen vanaf nu: het generereerd een tweede sprite met 2x grotere iconen(een tweede bestand met "@x2.png" suffix). CSS3 zorgt ervoor dat het hoge resolutie bestand wordt geladen op aparaten die dit ondersteunen\newline
				(dit heeft geen effect op de performance op andere apparaten).
			\normalsize

% *TODO* line height of the following item seems to be different compared to the point above. Why?
		\item \textbf{Proxy NTLM authenticatie:}\newline
			\small
				Ondersteuning voor proxy NTLM authenticatie (\textbf{NT} \textbf{L}AN \textbf{M}anager: een pakket van Microsoft security protocollen) toegevoegd. Deze feature kan worden geactiveerd in de Install Tool:\newline
				\newline
				\texttt{\$GLOBALS['TYPO3\_CONF\_VARS']['SYS']['curlProxyNTLM']}\newline
				\newline
				\smaller
				\emph{(nog even dit: deze functie was al 8 jaar geleden aangevraagd :-)}

			\normalsize

	\end{itemize}
	
\end{frame}

% ------------------------------------------------------------------------------

