% ------------------------------------------------------------------------------
% TYPO3 CMS 6.2 LTS - What's New - Chapter "Application Programming Interface" (Dutch version)
%
% @author	Christiaan Wiesenekker <cwiesenekker@gmail.com>
% @author	Ric van Westhreenen <ric.vanwesthreenen@typo3.org>
% @license	Creative Commons BY-NC-SA 3.0
% @link		http://typo3.org/download/release-notes/whats-new/
% @language	Dutch
% ------------------------------------------------------------------------------
% Chapter: Application Programming Interface
% ------------------------------------------------------------------------------

\section{Application Programming Interface}
\begin{frame}[fragile]
	\frametitle{Application Programming Interface}

	\begin{center}\huge{Hoofdstuk 6:}\end{center}
	\begin{center}\huge{\color{typo3darkgrey}\textbf{Application Programming Interface (API)}}\end{center}

\end{frame}

% ------------------------------------------------------------------------------
% Hook: tsfe::checkEnableFields
% ------------------------------------------------------------------------------
% http://forge.typo3.org/issues/48981

\begin{frame}[fragile]
	\frametitle{Application Programming Interface}
	\framesubtitle{Hook: \texttt{tsfe::checkEnableFields}}

	\begin{itemize}
		\item In TYPO3 < 6.2, "\emph{breid uit naar subpagina's}" kan niet worden gebruikt in eigen extensies die extra regels hebben voor pagina zichtbaarheid\newline
			\small(lijst van velden om te controleren is hard-coded in \texttt{tsfe::checkEnableFields()})\normalsize

		\item In TYPO3 >= 6.2, een nieuwe hook staat het extensies toe om nieuwe regels toe te maken voor pagina zichtbaarheid wanneer 'parent pages' when parent pages "extend to subpages" hebben geactiveerd.
		\item Class:\newline
			\smaller
				\texttt{\textbackslash
					TYPO3\textbackslash
					CMS\textbackslash
					Frontend\textbackslash
					Controller\textbackslash
					TypoScriptFrontendController}\normalsize

			\lstset{
				basicstyle=\smaller\ttfamily
			}

			\begin{lstlisting}
				$GLOBALS['TYPO3_CONF_VARS']['SC_OPTIONS']
				  ['tslib/class.tslib_fe.php']['hook_checkEnableFields']
			\end{lstlisting}

	\end{itemize}

\end{frame}

% ------------------------------------------------------------------------------
% Hook: checkFlexFormValue in DataHandler
% ------------------------------------------------------------------------------
% http://forge.typo3.org/issues/49699

\begin{frame}[fragile]
	\frametitle{Application Programming Interface}
	\framesubtitle{Hook: \texttt{checkFlexFormValue} in DataHandler}

	\begin{itemize}
		\item In TYPO3 < 6.2, wanneer je de Flexform waardes update, is er geen controle of er een bestaande waarde in de database ook echt is verwijderd. 
		\item Dit wordt een probleem, e.g. wanneer je switchable controller actions opslaat (Extbase) in de Flexform: oude acties die niet meer aanwezig mogen zijn moeten handmatig worden verwijderd

		\item In TYPO3 >= 6.2, een nieuwe hook staat het to om de oude Flexform data aan te passen voor het wordt gemerged met de nieuwe
		\item Class:\newline
			\smaller
				\texttt{\textbackslash
					TYPO3\textbackslash
					CMS\textbackslash
					Core\textbackslash
					DataHandling\textbackslash
					DataHandler}\normalsize

			\lstset{
				basicstyle=\smaller\ttfamily
			}

			\begin{lstlisting}
				$GLOBALS['TYPO3_CONF_VARS']['SC_OPTIONS']
				  ['t3lib/class.t3lib_tcemain.php']['checkFlexFormValue']
			\end{lstlisting}

		\item Method:\newline
			\smaller
				\texttt{checkFlexFormValue\_beforeMerge()}

	\end{itemize}

\end{frame}

% ------------------------------------------------------------------------------
% Hook to customize header in module "Web > Page"
% ------------------------------------------------------------------------------
% http://forge.typo3.org/issues/52579

\begin{frame}[fragile]
	\frametitle{Application Programming Interface}
	\framesubtitle{Hook om de header aan te passen}

	\begin{itemize}
		\item In TYPO3 >= 6.2, een nieuwe hook maakt het mogelijk om de header van een pagina aan te passen in de page module (Module: "Web > Page")
		\item De hook wordt aangeroepen voor de content van de pagina wordt gerendered.
		\item Class:\newline
			\smaller
				\texttt{\textbackslash
					TYPO3\textbackslash
					CMS\textbackslash
					Backend\textbackslash
					Controller\textbackslash
					PageLayoutController}\normalsize

			\lstset{
				basicstyle=\smaller\ttfamily
			}

			\begin{lstlisting}
				$GLOBALS['TYPO3_CONF_VARS']['SC_OPTIONS']
				  ['cms/layout/db_layout.php']['drawHeaderHook']
			\end{lstlisting}

		\item Method:\newline
			\smaller
				\texttt{callUserFunction()}

	\end{itemize}

\end{frame}

% ------------------------------------------------------------------------------
% IRRE: Provide default values for created records
% ------------------------------------------------------------------------------
% http://forge.typo3.org/issues/46124

\begin{frame}[fragile]
	\frametitle{Application Programming Interface}
	\framesubtitle{IRRE: standaard waardes voor gecreeerde records}

	\begin{itemize}
		\item De nieuwe TCA optie staat het to om "inline" fields in te stellen
		\item Key \texttt{foreign\_record\_defaults} staat het toe om (default) waardes in nieuwe gecreeerde records in te stellen

			\begin{lstlisting}
				config => array(
				  'type' => 'inline',
				  'foreign_table' => 'tt_content',
				  'foreign_record_defaults' => array(
				    'CType' => 'image'
				  ),
				)
			\end{lstlisting}

			\small
				Voorbeeld hierboven: \texttt{tt\_content} elementen die zijn gecreeerd voor dit IRRE veld worden standaard \textbf{image content elements}. De editor kan dit andere type instellen voor het opslaan.
			\normalsize

	\end{itemize}

\end{frame}

% ------------------------------------------------------------------------------
% Workspaces
% ------------------------------------------------------------------------------
% http://forge.typo3.org/issues/46124

\begin{frame}[fragile]
	\frametitle{Application Programming Interface}
	\framesubtitle{Workspaces (1)}

	\begin{itemize}
		\item In TYPO3 < 6.2, de module "Workspaces" kan alleen worden uitgebreid door het overschrijven van de PHP en JavaScript componenten
		\item In TYPO3 >= 6.2, is het nu mogelijk om de definitie en het gedrag van de getoonde kolommen uit te breiden in de module
		\item Een paar voorbeelden op de volgende slides...
	\end{itemize}

\end{frame}

% ------------------------------------------------------------------------------
% Workspaces
% ------------------------------------------------------------------------------
% http://forge.typo3.org/issues/46124

\begin{frame}[fragile]
	\frametitle{Application Programming Interface}
	\framesubtitle{Workspaces (2)}

	\lstset{
		basicstyle=\tiny\ttfamily
	}

	Voorbeeld 1 (file \texttt{ext\_localconf.php}):

	\begin{lstlisting}
		$GLOBALS['TYPO3_CONF_VARS']['SC_OPTIONS']
		  ['t3lib/class.t3lib_tcemain.php']['processCmdmapClass']['workspaces_logger'] =
		  'Vendor\\WorkspacesLogger\\Hook\\DataHandlerHook';
	\end{lstlisting}

	Voorbeeld 2 (file \texttt{ext\_tables.php}):

	\begin{lstlisting}
		\TYPO3\CMS\Workspaces\Service\AdditionalColumnService::getInstance()->register(
		  'WorkspacesLogger_StageChange',
		  'Vendor\\WorkspacesLogger\\DataProvider'
		);

		\TYPO3\CMS\Workspaces\Service\AdditionalResourceService::getInstance()->addJavaScriptResource(
		  'WorkspacesLogger',
		  'EXT:myextension/Resources/Public/JavaScript/StageChange.js'
		);
	\end{lstlisting}

\end{frame}

% ------------------------------------------------------------------------------
% Workspaces
% ------------------------------------------------------------------------------
% http://forge.typo3.org/issues/46124

\begin{frame}[fragile]
	\frametitle{Application Programming Interface}
	\framesubtitle{Workspaces (3)}

	Voorbeeld (file \texttt{Vendor\textbackslash
		WorkspacesLogger\textbackslash
		Hook\textbackslash
		DataHandlerHook}):

	\lstset{
		basicstyle=\tiny\ttfamily
	}

	\begin{lstlisting}
		<?php
		namespace Vendor\WorkspacesLogger\Hook;
		use TYPO3\CMS\Core\SingletonInterface;

		class DataHandlerHook implements SingletonInterface {

		  const TABLE_Name = 'tx_workspaceslogger_event';
		  const EVENT_SetStage = 91;

		  /**
		   * hook that is called when no prepared command was found
		   */
		  public function processCmdmap($command, $table, $id, $value, &$commandIsProcessed,
		    \TYPO3\CMS\Core\DataHandling\DataHandler $tcemainObj) {
		    ...
		    $action = (string) $value['action'];
		    if ($command === 'version' && $action === 'setStage' && $commandIsProcessed) {
		      ...
		    }
		  }
		}
	\end{lstlisting}

\end{frame}

% ------------------------------------------------------------------------------
% PSR-3 compatible Logger
% ------------------------------------------------------------------------------
% http://forge.typo3.org/issues/48880

\begin{frame}[fragile]
	\frametitle{Application Programming Interface}
	\framesubtitle{PSR-3 compatible Logger}

	\begin{itemize}
		\item De TYPO3 CMS 6.2 logging API is nu PSR-3 compatible
		\item PSR-3 richt zich op het maken van een standaard voor logging in PHP (standard of the PHP Framework Interop Group)

		\item Het hoofddoel van PSR-3 is
			"\emph{libraries toestaan tot het ontvangen van een LoggerInterface object en de logs er naar toe te schrijven in een eenvoudige en op een universele manier.}"

		\item Logger interface bevat snelle log methoden zoals\newline
			\texttt{debug()}, \texttt{warning()}, \texttt{notice()}, \texttt{alert()}, \texttt{error()}, etc.

		\item  Andere bronnen:
			\begin{itemize}
				\item \url{http://www.php-fig.org/psr/3/}
			\end{itemize}

	\end{itemize}

\end{frame}

% ------------------------------------------------------------------------------
% Miscellaneous
% ------------------------------------------------------------------------------
% http://forge.typo3.org/issues/49144 (MathUtility: Add canBeInterpretedAsFloat)
% http://forge.typo3.org/issues/52707 (Introduce a PHP Enumeration type)
% http://forge.typo3.org/issues/52762 (Add type converter for core types like Enumeration)

\begin{frame}[fragile]
	\frametitle{Application Programming Interface}
	\framesubtitle{Miscellaneous}

	\begin{itemize}
		\item Nieuwe methode \texttt{canBeInterpretedAsFloat()} in class: \texttt{MathUtility}\newline
			\small(Dit is een analogie van: \texttt{canBeInterpretedAsInteger()})\normalsize
		\item Nieuwe enumeration type (zonder relatie naar derde partij PHP modules):\newline
			\texttt{\textbackslash
				TYPO3\textbackslash
				CMS\textbackslash
				Core\textbackslash
				Type\textbackslash
				Enumeration}\newline

			Als voorbeeld gebruikt in in:\newline
			\texttt{\textbackslash
				TYPO3\textbackslash
				CMS\textbackslash
				Core\textbackslash
				Versioning\textbackslash
				VersionState}\newline

			...en dan als:\newline
			\texttt{new VersionState(VersionState::DEFAULT\_STATE);}

	\end{itemize}

\end{frame}

% ------------------------------------------------------------------------------

