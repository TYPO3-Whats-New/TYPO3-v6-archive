% ------------------------------------------------------------------------------
% TYPO3 CMS 6.2 LTS - What's New - Chapter "Package Management" (Serbian Version)
%
% @author	Sinisa Mitrovic <mitrovic.sinisaa@gmail.com>
% @license	Creative Commons BY-NC-SA 3.0
% @link		http://typo3.org/download/release-notes/whats-new/
% @language	Serbian
% ------------------------------------------------------------------------------
% Chapter: Package Management
% ------------------------------------------------------------------------------

\section{Package Management}
\begin{frame}[fragile]
	\frametitle{Package Management}

	\begin{center}\huge{Poglavlje 5:}\end{center}
	\begin{center}\huge{\color{typo3darkgrey}\textbf{Package Management}}\end{center}

\end{frame}

% ------------------------------------------------------------------------------
% Package Manager
% ------------------------------------------------------------------------------
% http://wiki.typo3.org/Blueprints/Packagemanager
% http://forge.typo3.org/issues/47018
% http://forge.typo3.org/issues/52737

	\begin{frame}[fragile]
	\frametitle{Package Management}
	\framesubtitle{Package Manager}

	\begin{itemize}
		\item \textbf{Package Manager} iz Typo3 Flow je uveden u Typo3 CMS
		\item Razvoj/istrazivanje zapoceto je za vreme razvijanja TYPO3 CMS 6.1 verzije
		\item Ovaj projekat ima za cilj da uskladi formate razlicitih paketa
		\item Prosirenja u TYPO3 CMS su samo specificni tipovi paketa ("Packages")
		\item Glavni ciljevi projekta:

			\begin{itemize}
				\item Odgovarajuci API za Package Management
				\item Vendor Namespace podrska
				\item Composer Package podrska
				\item Flow Package podrska
				\item Autoloader Re-factoring
			\end{itemize}

	\end{itemize}

\end{frame}

% ------------------------------------------------------------------------------
% Package Manager Integration
% ------------------------------------------------------------------------------

\begin{frame}[fragile]
	\frametitle{Package Management}
	\framesubtitle{Implementacija Package Manager-a}

	\begin{itemize}
		\item Uklonjen je \texttt{\$TYPO3\_CONF['EXT']['extListArray']} iz fajla: \newline
			\smaller\texttt{typo3conf/LocalConfiguration.php}\normalsize

		\item Stari sadrzaj fajla \small\texttt{typo3conf/LocalConfiguration.php} prekopiran je u \normalsize\newline
			\smaller\texttt{typo3conf/LocalConfiguration.beforePackageStatesMigration.php}\normalsize

		\item Fajl \texttt{typo3conf/PackageStates.php} sadrzi:

			\begin{itemize}
				\item stanje paketa (aktivno/neaktivno)
				\item lokacija prosirenja u fajl sistemu
			\end{itemize}

		\item Prosirenja iz sledecih direktorijuma se prepoznaju automatski:

			\begin{itemize}
				\item \texttt{typo3/sysext/}
				\item \texttt{typo3/ext/}
				\item \texttt{typo3/contrib/}
				\item \texttt{typo3conf/ext/}
				\item \texttt{Packages/} \emph{(rekurzivno)}
			\end{itemize}

	\end{itemize}

\end{frame}

% ------------------------------------------------------------------------------
% Package Manager Integration
% ------------------------------------------------------------------------------

\begin{frame}[fragile]
	\frametitle{Package Management}
	\framesubtitle{Implementacija Package Manager-a}

	\begin{itemize}

		\item Dva nova (dodatna) fajla u direktorijumu prosirenja:

			\begin{itemize}
				\item \texttt{composer.json}
				\item \texttt{Classes/Package.php}
			\end{itemize}

		\item Ukoliko je prosirenje neophodno, oznaka \texttt{protected}\newline
			se postavlja u fajlu \texttt{composer.json}

		\item Ukoliko fajl \texttt{PackageStates.php} on ce biti (ponovno) kreiran\newline
			i sadrzace sva prosirenja kojima je gore navedena osobina podesena na \texttt{TRUE}

		\item Autoloader dobija sopstveni administrativni panel koji se kesira

		\item Dodatne informacije:\newline
			\url{http://wiki.typo3.org/Blueprints/Packagemanager}

	\end{itemize}

\end{frame}

% ------------------------------------------------------------------------------
% Examples
% ------------------------------------------------------------------------------

\begin{frame}[fragile]
	\frametitle{Package Management}
	\framesubtitle{Implementacija Package Manager-a}

	Primer: \texttt{typo3conf/PackageManager.php}

	\lstset{
		basicstyle=\tiny\ttfamily
		% basicstyle=\fontsize{5}{7}\selectfont\ttfamily
	}

	\begin{lstlisting}
		return array ('packages' =>
		    array (
		      'core' =>
		        array (
		          'manifestPath' => '',
		          'composerName' => 'typo3/cms/core',
		          'state' => 'active',
		          'packagePath' => 'typo3/sysext/core/',
		          'classesPath' => 'Classes/',
		        ),
		      'workspaces' =>
		        array (
		          'manifestPath' => '',
		          'composerName' => 'typo3/cms/workspaces',
		          'state' => 'inactive',
		          'packagePath' => 'typo3/sysext/workspaces/',
		          'classesPath' => 'Classes/',
		        ),
		      ...
		    ),
		    'version' => 4,
		);
	\end{lstlisting}

\end{frame}

% ------------------------------------------------------------------------------
% Examples
% ------------------------------------------------------------------------------

\begin{frame}[fragile]
	\frametitle{Package Management}
	\framesubtitle{Implementacija Package Manager-a}

	Primer: \texttt{composer.json}

	\lstset{
		basicstyle=\tiny\ttfamily
	}

	\begin{lstlisting}
		{
		  "name": "typo3/cms-indexed-search",
		  "type": "typo3-cms-framework",
		  "description": "TYPO3 Core",
		  "homepage": "http://typo3.org",
		  "license": ["GPL-2.0+"],
		  "version": "6.2.0",
		  "require": {
		    "typo3/cms-core": "*"
		  },
		  "replace": {
		    "indexed_search": "*"
		  }
		}
	\end{lstlisting}

\end{frame}

% ------------------------------------------------------------------------------
% Miscellaneous
% (slide added in March 2014)
% ------------------------------------------------------------------------------

\begin{frame}[fragile]
	\frametitle{Package Management}
	\framesubtitle{Implementacija Package Manager-a}

	\lstset{
		basicstyle=\smaller\ttfamily
	}

	\begin{itemize}
		\item Paketi se takodje mogu aktivirati za vreme rada uz pomoc kljuca:
			\smaller\texttt{\$GLOBALS['TYPO3\_CONF\_VARS']['EXT']['runtimeActivatedPackages'] = array(}\space\textit{packageKey}\space\texttt{);}\normalsize

		\item Ovaj kljuc se aktivira odmah nakon inicijalizacije Package Management-a

	\end{itemize}

\end{frame}

% ------------------------------------------------------------------------------

