% ------------------------------------------------------------------------------
% TYPO3 CMS 6.2 LTS - What's New - Chapter "Application Programming Interface" (Serbian Version)
%
% @author	Sinisa Mitrovic <mitrovic.sinisaa@gmail.com>
% @license	Creative Commons BY-NC-SA 3.0
% @link		http://typo3.org/download/release-notes/whats-new/
% @language	Serbian
% ------------------------------------------------------------------------------
% Chapter: Application Programming Interface
% ------------------------------------------------------------------------------

\section{Application Programming Interface}
\begin{frame}[fragile]
	\frametitle{Application Programming Interface}

	\begin{center}\huge{Poglavlje 7:}\end{center}
	\begin{center}\huge{\color{typo3darkgrey}\textbf{Application Programming Interface (API)}}\end{center}

\end{frame}

% ------------------------------------------------------------------------------
% Hook: tsfe::checkEnableFields
% ------------------------------------------------------------------------------
% http://forge.typo3.org/issues/48981

\begin{frame}[fragile]
	\frametitle{Application Programming Interface}
	\framesubtitle{Hook: \texttt{tsfe::checkEnableFields}}

	\begin{itemize}
		\item U TYPO3 < 6.2, "\emph{extend to subpages}" se ne moze koristiti u prosirenjima koja imaju dodatna pravila o vidljivosti strane\newline
			\small(lista polja koja se mogu cekirati su kodovana u \texttt{tsfe::checkEnableFields()})\normalsize

		\item U TYPO3 >= 6.2, novi hook omogucava porsirenjima da dodaju nova pravila koja se ticu vidljivosti strane, onda kada nadstrana ima aktivirano "extend to subpages"
		\item Klasa:\newline
			\smaller
				\texttt{\textbackslash
					TYPO3\textbackslash
					CMS\textbackslash
					Frontend\textbackslash
					Controller\textbackslash
					TypoScriptFrontendController}\normalsize

			\lstset{
				basicstyle=\smaller\ttfamily
			}

			\begin{lstlisting}
				$GLOBALS['TYPO3_CONF_VARS']['SC_OPTIONS']
				  ['tslib/class.tslib_fe.php']['hook_checkEnableFields']
			\end{lstlisting}

	\end{itemize}

\end{frame}

% ------------------------------------------------------------------------------
% Hook: checkFlexFormValue in DataHandler
% ------------------------------------------------------------------------------
% http://forge.typo3.org/issues/49699

\begin{frame}[fragile]
	\frametitle{Application Programming Interface}
	\framesubtitle{Hook: \texttt{checkFlexFormValue} u DataHandler-u}

	\begin{itemize}
		\item U TYPO3 < 6.2 kada se azuriraju vrednosti iz Flexform-e ne postoji provera da li je postojeca vrednost u bazi podataka zapravo obrisana.
		\item Ovo postaje problem, npr. pri pokusaju snimanja switchable controller actions (Extbase) u Flexform-i: stare akcije koje mozda vise nisu prisutne morale su se ukloniti rucno

		\item U TYPO3 >= 6.2 novi hook dozvoljava da se podaci stare Flexform-e prilagode pre nego sto se integrisu sa novim
		\item Klasa:\newline
			\smaller
				\texttt{\textbackslash
					TYPO3\textbackslash
					CMS\textbackslash
					Core\textbackslash
					DataHandling\textbackslash
					DataHandler}\normalsize

			\lstset{
				basicstyle=\smaller\ttfamily
			}

			\begin{lstlisting}
				$GLOBALS['TYPO3_CONF_VARS']['SC_OPTIONS']
				  ['t3lib/class.t3lib_tcemain.php']['checkFlexFormValue']
			\end{lstlisting}

		\item Metoda:\newline
			\smaller
				\texttt{checkFlexFormValue\_beforeMerge()}

	\end{itemize}

\end{frame}

% ------------------------------------------------------------------------------
% Hook to customize header in module "Web > Page"
% ------------------------------------------------------------------------------
% http://forge.typo3.org/issues/52579

\begin{frame}[fragile]
	\frametitle{Application Programming Interface}
	\framesubtitle{Hook za prilagodjavanje zaglavlja}

	\begin{itemize}
		\item U TYPO3 >= 6.2 novi hook dozvoljava modifikovanje zaglavlja strane u page modulu (Module: "Web > Page")
		\item Ovaj hook se poziva pre nego sto se izrendera sadrzaj strane
		\item Klasa:\newline
			\smaller
				\texttt{\textbackslash
					TYPO3\textbackslash
					CMS\textbackslash
					Backend\textbackslash
					Controller\textbackslash
					PageLayoutController}\normalsize

			\lstset{
				basicstyle=\smaller\ttfamily
			}

			\begin{lstlisting}
				$GLOBALS['TYPO3_CONF_VARS']['SC_OPTIONS']
				  ['cms/layout/db_layout.php']['drawHeaderHook']
			\end{lstlisting}

		\item Metoda:\newline
			\smaller
				\texttt{callUserFunction()}

	\end{itemize}

\end{frame}

% ------------------------------------------------------------------------------
% IRRE: Provide default values for created records
% ------------------------------------------------------------------------------
% http://forge.typo3.org/issues/46124

\begin{frame}[fragile]
	\frametitle{Application Programming Interface}
	\framesubtitle{IRRE: osnovne vrednosti za napravljene zapise}

	\begin{itemize}
		\item Novi TCA omogucava konfigurisanje "inline" polja
		\item Kljuc \texttt{foreign\_record\_defaults} dozvoljava da (osnovne) vrednosti budu postavljene u novokreiranom zapisu

			\begin{lstlisting}
				'config' => array(
				  'type' => 'inline',
				  'foreign_table' => 'tt_content',
				  'foreign_record_defaults' => array(
				    'CType' => 'image'
				  ),
				)
			\end{lstlisting}

			\small
				Primer iznad: \texttt{tt\_content} elementi kreirani za ovo IRRE polje ce biti \textbf{image content elements}. Urednik moze podesiti neki drugi tip pre snimanja.
			\normalsize
	\end{itemize}

\end{frame}

% ------------------------------------------------------------------------------
% Workspaces
% ------------------------------------------------------------------------------
% http://forge.typo3.org/issues/46124

\begin{frame}[fragile]
	\frametitle{Application Programming Interface}
	\framesubtitle{Workspaces (1)}

	\begin{itemize}
		\item U TYPO3 < 6.2 modul workspace mogao se prosiriti samo ukoliko se pregaze PHP i JavaScript komponente
		\item U TYPO3 >= 6.2 sada je moguce prosiriti definisanost i ponasanje prikazanih kolona u modulu
		\item Neki od primera su na sledecim slajdovima...
	\end{itemize}

\end{frame}

% ------------------------------------------------------------------------------
% Workspaces
% ------------------------------------------------------------------------------
% http://forge.typo3.org/issues/46124

\begin{frame}[fragile]
	\frametitle{Application Programming Interface}
	\framesubtitle{Workspaces (2)}

	\lstset{
		basicstyle=\tiny\ttfamily
	}

	Primer (fajl \texttt{ext\_localconf.php}):

	\begin{lstlisting}
		$GLOBALS['TYPO3_CONF_VARS']['SC_OPTIONS']
		  ['t3lib/class.t3lib_tcemain.php']['processCmdmapClass']['workspaces_logger'] =
		  'Vendor\\WorkspacesLogger\\Hook\\DataHandlerHook';
	\end{lstlisting}

	Primer (fajl \texttt{ext\_tables.php}):

	\begin{lstlisting}
		\TYPO3\CMS\Workspaces\Service\AdditionalColumnService::getInstance()->register(
		  'WorkspacesLogger_StageChange',
		  'Vendor\\WorkspacesLogger\\DataProvider'
		);

		\TYPO3\CMS\Workspaces\Service\AdditionalResourceService::getInstance()->addJavaScriptResource(
		  'WorkspacesLogger',
		  'EXT:myextension/Resources/Public/JavaScript/StageChange.js'
		);
	\end{lstlisting}

\end{frame}

% ------------------------------------------------------------------------------
% Workspaces
% ------------------------------------------------------------------------------
% http://forge.typo3.org/issues/46124

\begin{frame}[fragile]
	\frametitle{Application Programming Interface}
	\framesubtitle{Workspaces (3)}

	Primer (fajl \texttt{Vendor\textbackslash
		WorkspacesLogger\textbackslash
		Hook\textbackslash
		DataHandlerHook}):

	\lstset{
		basicstyle=\tiny\ttfamily
	}

	\begin{lstlisting}
		<?php
		namespace Vendor\WorkspacesLogger\Hook;
		use TYPO3\CMS\Core\SingletonInterface;

		class DataHandlerHook implements SingletonInterface {

		  const TABLE_Name = 'tx_workspaceslogger_event';
		  const EVENT_SetStage = 91;

		  /**
		   * hook that is called when no prepared command was found
		   */
		  public function processCmdmap($command, $table, $id, $value, &$commandIsProcessed,
		    \TYPO3\CMS\Core\DataHandling\DataHandler $tcemainObj) {
		    ...
		    $action = (string) $value['action'];
		    if ($command === 'version' && $action === 'setStage' && $commandIsProcessed) {
		      ...
		    }
		  }
		}
	\end{lstlisting}

\end{frame}

% ------------------------------------------------------------------------------
% PSR-3 compatible Logger
% ------------------------------------------------------------------------------
% http://forge.typo3.org/issues/48880

\begin{frame}[fragile]
	\frametitle{Application Programming Interface}
	\framesubtitle{Kompatabilni PSR-3 Logger}

	\begin{itemize}
		\item TYPO3 CMS 6.2 API za kreiranje logova sada je PSR-3 kompatabilan
		\item PSR-3 ima za cilj da napravi standard za belezenje logova u PHP-u (standard za PHP Framework Interop Group)

		\item Glavni zadatak PSR-3 je
			"\emph{omoguciti da biblioteke sadrze LoggerInterface objekte, kako bi se logovi zapsivali na jednostavan i univerzalan nacin}"

		\item Logger interfejs sadrzi kratke log metode kao sto su\newline
			\texttt{debug()}, \texttt{warning()}, \texttt{notice()}, \texttt{alert()}, \texttt{error()}, etc.

		\item Dodatne informacije:\newline
			\url{http://www.php-fig.org/psr/3/}

	\end{itemize}

\end{frame}

% ------------------------------------------------------------------------------
% API to CSRF protect Ajax calls in Backend
% (slide added in March 2014)
% ------------------------------------------------------------------------------
% http://forge.typo3.org/issues/56345
% http://forge.typo3.org/issues/56347 (documentation)

\begin{frame}[fragile]
	\frametitle{Application Programming Interface}
	\framesubtitle{CSRF zasticeni Ajax pozivi}

	\lstset{
		basicstyle=\tiny\ttfamily
	}

	\begin{itemize}
		\item Ajax pozivi u TYPO3 administratorskom panelu mogu se zastititi od CSRF (\textit{cross-site request forgery}) ako se registruju njihovi hendleri

			\begin{lstlisting}
				\TYPO3\CMS\Core\Utility\ExtensionManagementUtility::registerAjaxHandler(
				  'TxMyExt::process',
				  '\Vendor\MyExt\AjaxHandler->process'
				);
			\end{lstlisting}

		\item URL za dati Ajax ID sadrzi CSRF zastitni token, koji ce biti proveren od strane \texttt{ajax.php} dispecera

			\begin{lstlisting}
				$ajaxUrl = \TYPO3\CMS\Core\Utility\BackendUtility::getAjaxUrl('TxMyExt::process');
			\end{lstlisting}

		\item Ovim podesavanjima se moze pristupiti u JavaScript context-u strane

			\begin{lstlisting}
				var ajaxUrl = TYPO3.settings.MyExt.ajaxUrl;
			\end{lstlisting}

	\end{itemize}

\end{frame}

% ------------------------------------------------------------------------------
% Miscellaneous
% ------------------------------------------------------------------------------
% http://forge.typo3.org/issues/49144 (MathUtility: Add canBeInterpretedAsFloat)
% http://forge.typo3.org/issues/52707 (Introduce a PHP Enumeration type)
% http://forge.typo3.org/issues/52762 (Add type converter for core types like Enumeration)

\begin{frame}[fragile]
	\frametitle{Application Programming Interface}
	\framesubtitle{Razno}

	\begin{itemize}
		\item Nova metoda \texttt{canBeInterpretedAsFloat()} u klasi: \texttt{MathUtility}\newline
			\small(analogno metodi: \texttt{canBeInterpretedAsInteger()})\normalsize
		\item Novi tip za nabrajanje (nije u relaciji sa nekim drugim PHP modulima):\newline
			\texttt{\textbackslash
				TYPO3\textbackslash
				CMS\textbackslash
				Core\textbackslash
				Type\textbackslash
				Enumeration}\newline

			Na primer korisceno kod:\newline
			\texttt{\textbackslash
				TYPO3\textbackslash
				CMS\textbackslash
				Core\textbackslash
				Versioning\textbackslash
				VersionState}\newline

			...a zatim kao:\newline
			\texttt{new VersionState(VersionState::DEFAULT\_STATE);}

	\end{itemize}

\end{frame}

% ------------------------------------------------------------------------------

