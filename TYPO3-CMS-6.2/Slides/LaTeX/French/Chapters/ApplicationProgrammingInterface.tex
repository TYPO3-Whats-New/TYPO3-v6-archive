% ------------------------------------------------------------------------------
% TYPO3 CMS 6.2 LTS - What's New - Chapter "Application Programming Interface" (French Version)
%
% @author	Paul Blondiaux <pblondiaux@sodifrance.fr>
% @author	Philippe Herault <philippe.herault@plan-net.fr>
% @license	Creative Commons BY-NC-SA 3.0
% @link		http://typo3.org/download/release-notes/whats-new/
% @language	French
% ------------------------------------------------------------------------------
% Chapter: Application Programming Interface
% ------------------------------------------------------------------------------

\section{Application Programming Interface}
\begin{frame}[fragile]
	\frametitle{Application Programming Interface}

	\begin{center}\huge{Chapitre 7 :}\end{center}
	\begin{center}\huge{\color{typo3darkgrey}\textbf{Application Programming Interface (API)}}\end{center}

\end{frame}

% ------------------------------------------------------------------------------
% Hook: tsfe::checkEnableFields
% ------------------------------------------------------------------------------
% http://forge.typo3.org/issues/48981

\begin{frame}[fragile]
	\frametitle{Application Programming Interface}
	\framesubtitle{Hook : \texttt{tsfe::checkEnableFields}}

	\begin{itemize}
		\item Dans TYPO3 < 6.2, « \emph{étendre aux sous-pages} » n'est pas utilisable pour les extensions qui fournissent des règles supplémentaires de visibilité des pages (la liste des champs à vérifier est codée en dur dans \texttt{tsfe::checkEnableFields()})

		\item Dans TYPO3 >= 6.2, un nouveau hook permet aux extensions de fournir des règles supplémentaires de visibilité des pages lorsque les pages parentes ont « étendre aux sous-pages » activé.
		\item Classe :\newline
			\smaller
				\texttt{\textbackslash
					TYPO3\textbackslash
					CMS\textbackslash
					Frontend\textbackslash
					Controller\textbackslash
					TypoScriptFrontendController}\normalsize

			\lstset{
				basicstyle=\smaller\ttfamily
			}

			\begin{lstlisting}
				$GLOBALS['TYPO3_CONF_VARS']['SC_OPTIONS']
				  ['tslib/class.tslib_fe.php']['hook_checkEnableFields']
			\end{lstlisting}

	\end{itemize}

\end{frame}

% ------------------------------------------------------------------------------
% Hook: checkFlexFormValue in DataHandler
% ------------------------------------------------------------------------------
% http://forge.typo3.org/issues/49699

\begin{frame}[fragile]
	\frametitle{Application Programming Interface}
	\framesubtitle{Hook : \texttt{checkFlexFormValue} dans DataHandler (1)}

	\begin{itemize}
		\item Dans TYPO3 < 6.2, lors de la mise à jour des valeurs d'un Flexform, aucun contrôle ne vérifie si une valeur existante en base a en réalité été supprimée
		\item C'est devenu un problème, par exemple, lors de la sauvegarde des switchable controller actions (Extbase) dans le Flexform : les anciennes actions qui peuvent ne plus être présentes doivent être retirées manuellement

		\item Dans TYPO3 >= 6.2, un nouveau hook permet d'ajuster l'ancienne valeur Flexform juste avant d'être fusionnée avec la nouvelle

	\end{itemize}

\end{frame}

% ------------------------------------------------------------------------------
% Hook: checkFlexFormValue in DataHandler
% ------------------------------------------------------------------------------
% http://forge.typo3.org/issues/49699

\begin{frame}[fragile]
	\frametitle{Application Programming Interface}
	\framesubtitle{Hook : \texttt{checkFlexFormValue} dans DataHandler (2)}

	\begin{itemize}
		\item Classe :\newline
			\smaller
				\texttt{\textbackslash
					TYPO3\textbackslash
					CMS\textbackslash
					Core\textbackslash
					DataHandling\textbackslash
					DataHandler}\normalsize

			\lstset{
				basicstyle=\smaller\ttfamily
			}

			\begin{lstlisting}
				$GLOBALS['TYPO3_CONF_VARS']['SC_OPTIONS']
				  ['t3lib/class.t3lib_tcemain.php']['checkFlexFormValue']
			\end{lstlisting}

		\item Méthode :\newline
			\smaller
				\texttt{checkFlexFormValue\_beforeMerge()}

	\end{itemize}

\end{frame}

% ------------------------------------------------------------------------------
% Hook to customize header in module "Web > Page"
% ------------------------------------------------------------------------------
% http://forge.typo3.org/issues/52579

\begin{frame}[fragile]
	\frametitle{Application Programming Interface}
	\framesubtitle{Hook pour personnaliser l'en-tête}

	\begin{itemize}
		\item Dans TYPO3 >= 6.2, un nouveau hook permet de modifier l'en-tête d'une page dans le module page (Module: « Web > Page »)
		\item Ce hook est appelé avant la génération du contenu de la page
		\item Classe :\newline
			\smaller
				\texttt{\textbackslash
					TYPO3\textbackslash
					CMS\textbackslash
					Backend\textbackslash
					Controller\textbackslash
					PageLayoutController}\normalsize

			\lstset{
				basicstyle=\smaller\ttfamily
			}

			\begin{lstlisting}
				$GLOBALS['TYPO3_CONF_VARS']['SC_OPTIONS']
				  ['cms/layout/db_layout.php']['drawHeaderHook']
			\end{lstlisting}

		\item Méthode :\newline
			\smaller
				\texttt{callUserFunction()}

	\end{itemize}

\end{frame}

% ------------------------------------------------------------------------------
% IRRE: Provide default values for created records
% ------------------------------------------------------------------------------
% http://forge.typo3.org/issues/46124

\begin{frame}[fragile]
	\frametitle{Application Programming Interface}
	\framesubtitle{IRRE : valeurs par défaut pour les enregistrements créés}

	\begin{itemize}
		\item Une nouvelle option TCA permet de configurer les champs « inline »
		\item La clé \texttt{foreign\_record\_defaults} permet de définir des valeurs (par défaut) dans les nouveaux enregistrements créés

			\begin{lstlisting}
				'config' => array(
				  'type' => 'inline',
				  'foreign_table' => 'tt_content',
				  'foreign_record_defaults' => array(
				    'CType' => 'image'
				  ),
				)
			\end{lstlisting}

			\small
				Exemple ci-dessus : les éléments \texttt{tt\_content} qui sont créés pour ce champ IRRE seront des \textbf{éléments de contenus de type image} par défaut.\newline
				Le contributeur peut paramétrer le champ sur un autre type de contenu avant d'enregistrer.
			\normalsize
	\end{itemize}

\end{frame}

% ------------------------------------------------------------------------------
% Workspaces
% ------------------------------------------------------------------------------
% http://forge.typo3.org/issues/46124

\begin{frame}[fragile]
	\frametitle{Application Programming Interface}
	\framesubtitle{Espaces de travail (1)}

	\begin{itemize}
		\item Dans TYPO3 < 6.2, le module « Workspaces » ne peut être étendu qu'en surchargeant les composants PHP et JavaScript
		\item Dans TYPO3 >= 6.2, il est désormais possible d'étendre la définition et le comportement des colonnes affichées dans le module
		\item Quelques exemples dans les diapositives suivantes...
	\end{itemize}

\end{frame}

% ------------------------------------------------------------------------------
% Workspaces
% ------------------------------------------------------------------------------
% http://forge.typo3.org/issues/46124

\begin{frame}[fragile]
	\frametitle{Application Programming Interface}
	\framesubtitle{Espaces de travail (2)}

	\lstset{
		basicstyle=\tiny\ttfamily
	}

	Exemple (fichier \texttt{ext\_localconf.php}) :

	\begin{lstlisting}
		$GLOBALS['TYPO3_CONF_VARS']['SC_OPTIONS']
		  ['t3lib/class.t3lib_tcemain.php']['processCmdmapClass']['workspaces_logger'] =
		  'Vendor\\WorkspacesLogger\\Hook\\DataHandlerHook';
	\end{lstlisting}

	Exemple (fichier \texttt{ext\_tables.php}) :

	\begin{lstlisting}
		\TYPO3\CMS\Workspaces\Service\AdditionalColumnService::getInstance()->register(
		  'WorkspacesLogger_StageChange',
		  'Vendor\\WorkspacesLogger\\DataProvider'
		);

		\TYPO3\CMS\Workspaces\Service\AdditionalResourceService::getInstance()->addJavaScriptResource(
		  'WorkspacesLogger',
		  'EXT:myextension/Resources/Public/JavaScript/StageChange.js'
		);
	\end{lstlisting}

\end{frame}

% ------------------------------------------------------------------------------
% Workspaces
% ------------------------------------------------------------------------------
% http://forge.typo3.org/issues/46124

\begin{frame}[fragile]
	\frametitle{Application Programming Interface}
	\framesubtitle{Espaces de travail (3)}

	Exemple (fichier \texttt{Vendor\textbackslash
		WorkspacesLogger\textbackslash
		Hook\textbackslash
		DataHandlerHook}) :

	\lstset{
		basicstyle=\tiny\ttfamily
	}

	\begin{lstlisting}
		<?php
		namespace Vendor\WorkspacesLogger\Hook;
		use TYPO3\CMS\Core\SingletonInterface;

		class DataHandlerHook implements SingletonInterface {

		  const TABLE_Name = 'tx_workspaceslogger_event';
		  const EVENT_SetStage = 91;

		  /**
		   * hook that is called when no prepared command was found
		   */
		  public function processCmdmap($command, $table, $id, $value, &$commandIsProcessed,
		    \TYPO3\CMS\Core\DataHandling\DataHandler $tcemainObj) {
		    ...
		    $action = (string) $value['action'];
		    if ($command === 'version' && $action === 'setStage' && $commandIsProcessed) {
		      ...
		    }
		  }
		}
	\end{lstlisting}

\end{frame}

% ------------------------------------------------------------------------------
% PSR-3 compatible Logger
% ------------------------------------------------------------------------------
% http://forge.typo3.org/issues/48880

\begin{frame}[fragile]
	\frametitle{Application Programming Interface}
	\framesubtitle{Journalisation compatible PSR-3}

	\begin{itemize}
		\item L'API de journalisation de TYPO3 CMS 6.2 est maintenant compatible PSR-3
		\item Les objectifs de PSR-3 sont d'établir une norme pour la journalisation en PHP (norme du PHP Framework Interop Group)

		\item Le principal objectif de PSR-3 est
			"\emph{de permettre aux bibliothèques de recevoir un objet LoggerInterface et d'y écrire des messages de manière simple et universelle.}"

		\item Logger interface propose des méthodes raccourcis comme\newline
			\texttt{debug()}, \texttt{warning()}, \texttt{notice()}, \texttt{alert()}, \texttt{error()}, etc.

		\item Autres ressources :\newline
			\url{http://www.php-fig.org/psr/3/}

	\end{itemize}

\end{frame}

% ------------------------------------------------------------------------------
% API to CSRF protect Ajax calls in Backend
% (slide added in March 2014)
% ------------------------------------------------------------------------------
% http://forge.typo3.org/issues/56345
% http://forge.typo3.org/issues/56347 (documentation)

\begin{frame}[fragile]
	\frametitle{Application Programming Interface}
	\framesubtitle{Appels Ajax protégés contre CSRF}

	\lstset{
		basicstyle=\tiny\ttfamily
	}

	\begin{itemize}
		\item Les appels Ajax dans le Backend TYPO3 peuvent être protégés contre CSRF (\textit{cross-site request forgery}) en enregistrant leurs méthodes

			\begin{lstlisting}
				\TYPO3\CMS\Core\Utility\ExtensionManagementUtility::registerAjaxHandler(
				  'TxMyExt::process',
				  '\Vendor\MyExt\AjaxHandler->process'
				);
			\end{lstlisting}

		\item L'adresse pour un appel Ajax contient un jeton de protection CSRF qui sera vérifié dans le dispatcher \texttt{ajax.php}

			\begin{lstlisting}
				$ajaxUrl = \TYPO3\CMS\Core\Utility\BackendUtility::getAjaxUrl('TxMyExt::process');
			\end{lstlisting}

		\item Ces options sont ensuite accessibles dans le contexte JavaScript de la page

			\begin{lstlisting}
				var ajaxUrl = TYPO3.settings.MyExt.ajaxUrl;
			\end{lstlisting}

	\end{itemize}

\end{frame}

% ------------------------------------------------------------------------------
% Miscellaneous
% ------------------------------------------------------------------------------
% http://forge.typo3.org/issues/49144 (MathUtility: Add canBeInterpretedAsFloat)
% http://forge.typo3.org/issues/52707 (Introduce a PHP Enumeration type)
% http://forge.typo3.org/issues/52762 (Add type converter for core types like Enumeration)

\begin{frame}[fragile]
	\frametitle{Application Programming Interface}
	\framesubtitle{Divers}

	\begin{itemize}
		\item Nouvelle méthode \texttt{canBeInterpretedAsFloat()} dans la classe : \texttt{MathUtility}\newline
			(C'est un analogue à : \texttt{canBeInterpretedAsInteger()})
		\item Nouveau type d'énumération (sans relation à un module PHP tiers) :\newline
			\texttt{\textbackslash
				TYPO3\textbackslash
				CMS\textbackslash
				Core\textbackslash
				Type\textbackslash
				Enumeration}\newline

			Utilisé par exemple dans :\newline
			\texttt{\textbackslash
				TYPO3\textbackslash
				CMS\textbackslash
				Core\textbackslash
				Versioning\textbackslash
				VersionState}\newline

			...et donc ainsi :\newline
			\texttt{new VersionState(VersionState::DEFAULT\_STATE);}

	\end{itemize}

\end{frame}

% ------------------------------------------------------------------------------

