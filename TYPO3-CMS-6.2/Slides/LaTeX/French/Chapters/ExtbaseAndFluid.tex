% ------------------------------------------------------------------------------
% TYPO3 CMS 6.2 LTS - What's New - Chapter "Extbase & Fluid" (French Version)
%
% @author	Paul Blondiaux <pblondiaux@sodifrance.fr>
% @author	Philippe Herault <philippe.herault@plan-net.fr>
% @license	Creative Commons BY-NC-SA 3.0
% @link		http://typo3.org/download/release-notes/whats-new/
% @language	French
% ------------------------------------------------------------------------------
% Chapter: Extbase & Fluid
% ------------------------------------------------------------------------------

\section{Extbase \& Fluid}
\begin{frame}[fragile]
	\frametitle{Extbase \& Fluid}

	\begin{center}\huge{Chapitre 8 :}\end{center}
	\begin{center}\huge{\color{typo3darkgrey}\textbf{Extbase \& Fluid}}\end{center}

\end{frame}

% ------------------------------------------------------------------------------
% ObjectManager->getScope()
% ------------------------------------------------------------------------------
% http://forge.typo3.org/issues/48766

\begin{frame}[fragile]
	\frametitle{Extbase \& Fluid}
	\framesubtitle{ObjectManager->getScope()}

	\lstset{
		basicstyle=\tiny\ttfamily
	}

	\begin{itemize}
		\item La méthode \texttt{ObjectManager->getScope()} détermine si une classe est de type \textbf{prototype} ou \textbf{singleton}

			\begin{lstlisting}
				/**
				 * @var \TYPO3\CMS\Extbase\Object\ObjectManagerInterface
				 * @inject
				 */
				protected $objectManager;

				$this->objectManager->getScope($propertyTargetClassName) === \TYPO3\CMS
				\Extbase\Object\Container\Container::SCOPE_PROTOTYPE

				$this->objectManager->getScope($propertyTargetClassName) === \TYPO3\CMS
				\Extbase\Object\Container\Container::SCOPE_SINGLETON
			\end{lstlisting}

	\end{itemize}

\end{frame}

% ------------------------------------------------------------------------------
% Page type for URIs with format
% ------------------------------------------------------------------------------
% http://forge.typo3.org/issues/27498

\begin{frame}[fragile]
	\frametitle{Extbase \& Fluid}
	\framesubtitle{Type de page pour les URIs}

	\lstset{
		basicstyle=\tiny\ttfamily
	}

	\begin{itemize}
		\item Lors du rendu d'un format spécial, l'attribut personnalisé type de page n'est plus nécessaire

			\smaller\textbf{TYPO3 < 6.2 :}\normalsize
			\begin{lstlisting}
				<f:link.action arguments="{blog: blog}" pageType="{settings.plaintextPageType}"
				  format="txt">[plaintext]</f:link.action></li>
			\end{lstlisting}

		\item La nouvelle option TypoScript \texttt{formatToPageTypeMapping} permet une association globale :

			\begin{lstlisting}
				plugin.tx_myextension {
				  view.formatToPageTypeMapping {
				    txt = 99
				    pdf = 123
				  }
				}
			\end{lstlisting}

			\smaller\textbf{TYPO3 >= 6.2 :}\normalsize
			\begin{lstlisting}
				<f:link.action arguments="{blog: blog}"
				  format="txt">[plaintext]</f:link.action></li>
			\end{lstlisting}

	\end{itemize}

\end{frame}

% ------------------------------------------------------------------------------
% Object Type Converter
% ------------------------------------------------------------------------------
% http://forge.typo3.org/issues/48548

\begin{frame}[fragile]
	\frametitle{Extbase \& Fluid}
	\framesubtitle{Object Type Converter (1)}

	\begin{itemize}
		\item Associe des tableaux source à des objets non-persistant
		\item Utile lorsque l'on a besoin d'objets transitoires construits depuis les arguments de la requête
		\item Quelques exemples sur les diapositives suivantes...

	\end{itemize}

\end{frame}

% ------------------------------------------------------------------------------
% Object Type Converter
% ------------------------------------------------------------------------------
% http://forge.typo3.org/issues/48548

\begin{frame}[fragile]
	\frametitle{Extbase \& Fluid}
	\framesubtitle{Object Type Converter (2)}

	\lstset{
		basicstyle=\tiny\ttfamily
	}

	\smaller\textbf{Requête GET}\normalsize
	\begin{lstlisting}
		http://example.com/index.php?id=299
		  &tx_myextension[action]=list
		  &tx_myextension[controller]=Entity
		  &tx_myextension[demand][title]=foo
		  &tx_myextension[demand][relation]=1
	\end{lstlisting}

	\smaller\textbf{Entity controller : \texttt{initializeListAction()}}\normalsize
	\begin{lstlisting}
		use [Vendor]\myextension\Domain\Dto\Demand;
		public function initializeListAction() {
		  /**
		   * @var PropertyMappingConfiguration $demandConfiguration
		   */
		  $demandConfiguration = $this->arguments['demand']->getPropertyMappingConfiguration();
		  $demandConfiguration->allowAllProperties()->forProperty('relation')->allowAllProperties()->
		    setTypeConverterOption(
		      'TYPO3\\CMS\\Extbase\\Property\\TypeConverter\\PersistentObjectConverter',
		      PersistentObjectConverter::CONFIGURATION_CREATION_ALLOWED,
		      TRUE
		  );
		}
	\end{lstlisting}

\end{frame}

% ------------------------------------------------------------------------------
% Object Type Converter
% ------------------------------------------------------------------------------
% http://forge.typo3.org/issues/48548

\begin{frame}[fragile]
	\frametitle{Extbase \& Fluid}
	\framesubtitle{Object Type Converter (3)}

	\lstset{
		basicstyle=\tiny\ttfamily
	}

	\smaller\textbf{Entity controller : \texttt{listAction()}}\normalsize
	\begin{lstlisting}
		use [Vendor]\myextension\Domain\Dto\Demand;
		/**
		 * @var PropertyMappingConfiguration $demandConfiguration
		 */
		public function listAction(Demand $demand = NULL) {
		  $entities = $this->entityRepository->findAll();
		  $this->view->assign('entities', $entities);
		}
	\end{lstlisting}

	\smaller\textbf{Modèle : \texttt{[Vendor]\textbackslash myextension\textbackslash Domain\textbackslash Dto\textbackslash Demand.php}}\normalsize
	\begin{lstlisting}
		namespace [Vendor]\myextension\Domain\Dto;
		use [Vendor]\myextension\Domain\Model\Relation;
		class Demand {
		  protected $relation;
		  /**
		   * @param \TYPO3Friends\MapperExample\Domain\Model\Relation $relation
		   */
		  public function setRelation($relation) {
		    $this->relation = $relation;
		  }
		}
	\end{lstlisting}

\end{frame}

% ------------------------------------------------------------------------------
% Chaining of set* functions
% ------------------------------------------------------------------------------
% http://forge.typo3.org/issues/47684

\begin{frame}[fragile]
	\frametitle{Extbase \& Fluid}
	\framesubtitle{Enchaînement des fonctions set*}

	\lstset{
		basicstyle=\tiny\ttfamily
	}

	\begin{itemize}
		\item Les méthodes de manipulation \texttt{set*} peuvent maintenant être \emph{enchaînées} dans l'API QuerySettings
		\item Inclut de nouvelles options introduites par TYPO3 CMS 6.0 :\newline
			\texttt{setIncludeDeleted} et \texttt{setIgnoreEnableFields}

			\begin{lstlisting}
				$query->getQuerySettings()
				  ->setRespectStoragePage(FALSE)
				  ->setRespectSysLanguage(FALSE)
				  ->setIgnoreEnableFields(TRUE)
				  ->setIncludeDeleted(TRUE);
			\end{lstlisting}
	\end{itemize}

\end{frame}

% ------------------------------------------------------------------------------
% returnRawQueryResult as argument
% ------------------------------------------------------------------------------
% http://forge.typo3.org/issues/51145

\begin{frame}[fragile]
	\frametitle{Extbase \& Fluid}
	\framesubtitle{returnRawQueryResult en tant qu'argument}

	\lstset{
		basicstyle=\smaller\ttfamily
	}

	\begin{itemize}
		\item returnRawQueryResult n'est plus une configuration des requêtes,\newline
			mais un argument de la méthode : \texttt{execute()}
			\newline

			\smaller\textbf{TYPO3 < 6.2 :}\normalsize\newline
			\lstinline!$query->getQuerySettings()->setReturnRawQueryResult(TRUE);!
			\newline

			\smaller\textbf{TYPO3 >= 6.2 :}\normalsize\newline
			\lstinline!$query->execute(TRUE);!

	\end{itemize}

\end{frame}

% ------------------------------------------------------------------------------
% Recursive validation
% ------------------------------------------------------------------------------
% http://forge.typo3.org/issues/6893

\begin{frame}[fragile]
	\frametitle{Extbase \& Fluid}
	\framesubtitle{Validation récursive}

	\begin{itemize}
		\item Extbase utilise désormais la validation récursive (comme dans TYPO3 Flow)
		\item Cela signifie que lorsque des objets incorporés sont créés par le Property-Mapper,
			les objets dans les différentes propriétés sont validés comme l'objet englobant\newline
			(dans TYPO3 CMS < 6.2, seul l'objet englobant était validé)
		\item En outre, les validateurs autorisent désormais les valeurs vides
	\end{itemize}

	\breakingchange

	\smaller\begin{center} Afin de rendre obligatoire une propriété, vous \underline{devez} ajouter \textbf{NotEmptyValidator} explicitement !\end{center}\normalsize

\end{frame}

% ------------------------------------------------------------------------------
% Application Context
% ------------------------------------------------------------------------------
% http://forge.typo3.org/issues/49988

\begin{frame}[fragile]
	\frametitle{Extbase \& Fluid}
	\framesubtitle{Application Context}

	\begin{itemize}
		\item Accéder au contexte d'application actuel dans Extbase\newline
			(configuré par la variable d'environnement \texttt{TYPO3\_CONTEXT} ou dans l'Install Tool)\newline

			\lstinline!\TYPO3\CMS\Core\Core\Bootstrap::getInstance()->getContext();!
			\lstinline!\TYPO3\CMS\Core\Utility\GeneralUtility::getContext();!

	\end{itemize}

\end{frame}

% ------------------------------------------------------------------------------
% ViewHelper: Image
% ------------------------------------------------------------------------------
% http://forge.typo3.org/issues/47552

\begin{frame}[fragile]
	\frametitle{Extbase \& Fluid}
	\framesubtitle{ViewHelper : image}

	\begin{itemize}
		\item ViewHelper Fluid \textbf{Image} avec l'attribut \texttt{title} optionnel\newline

			\smaller\textbf{Exemple :}\normalsize\newline
			\lstinline!<f:image src="background.jpg" alt="Text" />!
			\newline

			\smaller\textbf{TYPO3 < 6.2 :}\normalsize\newline
			\lstinline!<img src="background.jpg" alt="Text" title="Text" />!
			\newline

			\smaller\textbf{TYPO3 >= 6.2 :}\normalsize\newline
			\lstinline!<img src="background.jpg" alt="Text" />!

	\end{itemize}

\end{frame}

% ------------------------------------------------------------------------------
% ViewHelpers: textfield and textarea
% ------------------------------------------------------------------------------
% http://forge.typo3.org/issues/45960
% http://forge.typo3.org/issues/48689 (Added autofocus attribute to textfield and textarea)

\begin{frame}[fragile]
	\frametitle{Extbase \& Fluid}
	\framesubtitle{ViewHelpers : textfield et textarea}

	\begin{itemize}
		\item Les arguments \texttt{autofocus} et \texttt{placeholder} (argument HTML5 valide) pour les ViewHelpers Fluid \textbf{form.textarea} et \textbf{form.textfield}\newline

			\smaller\textbf{Exemple (« placeholder ») :}\normalsize
			\begin{lstlisting}
				<f:form.textfield
				  id="powermail_field_{field.marker}"
				  ...
				  placeholder="{field.title -> vh:string.RawAndRemoveXss()}"
				  ...
				  name="field[{field.uid}]"
				  required="{field.mandatory}" />
			\end{lstlisting}

	\end{itemize}

\end{frame}

% ------------------------------------------------------------------------------
% ViewHelper: switch
% ------------------------------------------------------------------------------
% http://forge.typo3.org/issues/48653

\begin{frame}[fragile]
	\frametitle{Extbase \& Fluid}
	\framesubtitle{ViewHelper : switch}

	\lstset{
		basicstyle=\smaller\ttfamily
	}

	\begin{itemize}
		\item Nouveau ViewHelper Fluid \textbf{switch} générant le contenu suivant une valeur ou une expression donnée
		\item Se comporte comme l'énoncé \texttt{switch()} en PHP

			\begin{lstlisting}
				<f:switch expression="{person.gender}">
				  <f:case value="male">Mr.</f:case>
				  <f:case value="female">Mrs.</f:case>
				  <f:case default="TRUE">Mrs. or Mr.</f:case>
				</f:switch>
			\end{lstlisting}

		\item \textbf{\underline{Note :}} l'usage excessif de ce ViewHelper est l'indicateur d'une mauvaise conception ! L'exemple ci-dessus pourrait aussi être réalisé en utilisant les partials « \texttt{title.male.html} », « \texttt{title.female.html} » et ce qui suit :

			\begin{lstlisting}
				<f:render partial="title.{person.gender}" />
			\end{lstlisting}

	\end{itemize}

\end{frame}

% ------------------------------------------------------------------------------
% ViewHelper: fileSize
% ------------------------------------------------------------------------------
% http://forge.typo3.org/issues/49139

\begin{frame}[fragile]
	\frametitle{Extbase \& Fluid}
	\framesubtitle{ViewHelper : fileSize}

	\begin{itemize}
		\item Convertit la taille d'un fichier (entier) en chaîne lisible\newline
	\end{itemize}
	
	\begin{columns}[T]

		\begin{column}{.5\textwidth}
			\advance\leftskip+0.8cm
			\smaller
				\textbf{Exemple 1} (fileSize = 1263616):\newline
				\texttt{fileSize -> f:format.bytes()}\newline
				\newline
				Sortie : « 1234 KB »
			\normalsize
		\end{column}
		\begin{column}{.5\textwidth}
			\smaller
				\textbf{Exemple 2} (fileSize = 1263616):\newline
				\texttt{fileSize -> f:format.bytes(}\newline
				\texttt{
				decimals: 2,\newline
				decimalSeparator: '.',\newline
				thousandsSeparator: ','\newline
				)}\newline
				\newline
				Sortie : « 1,234.00 KB »
			\normalsize
		\end{column}

	\end{columns}

\end{frame}

% ------------------------------------------------------------------------------
% ViewHelper: format.date
% (slide added in March 2014)
% ------------------------------------------------------------------------------

\begin{frame}[fragile]
	\frametitle{Extbase \& Fluid}
	\framesubtitle{ViewHelper : format.date}

	\lstset{
		basicstyle=\smaller\ttfamily
	}

	\begin{itemize}
		\item La valeur par défaut du ViewHelper \textbf{format.date} est la valeur configurée dans l'Install Tool

			\lstinline!$GLOBALS['TYPO3_CONF_VARS']['SYS']['ddmmyy']!

		\item Si cette valeur n'est pas configurée, "\texttt{Y-m-d}" est utilisé (year, month, day)

	\end{itemize}

\end{frame}

% ------------------------------------------------------------------------------
% ViewHelper: Backend Container
% ------------------------------------------------------------------------------
% http://forge.typo3.org/issues/49749

\begin{frame}[fragile]
	\frametitle{Extbase \& Fluid}
	\framesubtitle{ViewHelper : Backend Container}

	\lstset{
		basicstyle=\smaller\ttfamily
	}

	\begin{itemize}
		\item Le ViewHelper Fluid backend container (\texttt{be.container}) retravaillé :\newline
			\smaller\texttt{typo3/sysext/fluid/Classes/ViewHelpers/Be/ContainerViewHelper.php}\normalsize\newline

			\smaller\textbf{Déprécié :}\normalsize
			\begin{itemize}
				\item \texttt{\$addCssFile} (remplacé par \texttt{\$includeCssFiles})
				\item \texttt{\$addJsFile} (remplacé par \texttt{\$includeJsFiles})
			\end{itemize}

			\smaller\textbf{Nouveau :}\normalsize
			\begin{itemize}
				\item \texttt{\$loadJQuery}
				\item \texttt{\$includeCssFiles}
				\item \texttt{\$includeJsFiles}
				\item \texttt{\$addJsInlineLabels}
			\end{itemize}

	\end{itemize}

\end{frame}

% ------------------------------------------------------------------------------
% ViewHelper: button.icon
% ------------------------------------------------------------------------------

\begin{frame}[fragile]
	\frametitle{Extbase \& Fluid}
	\framesubtitle{ViewHelper : button.icon}

	\lstset{
		basicstyle=\smaller\ttfamily
	}

	\begin{itemize}
		\item Le ViewHelper Fluid \textbf{button.icon} est finalisé (était « expérimental »)
		\item Crée une icône bouton (optionnellement avec un lien)

			\begin{lstlisting}
				<f:be.buttons.icon uri="{f:uri.action(action:'new')}"
				  icon="actions-document-new" title="Create new Foo" />

				<f:be.buttons.icon
				  icon="actions-document-new" title="Create new Foo" />
			\end{lstlisting}

		\item L'attribut \texttt{icon} accepte plus de 310 valeurs !\newline

			\smaller
				Rechercher :\newline
				\texttt{\$GLOBALS['TBE\_STYLES']['spriteIconApi']['coreSpriteImageNames']}\newline
				...dans le fichier :\newline
				\texttt{typo3/systext/core/ext\_tables.php}
			\normalsize

	\end{itemize}

\end{frame}

% ------------------------------------------------------------------------------
% Option addQueryStringMethod
% ------------------------------------------------------------------------------
% http://forge.typo3.org/issues/11441
% http://forge.typo3.org/issues/35281

\begin{frame}[fragile]
	\frametitle{Extbase \& Fluid}
	\framesubtitle{Option addQueryStringMethod (1)}

	\begin{itemize}
		\item L'option \texttt{addQueryString} supporte seulement les arguments \textbf{GET}\newline
			(qui sont ensuite ajoutés au lien généré)
		\item Les arguments \textbf{POST} (utilisés par les Widgets) ne fonctionnent pas avec cette option
		\item La nouvelle option \texttt{addQueryStringMethod} résout ce problème et permet de définir quelle méthode doit être prise en compte :\newline
			GET (par défaut), POST, GET/POST ou POST/GET

	\end{itemize}

\end{frame}

% ------------------------------------------------------------------------------
% Option addQueryStringMethod
% ------------------------------------------------------------------------------
% http://forge.typo3.org/issues/11441
% http://forge.typo3.org/issues/35281

\begin{frame}[fragile]
	\frametitle{Extbase \& Fluid}
	\framesubtitle{Option addQueryStringMethod (2)}

	\begin{itemize}
		\item Plusieurs ViewHelper Fluid supportent cette nouvelle option :

			\begin{itemize}\smaller
				\item \texttt{link.action}
				\item \texttt{link.page}
				\item \texttt{uri.action}
				\item \texttt{uri.page}
				\item \texttt{widget.link}
				\item \texttt{widget.uri}
				\item \texttt{widget.paginate}
			\end{itemize}

	\end{itemize}

\end{frame}

% ------------------------------------------------------------------------------
% Fluid: Fallback path for templates
% ------------------------------------------------------------------------------

\begin{frame}[fragile]
	\frametitle{Extbase \& Fluid}
	\framesubtitle{Fluid : Chemin alternatif pour les Template}

	\lstset{
		basicstyle=\tiny\ttfamily
	}

	\begin{itemize}
		\item Fluid supporte maintenant des chemins alternatifs pour les templates, partials et layouts :\newline
			\smaller\texttt{templateRootPaths}, \texttt{partialRootPaths}, \texttt{layoutRootPaths}\normalsize
		\item L'indice le plus élevé en premier, ensuite itère sur les indices inférieurs, jusqu'à ce qu'un template soit trouvé

			\begin{lstlisting}
				plugin.tx_myextension {
				  view {
				    templateRootPath = EXT:myextension/Resources/Private/Templates/
				  }
				}

				plugin.tx_myextension {
				  view {
				    templateRootPath >
				    templateRootPaths {
				      10 = fileadmin/myextension/Templates/
				      20 = EXT:myextension/Resources/Private/Templates/
				    }
				  }
				}
			\end{lstlisting}

	\end{itemize}

\end{frame}

% ------------------------------------------------------------------------------

