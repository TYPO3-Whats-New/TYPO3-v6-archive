% ------------------------------------------------------------------------------
% TYPO3 CMS 6.2 LTS - What's New - Chapter "Package Management" (French version)
%
% @author	Paul Blondiaux <pblondiaux@sodifrance.fr>
% @author	Philippe Herault <philippe.herault@plan-net.fr>
% @license	Creative Commons BY-NC-SA 3.0
% @link		http://typo3.org/download/release-notes/whats-new/
% @language	French
% ------------------------------------------------------------------------------
% Chapter: Package Management
% ------------------------------------------------------------------------------

\section{Package Management}
\begin{frame}[fragile]
	\frametitle{Gestion des Paquets}

	\begin{center}\huge{Chapitre 5 :}\end{center}
	\begin{center}\huge{\color{typo3darkgrey}\textbf{Gestion des paquets}}\end{center}

\end{frame}

% ------------------------------------------------------------------------------
% Package Manager
% ------------------------------------------------------------------------------
% http://wiki.typo3.org/Blueprints/Packagemanager
% http://forge.typo3.org/issues/47018
% http://forge.typo3.org/issues/52737

\begin{frame}[fragile]
	\frametitle{Gestion des paquets}
	\framesubtitle{Gestionnaire de paquets}

	\begin{itemize}
		\item Le \textbf{Package Manager} de TYPO3 Flow à été porté sur TYPO3 CMS
		\item Le développement et l'exploration de cette fonctionnalité ont débuté pendant le développement de la version TYPO3 CMS 6.1
		\item Ce projet vise à harmoniser les formats des paquets
		\item Les extensions dans TYPO3 CMS sont juste un type particulier de \newline
				« paquets »
		\item Objectifs principaux du projet :

			\begin{itemize}
				\item Une API ad hoc pour la gestion des paquets
				\item Un support « Vendor Namespace »
				\item Un support pour les « Composer Package »
				\item Un support pour les « Flow Package »
				\item Réécriture de l'autoloader
			\end{itemize}

	\end{itemize}

\end{frame}

% ------------------------------------------------------------------------------
% Package Manager Integration
% ------------------------------------------------------------------------------

\begin{frame}[fragile]
	\frametitle{Gestion des paquets}
	\framesubtitle{Intégration du gestionnaire de paquets (1)}

	\begin{itemize}
		\item Retrait de \texttt{\$TYPO3\_CONF['EXT']['extListArray']} du fichier\newline
			\smaller\texttt{typo3conf/LocalConfiguration.php}\normalsize

		\item L'ancien contenu du fichier \small\texttt{typo3conf/LocalConfiguration.php} a été copié dans \normalsize\newline
			\smaller\texttt{typo3conf/LocalConfiguration.beforePackageStatesMigration.php}\normalsize

		\item Le fichier \texttt{typo3conf/PackageStates.php} contient :

			\begin{itemize}
				\item le statut du paquet (actif/inactif)
				\item l'emplacement physique de l'extension
			\end{itemize}

		\item Les extensions placées dans les dossiers suivants sont automatiquement détectées :
			\begin{itemize}
				\item \texttt{typo3/sysext/}
				\item \texttt{typo3/ext/}
				\item \texttt{typo3/contrib/}
				\item \texttt{typo3conf/ext/}
				\item \texttt{Packages/} \emph{(recursif)}
			\end{itemize}

	\end{itemize}

\end{frame}

% ------------------------------------------------------------------------------
% Package Manager Integration
% ------------------------------------------------------------------------------

\begin{frame}[fragile]
	\frametitle{Gestion des paquets}
	\framesubtitle{Intégration du gestionnaire de paquets (2)}

	\begin{itemize}

		\item Deux nouveaux fichiers dans le répertoire de l'extension :

			\begin{itemize}
				\item \texttt{composer.json}
				\item \texttt{Classes/Package.php}
			\end{itemize}

		\item Si l'extension est requise, l'indicateur \texttt{protected} doit être défini dans le fichier \texttt{composer.json}

		\item Si le fichier \texttt{PackageStates.php} est manquant, il sera automatiquement (re)créé avec la liste de toutes les extensions qui ont la propriété ci-dessus à \texttt{TRUE}

		\item L'Autoloader a son propre cache Backend

		\item En savoir plus :\newline
			\url{http://wiki.typo3.org/Blueprints/Packagemanager}

	\end{itemize}

\end{frame}

% ------------------------------------------------------------------------------
% Examples
% ------------------------------------------------------------------------------

\begin{frame}[fragile]
	\frametitle{Gestion des paquets}
	\framesubtitle{Intégration du gestionnaire de paquets (3)}

	Exemple : \texttt{typo3conf/PackageManager.php}

	\lstset{
		basicstyle=\tiny\ttfamily
		% basicstyle=\fontsize{5}{7}\selectfont\ttfamily
	}

	\begin{lstlisting}
		return array ('packages' =>
		    array (
		      'core' =>
		        array (
		          'manifestPath' => '',
		          'composerName' => 'typo3/cms/core',
		          'state' => 'active',
		          'packagePath' => 'typo3/sysext/core/',
		          'classesPath' => 'Classes/',
		        ),
		      'workspaces' =>
		        array (
		          'manifestPath' => '',
		          'composerName' => 'typo3/cms/workspaces',
		          'state' => 'inactive',
		          'packagePath' => 'typo3/sysext/workspaces/',
		          'classesPath' => 'Classes/',
		        ),
		      ...
		    ),
		    'version' => 4,
		);
	\end{lstlisting}

\end{frame}

% ------------------------------------------------------------------------------
% Examples
% ------------------------------------------------------------------------------

\begin{frame}[fragile]
	\frametitle{Gestion des paquets}
	\framesubtitle{Intégration du gestionnaire de paquets (4)}

	Exemple : \texttt{composer.json}

	\lstset{
		basicstyle=\tiny\ttfamily
	}

	\begin{lstlisting}
		{
		  "name": "typo3/cms-indexed-search",
		  "type": "typo3-cms-framework",
		  "description": "TYPO3 Core",
		  "homepage": "http://typo3.org",
		  "license": ["GPL-2.0+"],
		  "version": "6.2.0",
		  "require": {
		    "typo3/cms-core": "*"
		  },
		  "replace": {
		    "indexed_search": "*"
		  }
		}
	\end{lstlisting}

\end{frame}


% ------------------------------------------------------------------------------
% Miscellaneous
% (slide added in March 2014)
% ------------------------------------------------------------------------------

\begin{frame}[fragile]
	\frametitle{Gestion des paquets}
	\framesubtitle{Intégration du gestionnaire de paquets (6)}

	\lstset{
		basicstyle=\smaller\ttfamily
	}

	\begin{itemize}
		\item Les paquets peuvent aussi être activés au cours du process grâce à l'instruction :
			\smaller\texttt{\$GLOBALS['TYPO3\_CONF\_VARS']['EXT']['runtimeActivatedPackages'] = array(}\space\textit{packageKey}\space\texttt{);}\normalsize

		\item Cette instruction est activée immédiatement après l'initialisation de la gestion de Packages

	\end{itemize}

\end{frame}

% ------------------------------------------------------------------------------
