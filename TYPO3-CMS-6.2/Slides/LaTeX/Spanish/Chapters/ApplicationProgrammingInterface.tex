% ------------------------------------------------------------------------------
% TYPO3 CMS 6.2 LTS - What's New - Chapter "Application Programming Interface" (Spanish Version)
%
% @author	Sergio Catalá <sergio.catala@e-net.info>
% @author	Michel Mix <mmix@autistici.org>
% @license	Creative Commons BY-NC-SA 3.0
% @link		http://typo3.org/download/release-notes/whats-new/
% @language	Spanish
% ------------------------------------------------------------------------------
% Chapter: Application Programming Interface
% ------------------------------------------------------------------------------

\section{Interfaz de Programación de Aplicaciones}
\begin{frame}[fragile]
	\frametitle{Interfaz de Programación de Aplicaciones}

	\begin{center}\huge{Capítulo 7:}\end{center}
	\begin{center}\huge{\color{typo3darkgrey}\textbf{Interfaz de Programación de Aplicaciones (API)}}\end{center}

\end{frame}

% ------------------------------------------------------------------------------
% Hook: tsfe::checkEnableFields
% ------------------------------------------------------------------------------
% http://forge.typo3.org/issues/48981

\begin{frame}[fragile]
	\frametitle{Interfaz de Programación de Aplicaciones}
	\framesubtitle{Hook: \texttt{tsfe::checkEnableFields}}

	\begin{itemize}
		\item En TYPO3 < 6.2, "\emph{extender a las subpáginas}" no se puede utilizar en las propias extensiones que proporcionan reglas adicionales para la visibilidad de la página\newline
			\small(la lista de campos para comprobar está programada a bajo nivel en \texttt{tsfe::checkEnableFields()})\normalsize

		\item En TYPO3 >= 6.2, un nuevo hook permite extensiones para proporcionar reglas adicionales para la visibilidad de la página cuando las páginas de encima tienen "extender a las subpáginas" activada
		\item Clase:\newline
			\smaller
				\texttt{\textbackslash
					TYPO3\textbackslash
					CMS\textbackslash
					Frontend\textbackslash
					Controller\textbackslash
					TypoScriptFrontendController}\normalsize

			\lstset{
				basicstyle=\smaller\ttfamily
			}

			\begin{lstlisting}
				$GLOBALS['TYPO3_CONF_VARS']['SC_OPTIONS']
				  ['tslib/class.tslib_fe.php']['hook_checkEnableFields']
			\end{lstlisting}

	\end{itemize}

\end{frame}

% ------------------------------------------------------------------------------
% Hook: checkFlexFormValue in DataHandler
% ------------------------------------------------------------------------------
% http://forge.typo3.org/issues/49699

\begin{frame}[fragile]
	\frametitle{Interfaz de Programación de Aplicaciones}
	\framesubtitle{Hook: \texttt{checkFlexFormValue} en DataHandler (1)}

	\begin{itemize}
		\item En TYPO3 < 6.2, al actualizar los valores de Flexform, no hay ninguna verificación si en realidad se ha eliminado un valor existente en la base de datos
		\item Esto se convirtió en un problema, p. ej. al guardar las acciones del controlador conmutables (Extbase) en el Flexform: acciones antiguas que tal vez ya no están presentes tienen que ser eliminadas manualmente

		\item Desde TYPO3 >= 6.2, un nuevo hook permite ajustar los datos anteriores del Flexform justo antes de que se combine con el nuevo
	\end{itemize}

\end{frame}

% ------------------------------------------------------------------------------
% Hook: checkFlexFormValue in DataHandler
% ------------------------------------------------------------------------------
% http://forge.typo3.org/issues/49699

\begin{frame}[fragile]
	\frametitle{Interfaz de Programación de Aplicaciones}
	\framesubtitle{Hook: \texttt{checkFlexFormValue} en DataHandler (2)}

	\begin{itemize}
		\item Clase:\newline
			\smaller
				\texttt{\textbackslash
					TYPO3\textbackslash
					CMS\textbackslash
					Core\textbackslash
					DataHandling\textbackslash
					DataHandler}\normalsize

			\lstset{
				basicstyle=\smaller\ttfamily
			}

			\begin{lstlisting}
				$GLOBALS['TYPO3_CONF_VARS']['SC_OPTIONS']
				  ['t3lib/class.t3lib_tcemain.php']['checkFlexFormValue']
			\end{lstlisting}

		\item Método:\newline
			\smaller
				\texttt{checkFlexFormValue\_beforeMerge()}

	\end{itemize}

\end{frame}

% ------------------------------------------------------------------------------
% Hook to customize header in module "Web > Page"
% ------------------------------------------------------------------------------
% http://forge.typo3.org/issues/52579

\begin{frame}[fragile]
	\frametitle{Interfaz de Programación de Aplicaciones}
	\framesubtitle{Hook para personalizar la cabecera}

	\begin{itemize}
		\item En TYPO3 >= 6.2, un nuevo hook permite ajustar la cabecera de una página en el módulo página (Módulo: "Web > Página")
		\item Este hook se llama antes de generar el contenido de la página
		\item Clase:\newline
			\smaller
				\texttt{\textbackslash
					TYPO3\textbackslash
					CMS\textbackslash
					Backend\textbackslash
					Controller\textbackslash
					PageLayoutController}\normalsize

			\lstset{
				basicstyle=\smaller\ttfamily
			}

			\begin{lstlisting}
				$GLOBALS['TYPO3_CONF_VARS']['SC_OPTIONS']
				  ['cms/layout/db_layout.php']['drawHeaderHook']
			\end{lstlisting}

		\item Método:\newline
			\smaller
				\texttt{callUserFunction()}

	\end{itemize}

\end{frame}

% ------------------------------------------------------------------------------
% IRRE: Provide default values for created records
% ------------------------------------------------------------------------------
% http://forge.typo3.org/issues/46124

\begin{frame}[fragile]
	\frametitle{Interfaz de Programación de Aplicaciones}
	\framesubtitle{IRRE: valores predeterminados para los registros creados}

	\begin{itemize}
		\item Nueva opción de TCA permite configurar los campos "inline"
		\item Clave \texttt{foreign\_record\_defaults} permite para ajustar los valores (predeterminados) en nuevos registros creados

			\begin{lstlisting}
				config => array(
				  'type' => 'inline',
				  'foreign_table' => 'tt_content',
				  'foreign_record_defaults' => array(
				    'CType' => 'image'
				  ),
				)
			\end{lstlisting}

			\small
				Ejemplo de arriba: elementos \texttt{tt\_content} que se crean para este campo IRRE serán \textbf{elementos de contenido imagen} por defecto. El editor puede cambiar esto a otro tipo antes de guardar.
			\normalsize

	\end{itemize}

\end{frame}

% ------------------------------------------------------------------------------
% Workspaces
% ------------------------------------------------------------------------------
% http://forge.typo3.org/issues/46124

\begin{frame}[fragile]
	\frametitle{Interfaz de Programación de Aplicaciones}
	\framesubtitle{Espacios de trabajo (1)}

	\begin{itemize}
		\item En TYPO3 < 6.2, el módulo "Espacios de trabajo" sólo puede ser extendido reemplazando los componentes PHP y JavaScript
		\item En TYPO3 >= 6.2, es posible extender ahora la definición y el comportamiento de las columnas mostradas en el módulo
		\item Algunos ejemplos en las siguientes diapositivas...
	\end{itemize}

\end{frame}

% ------------------------------------------------------------------------------
% Workspaces
% ------------------------------------------------------------------------------
% http://forge.typo3.org/issues/46124

\begin{frame}[fragile]
	\frametitle{Interfaz de Programación de Aplicaciones}
	\framesubtitle{Espacios de trabajo (2)}

	\lstset{
		basicstyle=\tiny\ttfamily
	}

	Ejemplo (archivo \texttt{ext\_localconf.php}):

	\begin{lstlisting}
		$GLOBALS['TYPO3_CONF_VARS']['SC_OPTIONS']
		  ['t3lib/class.t3lib_tcemain.php']['processCmdmapClass']['workspaces_logger'] =
		  'Vendor\\WorkspacesLogger\\Hook\\DataHandlerHook';
	\end{lstlisting}

	Ejemplo (archivo \texttt{ext\_tables.php}):

	\begin{lstlisting}
		\TYPO3\CMS\Workspaces\Service\AdditionalColumnService::getInstance()->register(
		  'WorkspacesLogger_StageChange',
		  'Vendor\\WorkspacesLogger\\DataProvider'
		);

		\TYPO3\CMS\Workspaces\Service\AdditionalResourceService::getInstance()->addJavaScriptResource(
		  'WorkspacesLogger',
		  'EXT:myextension/Resources/Public/JavaScript/StageChange.js'
		);
	\end{lstlisting}

\end{frame}

% ------------------------------------------------------------------------------
% Workspaces
% ------------------------------------------------------------------------------
% http://forge.typo3.org/issues/46124

\begin{frame}[fragile]
	\frametitle{Interfaz de Programación de Aplicaciones}
	\framesubtitle{Espacios de trabajo (3)}

	Ejemplo (archivo \texttt{Vendor\textbackslash
		WorkspacesLogger\textbackslash
		Hook\textbackslash
		DataHandlerHook}):

	\lstset{
		basicstyle=\tiny\ttfamily
	}

	\begin{lstlisting}
		<?php
		namespace Vendor\WorkspacesLogger\Hook;
		use TYPO3\CMS\Core\SingletonInterface;

		class DataHandlerHook implements SingletonInterface {

		  const TABLE_Name = 'tx_workspaceslogger_event';
		  const EVENT_SetStage = 91;

		  /**
		   * hook that is called when no prepared command was found
		   */
		  public function processCmdmap($command, $table, $id, $value, &$commandIsProcessed,
		    \TYPO3\CMS\Core\DataHandling\DataHandler $tcemainObj) {
		    ...
		    $action = (string) $value['action'];
		    if ($command === 'version' && $action === 'setStage' && $commandIsProcessed) {
		      ...
		    }
		  }
		}
	\end{lstlisting}

\end{frame}

% ------------------------------------------------------------------------------
% PSR-3 compatible Logger
% ------------------------------------------------------------------------------
% http://forge.typo3.org/issues/48880

\begin{frame}[fragile]
	\frametitle{Interfaz de Programación de Aplicaciones}
	\framesubtitle{Historial PSR-3 compatible}

	\begin{itemize}
		\item Ahora el historial de la API de TYPO3 CMS 6.2 es compatible con PSR-3
		\item PSR-3 tiene como objetivo establecer un estándar para el historial en PHP (estándar del PHP Framework Interop Group)

		\item El objetivo principal del PSR-3 es
			"\emph{permitir que las bibliotecas reciban un objeto LoggerInterface y escribir en él los historiales de una forma simple y universal.}"

		\item La interfaz del historial contiene métodos sencillos de historial tales como\newline
			\texttt{debug()}, \texttt{warning()}, \texttt{notice()}, \texttt{alert()}, \texttt{error()}, etc.

		\item Recursos adicionales:
			\begin{itemize}
				\item \url{http://www.php-fig.org/psr/3/}
			\end{itemize}

	\end{itemize}

\end{frame}

% ------------------------------------------------------------------------------
% API to CSRF protect Ajax calls in Backend
% (slide added in March 2014)
% ------------------------------------------------------------------------------
% http://forge.typo3.org/issues/56345
% http://forge.typo3.org/issues/56347 (documentation)

\begin{frame}[fragile]
	\frametitle{Interfaz de Programación de Aplicaciones}
	\framesubtitle{Llamadas Ajax Protegidas CSRF}

	\lstset{
		basicstyle=\tiny\ttfamily
	}

	\begin{itemize}
		\item Las llamadas Ajax en el backend de TYPO3 pueden ser protegidas contra CSRF (\textit{cross-site request forgery}) registrando sus manejadores

			\begin{lstlisting}
				\TYPO3\CMS\Core\Utility\ExtensionManagementUtility::registerAjaxHandler(
				  'TxMyExt::process',
				  '\Vendor\MyExt\AjaxHandler->process'
				);
			\end{lstlisting}

		\item URL para un Ajax ID determinado contiene un token de protección CSRF, que será chequeado en el despachador \texttt{ajax.php}

			\begin{lstlisting}
				$ajaxUrl = \TYPO3\CMS\Core\Utility\BackendUtility::getAjaxUrl('TxMyExt::process');
			\end{lstlisting}

		\item Pueden accederse a estos ajustes en el contexto JavaScript de la página

			\begin{lstlisting}
				var ajaxUrl = TYPO3.settings.MyExt.ajaxUrl;
			\end{lstlisting}

	\end{itemize}

\end{frame}

% ------------------------------------------------------------------------------
% Miscellaneous
% ------------------------------------------------------------------------------
% http://forge.typo3.org/issues/49144 (MathUtility: Add canBeInterpretedAsFloat)
% http://forge.typo3.org/issues/52707 (Introduce a PHP Enumeration type)
% http://forge.typo3.org/issues/52762 (Add type converter for core types like Enumeration)

\begin{frame}[fragile]
	\frametitle{Interfaz de Programación de Aplicaciones}
	\framesubtitle{Varios}

	\begin{itemize}
		\item Nuevo método \texttt{canBeInterpretedAsFloat()} en clase: \texttt{MathUtility}\newline
			\small(Esto es un análogo de: \texttt{canBeInterpretedAsInteger()})\normalsize
		\item Nuevo tipo de enumeración (sin una relación con módulos PHP externos):\newline
			\texttt{\textbackslash
				TYPO3\textbackslash
				CMS\textbackslash
				Core\textbackslash
				Type\textbackslash
				Enumeration}\newline

			Por ejemplo utilizado en:\newline
			\texttt{\textbackslash
				TYPO3\textbackslash
				CMS\textbackslash
				Core\textbackslash
				Versioning\textbackslash
				VersionState}\newline

			... y luego como:\newline
			\texttt{new VersionState(VersionState::DEFAULT\_STATE);}

	\end{itemize}

\end{frame}

% ------------------------------------------------------------------------------

